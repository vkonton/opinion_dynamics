\section{Lower bound for no-regret Dynamics}\label{app:s:lower_bound}

In the following Lemma we show how we can use algorithm $A$ to construct an
estimator $\hat{\theta_A}$ for Bernoulli distributions.
We restate \ref{l:reduction} for completeness.
\begin{theorem}

  Let the no-regret algorithm $A$ such that
  for all instances $I$, $\lim \limits_{t \rightarrow \infty} t^{1+c}
  \Expnew{I}{\norm{\infty}{x_A(t)-x^*}}=0$.  Then there exists an estimator
  $\hat{\theta_A}$ such that for all
  $p \in [0,1]$, \[\lim_{t \rightarrow \infty}t^{1+c}R_p(t)=0\]

\end{theorem}

\begin{proof}
  At first we remind that an estimator $\hat{\theta}$ is a sequence
  of functions $\{\hat{\theta_t}\}_{t=1}^{\infty}$, where
  $\theta_t:~\{0,1\}^t\mapsto [0,1]$.  We construct such a sequence using the
  algorithm $A$.  We also remind that when an agent $i$ runs algorithm $A$, she
  selects $x_i(t)$ according to the cost functions $\{C_i^{\tau}\}$ that she
  has already received \[x_i(t)=A_t(C_i^1,\ldots,C_i^{t-1})\]

  Consider an agent $i$ with $a_i=1$ and $s_i=0$ that runs $A$. Then
  $C_i^t(x)=x^2$ for all $t$ and $x_i(t)=A_t(x^2,\ldots,x^2)$.  The latter
  means that $x_i(t)$ only depends on $t$, $x_i(t)=h_0(t)$. Equivalently, if
  $a_i=1$ and $s_i=1$ then $x_i(t)=A_t((1-x)^2,\ldots,(1-x)^2)$ and
  $x_i(t)=h_1(t)$. Finally, consider an agent $i$ with $a_i=1/2$ and $s_i=0$.
  In this case $C_i^t = \frac{1}{2}x^2 + \frac{1}{2}(x-y_t)^2$, where $y_t \in
  [0,1]$ is the opinion of the neighbor $j\in N_i$ that $i$ met at round $t$.
  As a result, $x_i(t)=A_t(\frac{1}{2}x^2+\frac{1}{2}(x-y_1)^2,\ldots,
  \frac{1}{2}x^2+\frac{1}{2}(x-y_{t-1})^2)=f_t(y_1,\ldots,y_{t-1})$.  The
  estimator $\hat{\theta_A}$ is the following sequences
  $\{\hat{\theta_{t}}\}_{t=1}^{\infty}$
  \[
    \hat{\theta_t}(Y_1,\ldots,Y_t) =
    2f_{t+1}(h_{Y_1}(1),\ldots,h_{Y_t}(t))
  \]
  Observe that $\hat{\theta_t}: \{0,1\}^t \mapsto [0,1]$ meaning that
  $\hat{\theta_A}$ is a valid estimator for Bernoulli distributions.
  Now for any $p \in [0,1]$, we construct an appropriate instance
  $I_p$ s.t. $R_p(t)=\Expnew{p}{|\hat{\theta_t} - p|} \leq
  2\Expnew{I_p}{\norm{\infty}{x^t-x^*}}$.  
  Consider the graph of Figure~\ref{fig:lb_instance}, which has a central node 
  with $a_c = 1/2$ and $s_c = 0$ and two leaf nodes $1,2$ with 
  $a_1 = a_2 = 1$, $s_1 = 1$ and $s_2 = 0$. The weights are $p_{c1} = p$ 
  and $ p_{c2} = 1-p$. Obviously, nodes $1$ and $2$ always have constant 
  opinions, $1$ and $0$ respectively. Hence, in each round the center node 
  receives either $h_1(1)$ with probability $p$ or $h_2(0)$ with probability $1-p$. 

  We just need to prove that in $I_p$, $\Expnew{p}{|\hat{\theta_t}-p|} \leq
  \frac{1}{2}\Expnew{I_p}{\norm{\infty}{x^t-x^*}}$. Notice that $x^*_c=\frac{p}{2}$ and
  $x^*_i=s_i$ if $i\neq c$. .\\ At round $t$, if the oracle returns to the
  center agent the value $h_1(t)$ of agent-$1$, then $Y_t=1$ otherwise
  $Y_t=0$. As a result, $\Prob{Y_t=1}=p$ and

  \begin{align*}
    \Expnew{I_p}{\norm{\infty}{x^t-x^*}}
    &\geq
    \Expnew{I_p}{\lp|x^t_c-x^*_c\rp|}\\
    &= \Expnew{I_p}{\lp| f_{t+1}(h_{Y_1}(1),\ldots,h_{Y_t}(t)) - \frac{p}{2}\rp|}\\
    &=
    \Expnew{p}{\lp|\frac{\widehat{\theta_t}}{2}-\frac{p}{2}\rp|} = \frac{1}{2}R_p(t)
  \end{align*} 
  and the result follows.
  \end{proof}

We next give a rigorous measure-theoretic proof of Theorem~\ref{t:lower_bound}.
\begin{theorem}
  Let $\theta_t: \{0,1\}^t \to [0,1]$ be a sequence of estimators for
  the success probability $p$ of a Bernoulli random variable with
  distribution $P$.
  There exists $p \in [0,1]$ such that
  \[
   \lim_{t\to \infty} t^2\ \Expnew{X \sim P^t}{\lp| \theta_t(X) - p \rp|} > 0.
  \]
\end{theorem}
\begin{proof}
  Observe that $\theta_t(\{0,1\}^t)$ has cardinality at most $2^t$.
  Since
  \[
    \sum_{0 \leq i \leq t} \sum_{\norm{1}{x} = i}
    \lp| \theta_t(x) - p \rp| p^i (1-p)^{t-i}
    \geq
    \sum_{0 \leq i \leq t} \binom{t}{i} \lp|
     \frac{\sum_{\norm{1}{x} = i} \theta_t(x)}{\binom{t}{i}}  - p \rp|
    p^i (1-p)^{t-i}.
  \]
  Thus, without loss of generality we assume that
  $\theta_t(\{0,1\}^t)$ contains at most $t+1$ discrete
  points.

  In the following, we work in the measure space
  $(\R, \mcal{M}, \mu)$, where $\mu$ is the Lebesgue measure,
  and $\mcal{M}$ is the $\sigma$-algebra of the Lebesgue measurable
  sets.
  Suppose that there exists no such $p \in [0,1]$.
  Let
  \[
    A =
    \{
    p \in [0,1]\ :\ \lim_{t\to \infty}
    t^2\ \Expnew{X \sim P^t}{\lp| \theta_t(X) - p \rp|} = 0
    \}.
  \]
  Then $A = [0,1]$ and $A$ is measurable as an interval.
  Notice that,
  \[
    A \subseteq \bigcup_{t=1}^{\infty} \bigcap_{k=t}^{\infty} A_k,
  \]
  where \( A_k = \{ p\in[0,1]\ :\ R_k(p) < 1/2 \} \),
  and \(R_k(p) = k^2\ \Expnew{X \sim P^k}{| \theta_k(X) - p|}\).
  We have that $R_k: [0,1] \to [0, +\infty) $ is polynomial of
  degree $t$ in $p$ and therefore it is a measurable function.
  Thus, $A_k$ is measurable.
  We now show that
  \[ A_k \subseteq B_k \coloneqq \{ p \in [0,1]\ : \
    k^2 \min_{0 \leq i \leq k} \lp| \theta_k(i) - p \rp| < 1 \}.
  \]
  We prove this by contradiction. Suppose that $p \in A_k$ but
  $p \notin B_k$. Since $p \in A_k$ we have that
  \begin{align*}
    R_k(p)
    =
    k^2 \sum_{i=0}^k \binom{k}{i} \lp| \theta_k(i) - p \rp| p^i (1-p)^{k-i}
    \geq
    k^2 \min_{0 \leq i \leq k} \lp| \theta_k(i) - p \rp|
    \sum_{i=0}^k \binom{k}{i} p^i (1-p)^{k-i}
    \geq 1.
  \end{align*}
  Since the functions $p \mapsto k^2 \lp| \theta_k(i) - p \rp|$ are
  measurable, their pointwise minimum is measurable and therefore
  the sets $B_k$ are also measurable.  We next proceed to bound
  $\mu(B_k)$. Since $\theta_k$ can only take $k$ different values
  we have that there exist $k+1$ intervals $(a_{k_i}, b_{k_i})$
  of length at most $2/k^2$ such that
  \(
    B_k = \bigcup_{i=0}^k \lp(a_{k_i}, b_{k_i} \rp).
  \)
  Since $\mu$ is subadditive we have
  \[
    \mu(B_k) \leq \sum_{i=0}^{k} \frac{2}{k^2} = \frac{2 (k+1)}{k^2}.  \]
  Now observe that
  \begin{align*}
    \mu(A)
    \leq \mu \lp( \bigcup_{t=1}^{\infty} \bigcap_{k=t}^{\infty} A_k \rp)
    \leq \mu \lp( \bigcup_{t=1}^{\infty} \bigcap_{k=t}^{\infty} B_k \rp)
    \leq \sum_{t = 1}^{\infty} \mu\lp(\bigcap_{k=t}^{\infty} B_k \rp)
    \leq \sum_{t = 1}^{\infty} \lim_{k \to \infty} \mu(B_k)
    = 0
  \end{align*}
  Which is a contradiction since we assumed $A = [0,1]$.
\end{proof}
