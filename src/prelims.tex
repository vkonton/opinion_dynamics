\section{Introduction}

\subsection{Friedkin-Johsen Model}
In the Friedkin-Johsen model (FJ-model) an undirected graph $G(V,E)$ with $n$
nodes, is assumed, where $V$ denotes the agents and $E$ the social relations between
them. Each agent $i$ poses an internal opinion $s_i \in [0,1]$ and a self confidence coefficient
$\alpha_i \in[0,1]$. At each round $t\geq 1$, agent $i$ updates her opinion
as follows:

\begin{equation}\label{eq:best_response}
  x_i(t) = (1-\alpha_i) \frac{\sum_{j \in N_i}x_j(t-1)}{|N_i|} + \alpha_i s_i,
\end{equation}
where $N_i$ is the set of her neighbors. The simplicity of its update rule makes it plausible, because in
real social networks it is very unlikely that agents change their opinions
according to complex rules. This model has been studies from several perspectives.
Based on the above model, in \cite{BKO11}, they propose a game
in which the strategy that each agent $i$ plays, is the opinion $x_i \in [0,1]$
that she publicly expresses incurring her a cost
\begin{align}\label{eq:kleinberg_cost}
  C_i(x) = (1-a_i) \sum_{j \in N_i} (x_i - x_j)^2 + a_i (x_i - s_i)^2.
\end{align}
They prove that this game admits a unique equilibrium point and that
the price of anarchy is $9/8$ for undirected graphs and $O(n)$ for
directed networks.  In \cite{GS14} they study they show that the update
rule (\ref{eq:best_response}) is the \emph{best response} play for the above
game and they also study its convergence properties to the equilibrium point
$x^* \in [0,1]^n$. Therefore, Friedkin-Johsen model admits several nice properties
as it very simple to execute by the agents, rapidly converges to its equilibrium,
and it is a natural behavior for selfish agents.

Our work is motivated by the fact that the FJ-model requires that
agents interact with all their neighbors at each round. This requirement
could be under discussion in today's large social networks, where each individual
could have several hundreds of friends (neighbors). In such graphs it is quite
unnatural to assume that everybody updates her opinion using the opinions of all
her friends. In this work we propose simple and natural models, similar to
the original FJ-model, that require minimal interaction between agents at each
round with similar convergence properties. More precisely, we assume that
each agent can only interact with only one of her neighbors at each round.
To make our setting precise we define the following oracles.
\begin{itemize}
  \item The \emph{labeled} oracle $O^l_i(t)$: when
    agent $i$ calls the oracle at round $t$ the oracle $O^l_i(t)$ picks one
    of her neighbors $j$ uniformly at random and returns $(j, x_j(t-1))$,
    namely the label and the opinion of her neighbor $j$.

  \item The \emph{unlabeled} oracle $O^u_i(t)$: when
    agent $i$ calls the oracle $O^u_i(t)$ at round $t$ the oracle picks one
    of her neighbors $j$ uniformly at random and returns $x_j(t-1)$.
\end{itemize}
In the \emph{labeled} resp. \emph{unlabeled} setting, at the beginning of each
round $t$, each agent $i$ calls $O^l(i)$ resp. $O^u(i)$.
We motivate the study of the unlabeled setting by the fact that
in a huge social network, natural models would not take into account
the identities of the people that express an opinion. (CAREFUL OPINION LEADERS).
For the labeled setting we propose the following natural update rule.

In the unlabeled setting we consider the following update rule that closely
resembles the original FJ-model
\begin{align}
  x_i(0) &= s_i \nonumber \\
  x_i(t) &=
  (1-\alpha_i)\frac{\sum_{\tau=1}^{t} O^u_i(\tau)}{t} + \alpha_i s_i.
\label{eq:unlabeled_update_rule}
\end{align}
In the labeled setting each agent $i$ keeps a vector $M_i \in [0,1]^{|N_i|}$
with the opinions of all her neighbors. At each round $t$ she receives the
value $(j, x_j(t-1))$ from $O_i^l(t)$ and sets $M_i(j) = x_j(t-1)$. Initially
all agents have $M_i = 0$. Her update rule is
\begin{align}
  x_i(0) &= s_i \nonumber \\
  x_i(t) &=  (1-a_i) \frac{\sum_{j\in N_i} M_i(j)}{|N_i|} + a_i s_i
  \label{eq:labeled_update_rule}
\end{align}

\subsection{Our Results}
We first show that in the unlabeled setting, using the update rule
(\ref{eq:unlabeled_update_rule}) gives a protocol that converges to the
same equilibrium point as the FJ-model. In the next theorem we provide
the rate of convergence of our model.
\begin{theorem}
  For any instance $I = (G, s, a)$ the update rule \ref{eq:unlabeled_update_rule}
  has the following convergence rate
\[
\Exp{\norm{\infty}{x^t - x^*}} \leq
    \begin{cases}
      C_1 \sqrt{\log n}\frac{(\log t)^{2}}{t^\rho}
      &\quad\text{if } \rho \leq 1/2\\
      C_2 \sqrt{\log n}\frac{(\log t)^{2}}{t^{1/2}}
      &\quad\text{if } \rho > 1/2\\
    \end{cases},
  \]
  where $\rho = \min_{i \in V} a_i$ and $C_1, C_2$ are universal constants.
\end{theorem}

For the labeled case we show that we even in our restricted setting we can still
achieve an almost linear rate of convergence as stated in the following theorem

\begin{theorem}
  For any instance $I = (G, s, a)$ the update rule \ref{eq:labeled_update_rule}
  has the following convergence rate
\[
\Exp{\norm{\infty}{x^t - x^*}} \leq

  \]
  where $\rho = \min_{i \in V} a_i$ and $C_1, C_2$ are universal constants.
\end{theorem}



% In this limited information environment, we propose simple and natural
% models that share similar convergence properties with the original FJ-model.

