We study opinion formation games based on the famous model proposed by Friedkin
and Johsen.  In today's huge social networks the assumption that in each round
agents update their opinions by taking into account the opinions of
\emph{all} their friends could be unrealistic. Therefore, we assume that in
each round each agent gets to meet with only one random friend of
hers.  Since it is more likely to meet some friends than others we assume
that agent $i$ meets agent $j$ with probability $p_{ij}$.
Specifically, we define an opinion formation game, where at round $t$,
agent $i$ with intrinsic opinion $s_i\in[0,1]$ and expressed opinion $x_i(t)
\in[0,1]$ meets with probability $p_{ij}$ neighbor $j$ with opinion $x_j(t)$
and suffers a disagreement cost that is a convex combination of
$(x_i(t) - s_i)^2$ and $(x_i(t) - x_j(t))^2$.


For a dynamics in the above setting to be considered as natural it must be
simple, converge to the equilibrium $x^*$, and perhaps most importantly, it
must be a rational choice for selfish agents.  In this work we show that
an intuitive game play, is a natural dynamics for the above game.
We prove that, after $O(1/\eps^2)$ rounds, the opinion vector
is within error $\eps$  of the equilibrium. Moreover, to show that selfish
agents would choose to update their opinions according to our update rule,
we show that it has no-regret.


The classical Friedkin-Johsen dynamics converges to the equilibrium within
error $\eps$ after only $O(\log(1/\eps))$ rounds whereas, in our imperfect
information setting, our update rule needs $\widetilde{O}(1/\eps^2)$ rounds.
We ask whether there exists a different natural dynamics for our problem
with better rate of convergence.  We answer this question in the negative
by showing that dynamics based on no-regret algorithms cannot converge with
less than $\mrm{poly}(1/\eps)$ rounds.
