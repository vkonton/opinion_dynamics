\section{Introduction}

The study of opinion formation dynamics has a long history (see e.g.
\cite{Jackson}).  Today, the advent of social media makes the study of opinion
formation in such large social networks even more important.  Apart from purely
theoretical interest, realistic models of how people form their opinions by
interacting with each other in such social networks are also of practical
interest for prediction, advertisement etc.  Opinion formation is based on
information exchange between socially connected people (e.g. family, friends,
colleagues), who interact often and affect each other's opinion.  Moreover,
opinion formation is often \emph{dynamic} in the sense that discussions and
interactions lead to changes in the expressed opinions. The wide use of the
internet and social media have made the dynamic aspects of opinion formation
even more dominant.  To capture opinion formation on a formal level, several
models have been proposed (see e.g., \cite{DeGroot,FJ90,HK,BKO11} for
continuous opinions and \cite{FGV12,YOASS13,BFM16} for discrete ones).  A
common assumption, that dates back to DeGroot \cite{DeGroot}, is that opinions
evolve through a form of repeated averaging of information collected from the
agent social neighborhoods.

Our work builds on the model proposed by Friedkin and Johnsen \cite{FJ90}(FJ
model).  The FJ model is a variation on the DeGroot model that captures the
fact consensus on the opinions of a social group is not generally reached.
According to FJ model there exists a set of $n$ agents.  Each agent $i$ holds
an internal opinion $s_i\in [0,1]$, which is private and invariant over time
and a public opinion $x_i \in [0,1]$.  Initially, agents start with their
internal opinion and at each round $t\geq1$, update their public opinion
$x_i(t)$ to a weighted average of public opinion of their social neighbors and
their internal opinion,
%
\begin{equation}\label{eq:FJ_model}
  %
  x_i(t)= \frac{\sum_{j\neq i}w_{ij}x_j(t-1) + w_{ii}s_i}{\sum_{j\neq
      i}w_{ij}+w_{ii}}
  %
\end{equation}
%
where the weights $w_{ij}\geq 0$ quantify the influence between the agents and
$w_{ii}>0$ the self confidence of the agents towards their internal belief.

The FJ model is one of most influential models for opinion formation.  Its has
a very simple update rule, making it plausible for modeling natural behavior
and its basic assumptions are aligned with empirical findings on the way
opinions are formed \cite{AFH05,K47}.  At the same time, it admits a unique
equilibrium point $x^* \in [0,1]^n$ to which it converges exponentially fast
\cite{GS14}.  The FJ model has also been studied under a game theoretic view
point.  Bindel et al. considered its update rule as the minimizer of a
quadratic disagreement cost function and based on it they defined the following
one shot opinion formation game \cite{BKO11}. The strategy of each agent $i$ is
her public opinion $x_i$, incurring her a disagreement cost
%
\begin{equation}\label{eq:BKO_cost}
  %
  C_i(x_i,x_{-i})= \sum_{j \neq i}w_{ij} (x_i-x_j)^2 + w_{ii}(x_i-s_i)^2
  %
\end{equation}
%
Note that the FJ model is the \emph{simultaneous best response dynamics} in the
repeated version of this one shot game and that the Nash equilibrium of the
above game is $x^*$. In \cite{BKO11} they quantified the inefficiency of the
equilibrium with respect to the total disagreement cost. Namely, the Price of
Anarchy is less than $9/8$ when $w_{ij}=w_{ji}$.

We remark that in \cite{BKO11}, Bindel et al. also introduced an alternative,
game-theoretic, framework for studying the dynamics of opinions.  Instead of
modeling the way opinions evolve by precise dynamical processes, agents can be
considered to repeatedly play a suitable one shot opinion formation game and
update their opinions so as to minimize their individual disagreement cost.
This framework allows for different aspects of the opinion formation process to
be captured by suitable one shot games.  For example, in \cite{BGM13,EFHS17}
variants of the above opinion formation game were introduced and the
convergence properties of the \emph{best response dynamics} were studied.
Conceptually, this framework, also allows for the agents to update their
opinions in more abstract ways than a specific update rule of a precise
dynamical process.  For example, in the above repeated game, agents, instead of
adopting \emph{best response}, could update their opinions according to a
\emph{no-regret} learning algorithm.

\subsection{Motivation and our setting}

The equilibrium point $x^*$ of the FJ model is generally considered to be a
good estimation of the final opinions formed in a social network. For example,
Gionis et al.  \cite{GTT13} considered the problem of identifying $k$ agents in
a network to set their internal opinion equal to 1 so as to maximize the sum
of the opinions in the equilibrium point $x^*$ ($\norm{1}{x^*}$). This work
belongs to a recent line of research concerning influence maximization problems
with respect to $x^*$, \cite{GTT13,AKPT18,MMT17}. The reason for modeling such
problems by the equilibrium $x^*$ is evident: the FJ model is one of the
standard models for opinion formation and it converges exponentially fast to
$x^*$.

Our work is motivated by the fact that the FJ model assumes a large amount of
information exchange between σthe agents. At each round, its update rule
requires that every agent learns all the opinions of her social neighbors.  In
today's large social networks where users usually have several hundreds of
friends it is unlikely that, each day, they learn the opinions of all their
social neighbors.  In such environments it is far more reasonable to assume
that individuals randomly meet a small subset of their acquaintances and these
are the only opinions that they learn.  Information exchange constraints
render the FJ model unsuitable for modeling the opinion formation process in
such large networks and therefore, it is not clear whether $x^*$ captures the
limiting behavior of the opinions. In this work we ask:
%
\begin{question}\label{q:motivation1}
  %
  Is the equilibrium $x^*$ the correct way to model the final formed opinions
  in a large social network? Namely, are there simple variants of the FJ model
  that require limited information exchange and converge fast to $x^*$? Can
  they be justified as natural behavior for selfish agents under a
  game-theoretic solution concept?
  %
\end{question}

To capture the fact that, in large networks, agents meet a small random subset
of their friends each day, we consider an imperfect information variant of the
opinion formation game defined in \cite{BKO11}. In equation~(\ref{eq:BKO_cost})
the coefficient $w_{ij}$ measures the influence that $j$ poses on $i$.
Following the common belief that \enquote{the more often we interact with
  someone, the more we are influenced by her}, $w_{ij}$ can be interpreted as a
measure on how frequently $i$ meets $j$. In our imperfect information variant,
each agent $i$ randomly meets an agent $j$ (with respect to her weights) and
experiences disagreement cost based on the distance of their opinions and on
the distance of the expressed opinion of $i$ and her internal opinion.  More
formally, we consider the following one shot game.
%
\begin{definition}\label{d:random_game}
  %
  For a given opinion vector $x \in [0,1]^n$, the disagreement cost of agent
  $i$ is the random variable $C_i(x_i,x_{-i})$ defined as follows:
  \begin{itemize}
      %
    \item Agent $i$ meets one of her neighbors $j$ with probability $p_{ij}=
      w_{ij}/\sum_{j\in N_i}w_{ij}$.
      %
    \item Agent $i$ suffers cost $C_i(x_i , x_{-i}) = (1-a_i)(x_i-x_j)^2 +
      a_i(x_i-s_i)^2$, where $\alpha_i = w_i/(\sum_{j\in N_i}w_{ij}+w_i)$.
      %
  \end{itemize}
  %
\end{definition}
%
Our one shot game departs from the original opinion formation game of
\cite{BKO11}.  In the game of \cite{BKO11} an opinion vector $x\in [0,1]^n$
defines \emph{deterministically} the cost $C_i(x)$ of each agent $i$, whereas
in our variant of Definition~\ref{d:random_game} it defines a probability
distribution on the cost $C_i(x)$ that $i$ suffers.  Notice that the expected
disagreement cost of each agent $i$ (scaled by the constant $\sum_{j\neq
  i}w_{ij}+w_{ii}$) is the same as the disagreement cost defined in
equation~(\ref{eq:BKO_cost}).  As a result, the Nash equilibrium with respect
to the expected cost of the agents, is the equilibrium point $x^*$.

In order to model the opinion formation process in limited information exchange
environments, we assume that the agents are engaged in the above one-shot
game. At round $t$, each agent $i$ selects an opinion $x_i(t)$ and suffers a
disagreement cost based on the opinion of the neighbor that she randomly met.
At the end of the round she is informed only about the opinion of this neighbor
and may use this information to update her opinion. Information
exchange is now vastly reduced: at each round each agent learns the opinion of
\emph{just one} of her neighbors.  Question~\ref{q:motivation1} now takes the
following more concrete form.
%
\begin{question}\label{q:motivation2}
  %
  Can the agents update their opinions according to the limited information
  that they receive such that the produced opinion vector $x(t)$ convergences
  fast to the equilibrium $x^*$ and the total disagreement cost that they
  experience is minimal?
  %
\end{question}

\subsection{Contribution}

We introduce a simple and intuitive update rule and we show that if the agents
adopt it to update their opinions, the
resulting opinion vector $x(t)$ converges to $x^*$.  Our update rule is a
\emph{Follow the Leader algorithm}, meaning that at round $t$, each agent
updates her opinion to the minimizer of total disagreement cost that she
experienced until $t-1$.  It also has a very simple form: it is roughly the time average
of the opinions that the agent observes.  In
Section~\ref{s:fictitious_convergence}, we bound its convergence rate and we
show that in order to achieve $\eps$ distance from $x^*$, $\mrm{poly}(1/\eps)$ rounds
are needed. The main technical difficulty, which lies in the stochastic nature of our
process, is overcome by using elegant concentration arguments.  In
Section~\ref{s:fictitious_noregret}, we show that this rule ensures
\emph{no-regret} to any agent that adopts it.  Namely, the average disagreement
cost (that the agent experiences) per round approaches that of expressing the
best opinion in hindsight.  The latter makes our algorithm a natural behavioral
assumption for agents that selfishly want to minimize their incurred
disagreement cost.  The no regret property of our update rule can be derived by
the more general results in \cite{HAK07}.  However, we present a short and
simple proof that may be of interest.  Our results imply that even if agents
are subjected to information exchange constraints, there exist natural
dynamics that converge to equilibrium $x^*$, making it a robust choice for
modeling the limiting behavior of the opinions.

In Section~\ref{s:lower_bound}, we show that for any update rule that ensures
\emph{no-regret} to the agents, the resulting opinion vector $x(t)$ cannot
converge to $x^*$ faster than polynomially.  We prove the latter for a larger
class of update rules, \emph{opinion dependent update rules}, which includes all
the update rules that only depend on the observed opinions.  We show that
the existence of opinion dependent update rules with fast convergence rate
implies the existence of Bernoulli estimators with small sample complexity rate and we
present a novel information theoretic argument to rule out their existence.
Interestingly, polynomial convergence is not a generic property of our limited
information exchange setting.  In Section~\ref{s:cc_convergence}, we present an
update rule that apart from the observed opinions, also uses the weights
$w_{ij}$ and converges exponentially fast to $x^*$. Our results provide a clear
and intuitive characterization of the update rules that can achieve exponential
convergence and serve as an \enquote{algorithmic guide} for future limited
information exchange variants of the FJ model.  Moreover, they indicate the
fundamental reason underlying the exponentially fast convergence of the FJ model.
It has
little to do with the \enquote{large} information exchange that it requires and
much more to do with the fact that it uses the exact values of the weights
$w_{ij}$.

\subsection{Related Work}

Apart from the aforementioned results there exists a large amount of literature
concerning the FJ model.  Many recent works \cite{BGM13,CKO13,BFM16,EFHS17}
bound the inefficiency of equilibrium in variants of opinion formation game
defined in \cite{BKO11}. In \cite{GS14} they bound the convergence time of
the FJ model in special graph topologies.  In \cite{BFM16}, a variant of the
opinion formation game in which social relations depend on the expressed
opinions is studied.  They prove that the discretized version of the above
game admits a potential function and thus best-response converges to the Nash
equilibrium. Convergence results in other discretized variants of the FJ model
can be found in \cite{YOASS13,FGV16}. In \cite{FPS16} the convergence
properties of limited information variants of the Heglesmann-Krause model
\cite{HK} and the FJ model are examined.

Other works that relate to ours concern the convergence properties of
dynamics based on no-regret learning algorithms.  In
\cite{FV97,FS99,SA00,SALS15} it is proved that in a finite $n$-person game if
each agent updates her mixed strategy according to a no-regret algorithm the
resulting \emph{time-averaged} strategy vector converges to Coarse Correlated
Equilibrium. The convergence properties of no-regret dynamics for games with
infinite strategy spaces were considered in \cite{EMN09}.  They proved that for
a large class of games with concave utility functions (socially concave games),
the time-averaged strategy vector converges to the PNE. More recent work
investigates a stronger notion of convergence of no-regret dynamics. In
\cite{CHM17} they show that, in $n$-person finite generic games that admit
unique Nash equilibrium, the strategy vector converges \emph{locally} and
exponentially fast to it. They also provide conditions for \emph{global}
convergence.  Our results fit in this line of research since we show that for a
game with \emph{infinite} strategy space, the strategy vector (and not the
time-averaged) converges to the unique Nash equilibrium.

Imperfect information settings have recently received substantial attention
from the scientific community since they provide realistic models for the
practical applications of game theory.  Perfect payoff information is rare in
practice; agents act based on random or noisy past payoff observations.
Kleinberg et al. in \cite{KPT09} treated load-balancing in distributed systems
as a repeated game and analyzed the convergence properties of no-regret online
learning algorithms under the \emph{full information assumption} that each
agent learns the load of every machine.  In a subsequent work \cite{KPT11}, the
same authors consider the same problem in a \emph{limited information setting}
(\enquote{bulletin board model}), in which each agent learns the load of just
the machine that served him. In \cite{HCM17,MS17} they examine the convergence
properties of online learning algorithms in case the payoffs observed by the
agents are contaminated with random noise.  Finally, in \cite{BM17} they show
stochastic dynamics that lead to no regret regardless of the amount of noise in
the player's observations and show that in specific zero-sum games time
averages converge to the NE for any noise level.
