\section{Introduction}

The study of \emph{Opinion Formation} has a long history (see e.g.
\cite{Jackson}). Opinion Formation is a \emph{dynamic process} in the sense
that socially connected people (e.g. family, friends, colleagues) exchange
information and this leads to changes in their expressed opinions over time.
Today, the advent of the internet and social media makes the study of opinion
formation in large social networks even more important; realistic models of how
people form their opinions by interacting with each other, are of great
practical interest for prediction, advertisement etc. In an attempt to
formalize the process of opinion formation, several models have been proposed
over the years (see e.g., \cite{DeGroot,FJ90,HK,DNAW00}.  The common assumption
underlying all these models, which dates back to DeGroot \cite{DeGroot}, is
that opinions evolve through a form of repeated averaging of information
collected from the agents' social neighborhoods.

Our work builds on the model proposed by Friedkin and Johnsen \cite{FJ90}.  The
FJ model is a variation on the DeGroot model capturing the fact that consensus
on the opinions is rarely reached.  According to FJ model each person $i$ has a
public opinion $x_i \in [0,1]$ and an internal opinion $s_i\in [0,1]$, which is
private and invariant over time. There also exists a weighted graph $G(V,E)$
representing a social network where $V$ stands for the persons ($|V|=n$) and
$E$ their social relations. Initially, all nodes start with their internal
opinion and at each round $t$, update their public opinion $x_i(t)$ to a
weighted average of the public opinions of their neighbors and their internal
opinion,
%
\begin{equation}\label{eq:FJ_model}
  %
  x_i(t)= \frac{\sum_{j\in N_i}w_{ij}x_j(t-1) + w_{ii}s_i}{\sum_{j\in
      N_i}w_{ij}+w_{ii}},
  %
\end{equation}
%
where $N_i =\{j \in V:(i,j) \in E\}$ is the set of $i$'s neighbors, the weight
$w_{ij}$ associated with the edge $(i,j) \in E$ measures the extend of the
influence that $j$ poses on $i$ and the weight $w_{ii}>0$ quantifies how
susceptible is $i$ in adopting opinions that differ from her internal opinion
$s_i$.

The FJ model is one of most influential models for opinion formation. It has a
very simple update rule, making it plausible for modeling natural behavior and
its basic assumptions are aligned with empirical findings on the way opinions
are formed \cite{AFH05,K47}.  At the same time, it admits a unique stable point
point $x^* \in [0,1]^n$ to which it converges with a \emph{linear} rate \cite{GS14}.
The FJ model has also been studied under a game theoretic viewpoint.  Bindel et
al. considered its update rule as the minimizer of a quadratic disagreement
cost function and based on it they defined the following opinion formation game
\cite{BKO11}. Each node $i$ is a selfish agent whose strategy is the public
opinion $x_i$ that she expresses incurring her a disagreement cost
%
\begin{equation}\label{eq:BKO_cost}
  %
  C_i(x_i,x_{-i})= \sum_{j \in N_i}w_{ij} (x_i-x_j)^2 + w_{ii}(x_i-s_i)^2
  %
\end{equation}
%
Note that the FJ model is the \emph{simultaneous best response dynamics} and
its stable point $x^*$ is the unique Nash equilibrium of the above game.  In
\cite{BKO11} they quantified its inefficiency with respect to the total
disagreement cost. They proved that the \emph{Price of Anarchy} (PoA) is $9/8$
in case $G$ is undirected and $w_{ij}=w_{ji}$. They also provided PoA bounds in
the case of unweighted Eulerian directed graphs.

%We remark that in \cite{BKO11} an alternative framework for studying the
%dynamics of opinions were introduced. Instead of modeling the way opinions
%evolve by precise dynamical processes, agents can be considered to repeatedly
%play a suitable one shot opinion formation game and update their opinions so
%as to minimize their individual disagreement cost.  This framework allows for
%different aspects of the opinion formation process to be captured by suitable
%one shot games.  For example, in \cite{BGM13,BFM16,EFHS17} variants of the
%above opinion formation game were introduced and the convergence properties of
%the \emph{best response dynamics} were studied.  Conceptually, this framework
%also allows for the agents to update their opinions in more abstract ways than
%a specific update rule of a precise dynamical process.  For example, in the
%above repeated game, agents, instead of adopting \emph{best response}, could
%update their opinions according to a \emph{no regret} learning algorithm.

We remark that in \cite{BKO11} an alternative framework, for studying the way
opinions evolve, were introduced. Namely, the opinion formation process can be
described as the \emph{dynamics} of an opinion formation game. This framework
is much more comprehensive for two reasons. At first, different aspects of the
opinion formation process can be easily captured by defining suitable games.
Secondly, it permits considering different \emph{natural dynamics} for the same
opinion formation game (e.g. \emph{best response dynamics, no regret dynamics,
  fictitious play}). Following this framework subsequent works
\cite{BGM13,BFM16,EFHS17} considered variants of the above game and studied the
convergence properties of the \emph{best response dynamics}.

\subsection{Motivation and our setting}

Many recent works study the Nash equilibrium $x^*$ of the opinion formation
game defined in \cite{BKO11} under various perspectives. In \cite{CCL16} they
extended the bounds for PoA in more general classes of directed graphs, while
many recently introduced influence maximization problems
\cite{GTT13,AKPT18,MMT17} are defined with respect to $x^*$.  For example,
Gionis et al. \cite{GTT13} considered the problem of identifying $k$ nodes in a
network to set their internal opinion equal to 1 so as to maximize the sum of
the opinions in the equilibrium point $x^*$ ($\norm{1}{x^*}$).  The reason for
this scientific interest is evident: the equilibrium $x^*$ is considered as an
appropriate way to model the final opinions formed in a social network, since
the \emph{well established} FJ model converges to it.

Our work is motivated by the fact that there are notable cases in which the FJ
model is not an appropriate model for the dynamic of the opinions, due to the
large amount of information exchange that it implies.  More precisely, at each
round its update rule~(\ref{eq:FJ_model}) requires that every agent learns all
the opinions of her social neighbors.  In today's large social networks where
users usually have several hundreds of friends it is highly unlikely that, each
day, they learn the opinions of all their social neighbors.  In such
environments it is far more reasonable to assume that individuals randomly meet
a small subset of their acquaintances and these are the only opinions that they
learn. Such information exchange constraints render the FJ model unsuitable for
modeling the opinion formation process in such large networks and therefore, it
is not clear whether $x^*$ captures the limiting behavior of the opinions. In
this work we ask:
%
\begin{question}\label{q:motivation1}
  %
  Is the equilibrium $x^*$ an adequate way to model the final formed opinions
  in a large social network? Namely, are there simple variants of the FJ model
  that require limited information exchange and converge fast to $x^*$? Can
  they be justified as natural behavior for selfish agents under a
  game-theoretic solution concept?
  %
\end{question}

To address these questions, one could define precise dynamical processes whose
update rules require limited information exchange between the agents and study
their convergence properties. Instead of doing that, we describe the opinion
formation process in such large networks as \emph{dynamics} of an adequate
opinion formation game that captures these information exchange constraints.
This way we can precisely define which \emph{dynamics} are \emph{natural}, and,
more importantly, to study general classes of \emph{dynamics} (e.g. no regret
dynamics) without explicitly defining their update rule.  The opinion formation
game that we consider is a straightforward variant of the game defined in
\cite{BKO11} based on interpreting the weight $w_{ij}$ as a measure on how
frequently $i$ meets $j$.
%
\begin{definition}\label{d:random_game}
  %
  For a given opinion vector $x \in [0,1]^n$, the disagreement cost of agent
  $i$ is the random variable $C_i(x_i,x_{-i})$ defined as follows:
  \begin{itemize}
      %
    \item Agent $i$ meets one of her neighbors $j$ with probability $p_{ij}=
      w_{ij}/\sum_{j\in N_i}w_{ij}$.
      %
    \item Agent $i$ suffers cost $C_i(x_i , x_{-i}) = (1-a_i)(x_i-x_j)^2 +
      a_i(x_i-s_i)^2$, where $\alpha_i = w_{ii}/(\sum_{j\in
        N_i}w_{ij}+w_{ii})$.
      %
  \end{itemize}
  %
\end{definition}
%
Note that the FJ model, the game defined in \cite{BKO11}, and the game of
Definition~\ref{d:random_game} are all defined using equivalent instances
$G(V,E,w)$.  This game provides us with a general template of all the \emph{dynamics}
examined in this paper.  At round $t$, each agent $i$ selects an opinion
$x_i(t)$ and suffers a disagreement cost based on the opinion of the neighbor
that she randomly met.  At the end of the round $t$, she is informed only about
the opinion and the index of this neighbor and may use this information to
update her opinion in the next round.  Obviously different update rules lead to
different \emph{dynamics}, however all of these respect the information
exchange constraints: at every round each agent learns the opinion of \emph{just
  one} of her neighbors.  Question~\ref{q:motivation1} now takes the following
more concrete form.

\begin{question}\label{q:motivation2}
  %
  Can the agents update their opinions according to the limited information
  that they receive such that the produced opinion vector $x(t)$ converges
  fast to the equilibrium $x^*$ and the total disagreement cost that they
  experience is minimal?
%
\end{question}

\subsection{Contribution}
We discuss here the contribution of this work at a high level; for
the formal statements of our theorems see Section~\ref{s:results}.
Since we are most concerned about the dependence of the rate of convergence
on the distance $\eps$ from the equilibrium, we shall suppress the dependence on
other constants of the problem such as the size of the graph, $n$.
\begin{definition}\label{d:convergence_rate}
  We say that a dynamics converges slow resp. fast to its stable point $x^*$ if it
  requires $\mrm{poly}(1/\eps)$ resp. $\mrm{poly}(\log(1/\eps))$ rounds to be
  within (expected) error $\eps$ of $x^*$.
\end{definition}

 We know that the FJ model converges fast to the equilibrium $x^*$: the opinion vector
 $x(t)$ is in distance $\eps$ in $O(\log 1/\eps)$ rounds \cite{GS14}.  We characterize
 the \emph{limited
  information dynamics} that can achieve such fast convergence.  We prove that
if the update rule, that the agents use, only depends on the opinions of the
neighbors that they meet, then there exists an instance such that the produced
opinion vector $x(t)$ requires at least $\Omega(1/\eps)$ rounds to be
$\eps$-close to $x^*$. We call these update rules \emph{opinion dependent} and
the respective dynamics \emph{opinion dependent dynamics}.  In
Section~\ref{s:lower_bound}, we show that the existence of opinion dependent
update rules with fast convergence rate implies the existence of Bernoulli
estimators with small sample complexity and we present a novel, information
theoretic argument, to rule out their existence.  The reason that we are
interested in \emph{opinion dependent dynamics} is that they subsume the class
of \emph{no regret dynamics} which is generally accepted as
\emph{natural behavior}  \cite{EMN09}.
We say that an update rule ensures no regret to any agent that adopts it, when
the time averaged disagreement cost (that the agent experiences) per round is
close to the cost of expressing the best opinion in hindsight independently of
the neighbors that she meets. Therefore, update rules that ensure no regret are
a reasonable choice for selfish agents.

We introduce a simple and intuitive \emph{opinion dependent update rule} and we
show that if the agents adopt it, the resulting opinion vector $x(t)$ converges
to $x^*$.  Our update rule is a \emph{Follow the Leader algorithm}, meaning that
at round $t$, each agent updates her
opinion to the minimizer of total disagreement cost that she experienced until
round $t-1$.  It also has a very simple form: it is roughly the time average of
the opinions that the agent observes.  In Section~\ref{s:fictitious_convergence},
we bound its convergence rate and we show that in order to achieve $\eps$ distance
from $x^*$, $\mrm{poly}(1/\eps)$ rounds are needed.  In Section~\ref{s:fictitious_noregret},
we show that this rule also ensures \emph{no regret} to the agents.  Its no regret
property can be derived by the more general results in \cite{HAK07}.  However, we
give a short and simple proof that may be of some interest.

In Section~\ref{s:cc_convergence} we show that \enquote{slow convergence} is
not a generic property of the \emph{limited information dynamics}.  We present
an update rule that apart from the observed opinions, also uses the weights
$w_{ij}$ and converges exponentially fast to $x^*$.  This update rule reveals
that the slow convergence of \emph{opinion dependent} update rules is not due
to the reduced information exchange (learning the opinion of only one agent),
but due to the fact that the agents are \enquote{oblivious} to the weighted
graph of the instance and they \enquote{learn} it during the game play.

Our results reveal that the equilibrium $x^*$ is a robust choice for modeling the
limiting behavior of the opinions of agents since even in our limited information
setting, there exist natural dynamics that converge to it.  Our lower bound
indicates that rationality comes at the price of \enquote{slow} convergence.

\subsection{Further Related Work}

There exists a large amount of literature concerning the FJ model.  Many recent
works \cite{BGM13,CKO13}, \cite{BFM16,EFHS17} bound the inefficiency of
equilibrium in variants of opinion formation game defined in \cite{BKO11}. In
\cite{GS14} they bound the convergence time of the FJ model in special graph
topologies.  In \cite{BFM16}, a variant of the opinion formation game in which
social relations depend on the expressed opinions is studied.  They prove that
the discretized version of the above game admits a potential function and thus
best-response converges to the Nash equilibrium.  Convergence results in other
discretized variants of the FJ model can be found in \cite{YOASS13,FGV16}. In
\cite{FPS16} they provide convergence results for limited information variants
of the Heglesmann-Krause model \cite{HK} and the FJ model. Although they
considered limited information variant of the FJ model is very similar to ours,
their convergence results are much weaker, since they concern the expected
value of the opinion vector.

Other works that relate to ours, concern the convergence properties of dynamics
based on no regret learning algorithms.  In \cite{FV97,FS99,SA00,SALS15} it is
proved that in a finite $n$-person game if each agent updates her mixed
strategy according to a no regret algorithm the resulting \emph{time-averaged}
strategy vector converges to Coarse Correlated Equilibrium. The convergence
properties of no regret dynamics for games with infinite strategy spaces were
considered in \cite{EMN09}.  They proved that for a large class of games with
concave utility functions (socially concave games), the time-averaged strategy
vector converges to Pure Nash Equilibrium (PNE). More recent work investigates
a stronger notion of convergence of no regret dynamics. In \cite{CHM17} they
show that, in $n$-person finite generic games that admit unique Nash
equilibrium, the strategy vector converges \emph{locally} and exponentially
fast to it. They also provide conditions for \emph{global} convergence.  Our
results fit in this line of research since we show that for a game with
\emph{infinite} strategy space, the strategy vector (and not the time-averaged)
converges to the Nash equilibrium $x^*$.

No regret dynamics in imperfect information settings have recently received
substantial attention from the scientific community since they provide
realistic models for the practical applications of game theory.  Perfect payoff
information is rare in practice; agents act based on random or noisy past
payoff observations.  Kleinberg et al. in \cite{KPT09} treated load-balancing
in distributed systems as a repeated game and analyzed the convergence
properties of no regret learning algorithms under the \emph{full information
  assumption} that each agent learns the load of every machine.  In a
subsequent work \cite{KPT11}, the same authors consider the same problem in a
\emph{imperfect information setting} (\enquote{bulletin board model}), in which
each agent learns the load of just the machine that served him. Most relevant
to ours, are the works \cite{HCM17,MS17,BM17,CHM17}, where they examine the
convergence properties of no regret learning algorithms when the agents observe
their payoffs with some additive zero-mean random noise. In our imperfect
information setting the agents experience random disagreement cost with
expected value equal to the actual cost.  The main difference is that our
\emph{noise} is not additive but due to a sampling process.
