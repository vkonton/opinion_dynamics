\section{Introduction}

The study of \emph{Opinion Formation} has a long history (see e.g.
\cite{Jackson}). Opinion Formation is a \emph{dynamic process} in the sense
that socially connected people (e.g. family, friends, colleagues) exchange
information and this leads to changes in their expressed opinions over time.
Today, the advent of the internet and social media makes the study of opinion
formation in large social networks even more important; realistic models of how
people form their opinions by interacting with each other, are of great
practical interest for prediction, advertisement etc. In an attempt to
formalize the process of opinion formation, several models have been proposed
over the years (see e.g., \cite{DeGroot,FJ90,HK,DNAW00}.  The common assumption
underlying all these models, which dates back to DeGroot \cite{DeGroot}, is
that opinions evolve through a form of repeated averaging of information
collected from the agents' social neighborhoods.

Our work builds on the model proposed by Friedkin and Johnsen \cite{FJ90}.  The
FJ model is a variation on the DeGroot model capturing the fact that consensus
on the opinions is rarely reached.  According to FJ model each person $i$ has a
public opinion $x_i \in [0,1]$ and an internal opinion $s_i\in [0,1]$, which is
private and invariant over time. There also exists a weighted graph $G(V,E)$
representing a social network where $V$ stands for the persons ($|V|=n$) and
$E$ their social relations. Initially, all nodes start with their internal
opinion and at each round $t$, update their public opinion $x_i(t)$ to a
weighted average of the public opinions of their neighbors and their internal
opinion,
%
\begin{equation}\label{eq:FJ_model}
  %
  x_i(t)= \frac{\sum_{j\in N_i}w_{ij}x_j(t-1) + w_{ii}s_i}{\sum_{j\in
      N_i}w_{ij}+w_{ii}},
  %
\end{equation}
%
where $N_i =\{j \in V:(i,j) \in E\}$ is the set of $i$'s neighbors, the weight
$w_{ij}$ associated with the edge $(i,j) \in E$ measures the extent of the
influence that $j$ poses on $i$ and the weight $w_{ii}>0$ quantifies how
susceptible $i$ is in adopting opinions that differ from her internal opinion
$s_i$.

The FJ model is one of most influential models for opinion formation. It has a
very simple update rule, making it plausible for modeling natural behavior and
its basic assumptions are aligned with empirical findings on the way opinions
are formed \cite{AFH05,K47}.  At the same time, it admits a unique stable
point $x^* \in [0,1]^n$ to which it converges with a \emph{linear} rate
\cite{GS14}.  The FJ model has also been studied under a game theoretic
viewpoint.  Bindel et al. considered its update rule as the minimizer of a
quadratic disagreement cost function and based on it they defined the following
opinion formation game \cite{BKO11}. Each node $i$ is a selfish agent whose
strategy is the public opinion $x_i$ that she expresses incurring her a
disagreement cost
%
\begin{equation}\label{eq:BKO_cost}
  %
  C_i(x_i,x_{-i})= \sum_{j \in N_i}w_{ij} (x_i-x_j)^2 + w_{ii}(x_i-s_i)^2
  %
\end{equation}
%
Note that the FJ model is the \emph{simultaneous best response dynamics} and
its stable point $x^*$ is the unique Nash equilibrium of the above game.  In
\cite{BKO11} they quantified its inefficiency with respect to the total
disagreement cost. They proved that the \emph{Price of Anarchy} (PoA) is $9/8$
in case $G$ is undirected and $w_{ij}=w_{ji}$. They also provided PoA bounds in
the case of unweighted Eulerian directed graphs.  We remark that in
\cite{BKO11} an alternative framework for studying the way opinions evolve
was introduced.  The opinion formation process can be described as the
\emph{dynamics} of an opinion formation game.  This framework is much more
comprehensive since different aspects of the opinion formation process can be
easily captured by defining suitable games.
% Secondly, it permits considering different \emph{natural dynamics} for the
% same opinion formation game (e.g. \emph{best response dynamics, no regret
% dynamics, fictitious play}).
Subsequent works \cite{BGM13,BFM16,EFHS17} considered variants of
the above game and studied the convergence properties of the \emph{best
 response dynamics}.

\subsection{Motivation and our setting}

Many recent works study the Nash equilibrium $x^*$ of the opinion formation
game defined in \cite{BKO11} under various perspectives. In \cite{CCL16} they
extended the bounds for PoA in more general classes of directed graphs, while
many recently introduced influence maximization problems
\cite{GTT13,AKPT18,MMT17} are defined with respect to $x^*$.
% For example, Gionis et al. \cite{GTT13} considered the problem of identifying
% $k$ nodes in a network to set their internal opinion equal to 1 so as to
% maximize the sum of the opinions in the equilibrium point $x^*$
% ($\norm{1}{x^*}$).
The reason for this scientific interest is evident: the equilibrium $x^*$ is
considered as an appropriate way to model the final opinions formed in a social
network, since the \emph{well established} FJ model converges to it.

Our work is motivated by the fact that there are notable cases in which the FJ
model is not an appropriate model for the dynamic of the opinions, due to the
large amount of information exchange that it implies.  More precisely, at each
round its update rule~(\ref{eq:FJ_model}) requires that every agent learns all
the opinions of her social neighbors.  In today's large social networks where
users usually have several hundreds of friends it is highly unlikely that, each
day, they learn the opinions of all their social neighbors.  In such
environments it is far more reasonable to assume that individuals randomly meet
a small subset of their acquaintances and these are the only opinions that they
learn. Such information exchange constraints render the FJ model unsuitable for
modeling the opinion formation process in such large networks and therefore, it
is not clear whether $x^*$ captures the limiting behavior of the opinions. In
this work we ask:
%
\begin{question}\label{q:motivation1}
  %
  Is the equilibrium $x^*$ an adequate way to model the final formed opinions
  in large social networks? Namely, are there simple variants of the FJ model
  that require limited information exchange and converge fast to $x^*$? Can
  they be justified as natural behavior for selfish agents under a
  game-theoretic solution concept?
  %
\end{question}

To address these questions, one could define precise dynamical processes whose
update rules require limited information exchange between the agents and study
their convergence properties. Instead of doing so, we describe the opinion
formation process in such large networks as \emph{dynamics} of an adequate
opinion formation game that captures these information exchange constraints.
This way we can precisely define which \emph{dynamics} are \emph{natural}, and,
more importantly, to study general classes of \emph{dynamics} (e.g. no regret
dynamics) without explicitly defining their update rule.  The opinion formation
game that we consider is a variant of the game in \cite{BKO11} based on
interpreting the weight $w_{ij}$ as a measure of how frequently $i$ meets $j$.
%
\begin{definition}\label{d:random_game}
  %
  For a given opinion vector $x \in [0,1]^n$, the disagreement cost of agent
  $i$ is the random variable $C_i(x_i,x_{-i})$ defined as follows:
  \begin{itemize}
      %
    \item Agent $i$ meets one of her neighbors $j$ with probability $p_{ij}=
      w_{ij}/\sum_{j\in N_i}w_{ij}$.
      %
    \item Agent $i$ suffers cost $C_i(x_i , x_{-i}) = (1-a_i)(x_i-x_j)^2 +
      a_i(x_i-s_i)^2$, where $\alpha_i = w_{ii}/(\sum_{j\in
        N_i}w_{ij}+w_{ii})$.
      %
  \end{itemize}
  %
\end{definition}
%
Note that the expected disagreement cost of each agent in the above game
is the same as the disagreement cost in \cite{BKO11} (scaled by $\sum_{j \in N_i}w_{ij}+w_{ii}$). 
Moreover its Nash equilibrium, with respect to the expected disagreement cost, is $x^*$.
This game provides us with a general template of all the \emph{dynamics} examined in this paper.  At
round $t$, each agent $i$ selects an opinion $x_i(t)$ and suffers a
disagreement cost based on the opinion of the neighbor that she randomly met.
At the end of the round $t$, she is informed only about the opinion and the
index of this neighbor and may use this information to update her opinion in
the next round.  Obviously different update rules lead to different
\emph{dynamics}, however all of these respect the information exchange
constraints: at every round each agent learns the opinion of \emph{just one} of
her neighbors.  Question~\ref{q:motivation1} now takes the following more
concrete form.

\begin{question}\label{q:motivation2}
  %
  Can the agents update their opinions according to the limited information
  that they receive such that the produced opinion vector $x(t)$ converges
  to the equilibrium $x^*$? How is the convergence rate affected by the limited
  information exchange? Are there dynamics that ensures that the cost that the agents experience
  is minimal?
  %
\end{question}

In what follows, we are mostly concerned about the dependence of
the rate of convergence on the distance $\eps$ from the equilibrium $x^*$, we shall
suppress the dependence on other parameters such as the size of
the graph, $n$.  We remark that the dependence of our dynamics on these
constants is in fact rather good (see Section~\ref{s:results}), and we only do
this only for clarity of exposition.

\begin{definition}[Informal]\label{d:convergence_rate}
  %
  We say that a dynamics converges slow resp. fast to the equilibrium $x^*$ if
  it requires $\mrm{poly}(1/\eps)$ resp. $\mrm{poly}(\log(1/\eps))$ rounds to
  be within (expected) error $\eps$ of $x^*$.
  %
\end{definition}

\subsection{Contribution}
The major contribution of the paper is proving an exponential separation on
the convergence rate of \emph{no regret dynamics} and the convergence rate of
more general dynamics produced by update rules that do not ensure no regret.

\emph{No regret dynamics} are produced by update rules that
ensure no regret to any agent that adopts them.
Namely, the total disagreement cost of an agent that follows such a
rule, is close to the total disagreement cost that she would experience by
selecting the best fixed opinion in hindsight. The latter must hold
regardless of the way the other agents update their opinions
and of the neighbors that the agent gets to meet.
This powerful property consists no regret dynamics to be \emph{natural dynamics}
for describing the behavior of agents \cite{CL03,EMN09,KPT11,SALS15}.
We prove that if all the agents adopt an update rule that ensures no regret,
then there exists an instance of the game such that the produced opinion
vector $x(t)$ requires roughly $\Omega(1/\eps)$ rounds to
be $\eps$-close to $x^*$. No regret comes at the price of slow convergence
because it provides robust guarantees.
Agents who adopt no regret update rules suffer minimal total disagreement
cost even if the other agents play irrationally or adversarially.
In order to provide such strong guarantees, no regret rules must only depend on
the opinions that the agent observes and do not take into account
the weights $w_{ij}$ of the outgoing edges (see Section~\ref{s:lower_bound}).
We call the update rules with the latter property, \emph{graph oblivious}.
In Section~\ref{s:lower_bound} we use a novel information theoretic argument
to prove the aforementioned lower bound for this more general class.

In Section~\ref{s:cc_convergence}, we present a simple update rule
whose resulting dynamics converges fast, i.e.
the opinion vector $x(t)$ is $\eps$-close to $x^*$ in
$O(\log^2 (1/\eps))$ rounds. The reason that the previous lower
bound doesn't apply is that this rule does not ensure \emph{no regret}
to the agents that adopt it. In fact there is a very
simple example with two agents, in which the first follows the rule
while the second selects her opinions adversarially, where 
the first agent experiences regret (see Example \ref{ex:regret_side_result} 
in Appendix~\ref{app:s:cc_convergence}).

We introduce an intuitive \emph{no regret} update rule and we
show that if all agents adopt it, the resulting opinion vector $x(t)$
converges to $x^*$.  Our rule is a \emph{Follow the Leader algorithm}, meaning
that at round $t$, each agent updates her opinion to the minimizer of total
disagreement cost that she experienced until round $t-1$.  It also has a very
simple form: it is roughly the time average of the opinions that the agent
observes.  In Section~\ref{s:fictitious_convergence}, we bound its convergence
rate and we show that in order to achieve $\eps$ distance from $x^*$,
$\mrm{poly}(1/\eps)$ rounds are sufficient.  In view of our lower bound this rate
is close to best possible.  In Section~\ref{s:fictitious_noregret}, we prove its
\emph{no regret} property. This can be derived by the more general results
in \cite{HAK07}.  However, we give a short and simple proof that may be of
interest.

In conclusion, our results reveal that the equilibrium $x^*$ is a robust choice
for modeling the limiting behavior of the opinions of agents since, even in our
limited information setting, there exist simple and natural dynamics that converge
to it.  The convergence rate crucially depends on whether the
agents act selfishly, i.e. they are only concerned about their individual
disagreement cost.
We present an update rule that selfish agents can adopt (no regret update rule)
and show that the resulting opinion vector converges $x^*$ but with a slow rate,
while, for non selfish agents, the update rule in Section~\ref{s:cc_convergence}
leads to a \emph{dynamics} with fast convergence rate.

\subsection{Related Work}

There exists a large amount of literature concerning the FJ model.  Many recent
works \cite{BGM13,CKO13}, \cite{BFM16,EFHS17} bound the inefficiency of
equilibrium in variants of opinion formation game defined in \cite{BKO11}. In
\cite{GS14} they bound the convergence time of the FJ model in special graph
topologies.  In \cite{BFM16}, a variant of the opinion formation game in which
social relations depend on the expressed opinions is studied.  They prove that
the discretized version of the above game admits a potential function and thus
best-response converges to the Nash equilibrium.  Convergence results in other
discretized variants of the FJ model can be found in \cite{YOASS13,FGV16}. In
\cite{FPS16} they provide convergence results for limited information variants
of the Heglesmann-Krause model \cite{HK} and the FJ model. Although they
considered limited information variant of the FJ model is very similar to ours,
their convergence results are much weaker, since they concern the expected
value of the opinion vector.

Other works that relate to ours, concern the convergence properties of dynamics
based on no regret learning algorithms.  In \cite{FV97,FS99,SA00,SALS15} it is
proved that in a finite $n$-person game if each agent updates her mixed
strategy according to a no regret algorithm the resulting \emph{time-averaged}
strategy vector converges to Coarse Correlated Equilibrium. The convergence
properties of no regret dynamics for games with infinite strategy spaces were
considered in \cite{EMN09}.  They proved that for a large class of games with
concave utility functions (socially concave games), the time-averaged strategy
vector converges to Pure Nash Equilibrium (PNE). More recent work investigates
a stronger notion of convergence of no regret dynamics. In \cite{CHM17} they
show that, in $n$-person finite generic games that admit unique Nash
equilibrium, the strategy vector converges \emph{locally} and fast to it.  They
also provide conditions for \emph{global} convergence.  Our results fit in this
line of research since we show that for a game with \emph{infinite} strategy
space, the strategy vector (and not the time-averaged) converges to the Nash
equilibrium $x^*$.

No regret dynamics in limited information settings have recently received
substantial attention from the scientific community since they provide
realistic models for the practical applications of game theory.  Perfect payoff
information is rare in practice; agents act based on random or noisy past
payoff observations.  Kleinberg et al. in \cite{KPT09} treated load-balancing
in distributed systems as a repeated game and analyzed the convergence
properties of no regret learning algorithms under the \emph{full information
  assumption} that each agent learns the load of every machine.  In a
subsequent work \cite{KPT11}, the same authors consider the same problem in a
\emph{limited information setting} (\enquote{bulletin board model}), in which
each agent learns the load of just the machine that served him. Most relevant
to ours are the works \cite{HCM17,MS17,BM17,CHM17}, where they examine the
convergence properties of no regret learning algorithms when the agents observe
their payoffs with some additive zero-mean random noise. In our limited
information setting the agents experience random disagreement cost with
expected value equal to the actual cost.  The main difference is that our
\emph{noise} is not additive but due to a sampling process.
