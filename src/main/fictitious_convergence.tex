\section{The Convergence Rate of FTL Dynamics}\label{s:fictitious_convergence}
In this section that we prove that FTL dynamics
converges to the unique equilibrium point $x^*$.
Notice that for an instance $I=(P,s,\alpha)$, the opinion vector $x(t) \in [0,1]^n$
of the FTL dynamics (see \ref{alg:FTL_dynamics}) can be written equivalently as follows:
\begin{itemize}
 \item Initially all agents adopt their internal opinion, $x_i(0)=s_i$
 \item At round $t \geq 1$, each agent $i$ updates her opinion
 \(
x_i(t)=(1-\alpha_i)(1/t)\sum_{\tau=0}^{t-1} x_{W_i^\tau}(\tau)+ \alpha_i s_i,
\)
where $W_i^\tau$ is the neighbor that $i$ met at round $t$.
\end{itemize}
Since the opinion vector $x(t)$ is a random vector,
our convergence metric is $\Expnew{}{\norm{\infty}{x(t) - x^*}}$
where the expectation is taken over the random meeting of the agents.
Our convergence result is stated in Theorem~\ref{t:fictitious_convergence}
and it is the main result of the section.
%
%\repeattheorem{t:fictitious_convergence}
%
At first we present a high level idea of the proof.
Notice that the unique equilibrium point $x^* \in [0,1]^n$ of the
instance $I=(P,s,\alpha)$ satisfies the following equations
for each agent $i \in V$,
\[x_i^*= (1-\alpha_i)\sum_{j \in N_i}p_{ij}x_j^* + \alpha_is_i\]
Since we are interested in bounding the $\Expnew{}{\norm{\infty}{x(t)-x^*}}$, we
can use the above equations to bound $|x_i(t)-x_i^*|$.
\begin{eqnarray*}
 |x_i(t)-x_i^*|
 &=& (1-\alpha_i)|\frac{\sum_{\tau=0}^{t-1} x_{W_i^\tau}(\tau)}{t}
 - \sum_{j \in N_i} p_{ij}x_j^*|\\
 &=& (1-\alpha_i)|\sum_{j \in N_i}\frac{\sum_{\tau=0}^{t-1} \1{W_i^\tau=j}x_j(\tau)}{t}-\sum_{j \in N_i} p_{ij}x_j^*|\\
 &\leq& (1-\alpha_i)\sum_{j \in N_i}|\frac{\sum_{\tau=0}^{t-1} \1{W_i^\tau=j}x_j(\tau)}{t}- p_{ij}x_j^*|
\end{eqnarray*}
Now assume that $|\frac{\sum_{\tau=0}^{t-1} \1{W_i^{\tau}=j}}{t}- p_{ij}| = 0$
for all $t\geq 1$, then with simple algebraic manipulations
one can prove that $\norm{\infty}{x(t)-x^*} \leq e(t)$ where $e(t)$
satisfies the recursive equation $e(t) = (1-\rho)\frac{\sum_{\tau=0}^{t-1}e(\tau)}{t}$.
It follows that $\norm{\infty}{x(t)-x^*} \leq 1/t^\rho$
meaning that $x(t)$ converges to $x^*$.
Obviously the latter assumption does not hold,
however since $W_i^{\tau}$ are independent random variables
with $\Prob{W_i^\tau}=p_{ij}$,
$|\frac{\sum_{\tau=0}^{t-1} \1{W_i^{\tau}=j}}{t} - p_{ij}|$
tends to $0$ with probability $1$.
In Lemma~\ref{l:recursive_equation} we use this fact
to obtain a similar recursive relation for $e(t)$
and then in Lemma~\ref{l:recursion_upper_bound}
we upper bound the solution of this recursive equation.
\begin{lemma}\label{l:recursive_equation}
Let $e(t)$ the solution of the following recursion,
\[e(t) =\delta(t) + (1-\rho)\frac{\sum_{\tau=0}^{t-1}e(\tau)}{t}\]
where $e(0)=\|x(0) - x^*\|_{\infty}$,\(\delta(t) = \sqrt{\frac{\ln(\pi^2n t^2/6p)}{t}}\)
and $\rho = \min_{i \in V}\alpha_i$. Then,
\[\Prob{\text{for all }t \geq 1, ~\norm{\infty}{x(t)-x^*}\leq e(t)} \geq 1-p\]
\end{lemma}
\begin{proof}
At first we prove that with probability at least $1-p$,
for all $t \geq 1$ and all agents $i$:
\begin{equation}\label{eq:error_per_round}
    \lp|
    \frac{\sum_{\tau=0}^{t-1} x^*_{W_i^\tau}}{t} -
    \sum_{j \in N_i} p_{ij} x^*_j
    \rp| \leq
    \sqrt{ \frac{\log(\pi^2 n t^2/(6 p))}{t}} :=
    \delta(t).
\end{equation}
Since $W_i^\tau$ are independent random variables with
$\Prob{W_i^\tau = j}=p_{ij}$ and \(\Exp{x^*_{W_i^\tau}} = \sum_{j \in N_i} p_{ij} x^*_j\).
By the Hoeffding's inequality we get
  \[
    \Prob
    {
      \lp|
      \frac{\sum_{\tau=0}^{t-1} x^*_{W_i^\tau}}{t}
      - \sum_{j \in N_i} p_{ij} x^*_j \rp|
      > \delta(t)
     }
    < 6 p / (\pi^2 n t^2).
  \]
To bound the probability of error for all rounds $t=1$ to $\infty$
and all agents $i$, we apply the union bound
  \begin{align*}
    \sum_{t=1}^{\infty}
    \Prob
    { \max_{i}
      \lp|
      \frac{\sum_{\tau=0}^{t-1} x^*_{W_i^{\tau}}}{t}
      - \sum_{j \in N_i} p_{ij} x^*_j \rp|
      > \delta(t)}
    \leq
    \sum_{t=1}^{\infty} \frac{6}{\pi^2} \frac{1}{t^2} \sum_{i=1}^n \frac{p}{n} =
    p
  \end{align*}
  %
As a result with probability at least $1-p$ we have that
inequality (\ref{eq:error_per_round}) holds for all $t\geq 1$ and all
agents $i$.
Now we can prove our claim by induction.
Assume that $\norm{\infty}{x(\tau)-x^*} \leq e(\tau)$ for all
$\tau \leq t-1$. Then
\begin{align}
    x_i(t)
    &=
    (1-\alpha_i)\frac{\sum_{\tau=0}^{t-1}x_{W_i^{\tau}}(\tau)}{t}
    + \alpha_i s_i \nonumber \\
    &\leq
    (1-\alpha_i)\frac{\sum_{\tau=0}^{t-1}x^*_{W_i^{\tau}} +
      \sum_{\tau=0}^{t-1} e(\tau)}{t} + \alpha_i s_i \label{step:induction_step}\\
    &\leq
    (1-\alpha_i)
    \lp(
    \frac{\sum_{\tau=0}^{t-1}x^*_{W_i^{\tau}}}{t}+
    \frac{\sum_{\tau=0}^{t-1} e(\tau)}{t}
    \rp)
    + \alpha_i s_i \nonumber\\
    &\leq
    (1-\alpha_i)\lp(\sum_{j \in N_i} p_{ij} x^*_{j} +
    \delta(t) + \frac{\sum_{\tau=0}^{t-1} e(\tau)}{t} \rp) +
    \alpha_i s_i \label{step:error} \\
    &\leq
    x_i^* + \delta(t) + (1-\rho)
    \lp(
    \frac{\sum_{\tau=0}^{t-1} e(\tau)}{t}
    \rp)
    \nonumber
  \end{align}
  We get (\ref{step:induction_step}) from the induction step and
  (\ref{step:error}) from inequality~(\ref{eq:error_per_round}).
  Similarly, we can prove that
  $x_i(t) \geq x_i^* - \delta(t) - (1-\rho)
  \frac{\sum_{\tau=0}^{t-1} e(\tau)}{t}$.
  As a result $\norm{\infty}{x(t)-x^*} \leq e(t)$ and the induction
  is complete.  Therefore, we have that with probability at least $1-p$,
  \(\norm{\infty}{x(t) - x^*} \leq e(t)\) for all $t\geq 1$.
\end{proof}

\begin{replemma}{l:recursion_upper_bound}
  Let $e(t)$ be a function satisfying the recursion
  \(
    e(t) =
    \delta(t) + (1-\rho)(\sum_{\tau=0}^{t-1}e(\tau))/{t}
    \text{ and } e(0)=\|x(0) - x^*\|_{\infty},
  \)
  where \(\delta(t) = \sqrt{\ln(D t^{2.5})/t} \), \(\delta(0) = 0 \),
  and $D > \me^{2.5}$ is a positive constant.  Then
  \(
    e(t) \leq
    \sqrt{2 \ln(D)} \frac{(\ln t)^{3/2}}{t^{\min(\rho,\, 1/2)}}.
  \)
\end{replemma}
\noindent The proof of Lemma~\ref{l:recursion_upper_bound}
and Theorem~\ref{t:fictitious_convergence}
can be found in the appendix. Theorem~\ref{t:fictitious_convergence}
is proved by a direct application of Lemma~\ref{l:recursive_equation}
and \ref{l:recursion_upper_bound}.
