\section{Faster Update Rules}\label{s:graph_aware}

We already discussed that the reason that opinion dependent dynamics suffer slow
convergence is that the update rule depends only on the expressed opinions.
In this section we provide $2$ update rules showing that information about the
graph $G$ combined with agents that do not act selfishly,
can restore the exponential convergence rate.
Our first update rule, depends not only on the expressed opinions of the
agents but also on their indices and matrix $P$.
\begin{algorithm}
  \caption{Tsitsiklis}
  \label{alg:tsitsiklis}
  \begin{algorithmic}[1]
    \STATE $x_i(0) \gets s_i$.
    \STATE Let $d_i$ be the number of non-negative entries of row $i$ of $P$.
    \STATE Keep an array $A$ of length $d_i$.
    \STATE At round $t$:
    \bindent
    \STATE Meet neighbor $k$, get the opinion $x_k(t-1)$ and the index $k$.
    \STATE $A_k \gets x_k(t-1)$.
    \STATE $x_i(t) = \sum_{j=1}^{d_i} p_{ij} A_j$
    \eindent
\end{algorithmic}
\end{algorithm}

In \cite{BT97}, section 6.3.5, they show a convergence rate guarantee for
\ref{alg:tsitsiklis} assuming that there exists
a window of $B$ rounds such that all agents get to meet all their
neighbors at least once from round $t$ to round $t+B$.
In our randomized setting this is not true but we can easily adapt this to
hold with high probability. In our problem
agent $i$ simply needs to wait to meet the neighbor $j$ with the smallest
weight $p_{ik}$. Therefore, after $\log(1/\delta)/\min_{j} p_{ij}$ rounds
we have that with probability at least $1-\delta$ agent $i$ saw all her
neighbors at least once.  Since we want this to be true for all agents
we shall roughly take $B = 1/\min_{p_{ij} > 0} {p_{ij}}$.
In section~\ref{s:cc} we show that it suffices to take $B = \frac{2}{\min_{p_{ij} > 0} {p_{ij}}}\ln\frac{nt}{p}$
in order for this to hold with probability $1-p$. 
Omitting the details of this straightforward adaptation
we get that with high probability
\[
  \norm{\infty}{x(t) - x^*} = O(\norm{\infty}{P}^{t/(B+1)}).
\]

Precisely, our update rule depends on the expressed opinions, the number of
neighbors of each agent, and the number of agents $n$.

The main idea of the protocol is straightforward: to counterbalance the
imperfect information, the agents can spend some rounds to simulate one round
of the original FJ-model. To do this, they agree to stop updating their
expressed opinion for a large enough window of rounds so that everybody learns,
with high probability, \emph{exactly} the average of the opinions of their
neighbors.  Following the ideas of Section~\ref{s:fictitious_convergence} an
agent could just average the opinions that she gets in this window.
Unfortunately this would again result in a $\poly(1/\eps)$-round protocol.
However this can be fixed by using the additional knowledge of the number
of agents and the number of neighbors. Precisely, each agent
$i$ keeps an array with the frequencies of the different opinions that she
observes. The catch is that at the end of the window, she rounds each frequency
to the closest multiple of $1/d_i$, where $d_i$ is the number of neighbors of
agent $i$. This rounding step is crucial to ensure the exponential convergence
rate. To see that this works, first notice that if all agents stop updating
their opinions for a number of rounds, each agents just needs to specify
exactly how many of her neighbors share a specific observed opinion.
If the length of the window is large the frequency of
a specific opinion at the end of the window will be sufficiently
close to the true frequency.  Since the frequencies of
the opinions that agent $i$ observes can only be multiples of $1/d_i$,
we can round the estimated frequencies to the closest multiple of $1/d_i$ to
recover the true frequencies and use them to get the exact average of the
opinions of the neighbors.  To bound the length of the window we use a
VC-dimension argument and show that with $n^2 \log n$ rounds each agent
knows the frequencies within error smaller than $1/n$ with constant
probability, which then can be trivially amplified by repeating the procedure.

We next state a version of the standard VC-Inequality that we will use
in our argument.  Let $P$ be a discrete distribution over $[n]$, and let
$S_1, \ldots, S_t$ be $t$ i.i.d samples drawn from $P$, i.e.
$(S_1, \ldots, S_t) \sim P^t$.  The empirical distribution $\hat{P}_t$ is the
following estimator of the density of $P$.
\begin{equation}\label{eq:empirical_distribution}
  \hat{P}_t(A) = \frac{\sum_{i=1}^t \1{S_i \in A}}{t},
\end{equation}
where $A \subseteq [n]$. In words, $\hat{P}_t$ simply counts
how many times the value $i$ appeared in the samples $S_1, \ldots S_t$.
We will use the following version of the classical result of Vapnik and
Chervonenkis.
\begin{lemma}\label{l:vc_inequality}
  Let $\mcal{A}$ be a collection of subsets of $\{1,\ldots, n\}$ and let
  $S_{\mcal{A}}(t)$ be the Vapnik-Chervonenkis shatter coefficient, defined
  by
  \[
    S_{\mcal{A}(t)} = \max_{x_1,\ldots, x_t \in [n]}
    \lp| \{ \{x_1, \ldots, x_t\} \cap A : A \in \mcal{A} \} \rp|.
  \]
  Then
  \[
    \Expnew{P^t}{\max_{A \in \mcal{A}} \lp| \hat{P}_t(A) - P(A) \rp|}
    \leq 2 \sqrt{\frac{\log{2 S_{\mcal{A}}(t)}}{t}}
  \]
\end{lemma}

\begin{algorithm}
  \caption{Graph Aware Update Rule}
  \label{alg:frequencies}
  \begin{algorithmic}[1]
    \STATE $x_i(0) \gets s_i$.
    \STATE $M_1 = O(\ln(n/\eps))$, $M_2 = O(d^2 \ln(d))$

    \FOR{$l = 1, \ldots,\ln(1/\eps))$}
    \STATE Keep a set $A$ of tuples $(x, \mrm{freq}(x))$
    of and an array $B$ of length $M_1$.

    \FOR{$j = 1, \ldots, M_1$}
    \FOR{$k = 1, \ldots, M_2$}

    \STATE Get the opinion $X_k$ of a random neighbor.
    \IF{$X_j$ is not in $A$}
    \STATE Insert $(X_k, 1)$ to $A$.
    \ELSE
    \STATE $(X_k, \mrm{freq}(X_k))
    \gets (X_k,  \mrm{freq}(X_k) +1) $.
  \ENDIF
\ENDFOR
\STATE Divide all frequencies of $A$ by $M_2$.  \label{alg:line:counters}
\STATE Round all frequencies of $Α$ to the closest multiple of $1/d_i$.
\STATE $B(j) \gets \alpha_i \sum_{x \in A} \mrm{freq}(x) + (1-\alpha_i) s_i$.
\ENDFOR
\STATE $x_i(t) \gets \mrm{majority}_j B(j)$.
\ENDFOR
\end{algorithmic}
\end{algorithm}

\begin{theorem}
  Let $I = (G(V,E) ,s, a)$ be an instance of the opinion formation
  game of Definition~\ref{d:random_payoff_game} with $a > 1/2$.
  Let $d$ be the maximum degree of the graph $G$ and $n = |V|$.
  There exists an update rule
  after $O(d^2 \log^2 n \log^2(1/\eps))$ rounds
  achieves expected error
  \(
    \Exp{\norm{\infty}{x_t- x^*}}
    \leq \eps.
  \)
\end{theorem}
\begin{proof}
  According to the update rule~\ref{alg:frequencies}
  all agents fix their opinions $x_i(t)$ for $M_1 \times M_2$ rounds.
  To estimate the sum of the opinions each agent estimates the
  frequencies $k_j/d_i$. Since the neighbors have at most $d_i$
  different opinions we can map the opinions to natural
  numbers in $[d_i]$.  At each round the agent gets the opinion
  of a random neighbor and therefore the samples $X_i$ that she
  observes are drawn from a discrete distribution $P$ supported on
  $[d_i]$.
  If $k_j$ be the (absolute) frequency of the opinion $j$
  namely the number of neighbors that express $j$ as their
  opinion, then the probability $P(j)$ of opinion $j$ is
  $k_j/d_i$.  To learn the probabilities $P(j)$ using samples from $P$,
  we let $\mcal{A} = \{\{1\},\{2\},\ldots, \{d_i\}\}$ and
  use Lemma~\ref{l:vc_inequality} to get that
  \[
    \Expnew{P^m}{\max_{j \in [d_i]} \lp| \hat{P}_m(j) - P(j) \rp|}
    \leq 2 \sqrt{\frac{\log{2 d_i}}{m}},
  \]
  since $S_{\mcal{A}} \leq n$. Therefore, an agent can draw
  $m = 100 n^2 \log(2 n)$ to learn the frequencies $k_j/d_i$
  within expected error $1/(5 d)$.
  Notice now that the array $A$ after line~\ref{alg:line:counters}
  corresponds to the empirical distribution of
  equation~(\ref{eq:empirical_distribution}).
  Notice that if the agents have estimations of the frequencies $k_j/d_i$
  with error smaller than $1/d$, then by rounding them to the closest
  multiple of $1/d_i$ they learn the frequencies exactly.
  By Markov's inequality we have that with probability at least $4/5$ the
  rounded frequencies are exactly correct.
  By standard Chernoff bounds we have that if the agents repeat the
  above procedure $\ln(1/\delta)$ times and keep the most frequent of the
  answers $B(j)$, then they will obtain the correct answer with
  probability at least $1-\delta$. We know that, having computed
  the \emph{exact} average of the opinions of the neighbors $Ο(\log(1/\eps))$
  rounds are enough to achieve error $\eps$.
  Since we need all nodes to succeed at computing the exact averages for
  $Ο(\log(1/\eps))$ rounds, from the union bound we get that for
  $\delta < \frac{\eps}{n \ln(1/\eps)}$, with
  probability at least $1-\eps$ the error is at most $\eps$.
  Finally, from the law of total expectation, after
  $T = O(d^2 \log d \log(\eps) (\log(n/\eps) + \log \log(1/\eps))$ rounds
  the expected error is
  $\Exp{\norm{\infty}{x_T - x^*}} = (1-\eps) \eps + \eps \leq 2 \eps$.
\end{proof}
