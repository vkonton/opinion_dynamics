\section{Preliminaries}
\subsection{Notation}
Let $G(V,E)$ be a graph. We denote by $V$ the set of agents,
and $N_{i} \subset V$ the set of neighbors of agent $i$. Let $n = |V|$.
We denote by $x(t)\in [0,1]^n$ the vector of opinions of the agents at round
$t$ and $x_i(t)$ its $i$-th coordinate. We denote by $Q([0,1])$
the set of rationals in $[0,1]$.

\subsection{Online Convex Optimization and No Regret Algorithms}
The \emph{Online Convex Optimization} (OCO) framework can be seen
as a game played between a player and an adversary. Let
$K \subseteq \reals^n$ be a convex set and a set of functions
$\mcal{F}$ defined over $K$. \emph{At round }$t$,
\begin{enumerate}
  \item \emph{the agent chooses }$x_t \in K$.

  \item \emph{the adversary observes the }$x_t$\emph{ and selects a function}
    $f_t(x) \in \mcal{F}$.

  \item \emph{the player receives cost }$f_t(x_t)$.
\end{enumerate}


\noindent The goal of the player is to pick $x_t$ based on the history
$(f_1,\ldots,f_{t-1})$ in a way that minimizes the total cost. We emphasize
that the agent has to choose $x_t$ before seeing $f_t$, otherwise the
problem becomes trivial.
\begin{definition}
An OCO algorithm $A$, at round $t$ selects a vector $x^t \in K$ according to the history, 
$x^t = A_t(f_1,\ldots,f_t)$.
\end{definition}

\begin{definition}
Let the OCO algorithm $A$ and the sequence of functions $\{f_1,\ldots,f_T\} \in F^T$. The regret of $A$ is defined by
\[R_A(T) = \sum_{t=1}^Tf_t(x_t) - \text{min}_{x \in K}\sum_{t=1}^Tf_t(x)\]
if $R_A(T) = o(T)$ for any sequence $\{f_1,\ldots,f_T\}$ then A is no-regret algorithm 
\end{definition}
According to the feasibility set $K$ and the set of functions $F$, different no-regret algorithms with differnt regret bounds can be derived. 
Some seminal examples areZinkevic's algorithm that acheives regret $R_A(T) =\bigOh{\sqrt{T}}$, when $K$ is convex, closed and bounded and 
$F$ is the set of convex functionswith bounded first derivative. Hazan et al \cite{} proposed an algorithm with $R_A(T) =\bigOh{\log T}$, 
when $F$ is the set of twice diffentiable strongly convex functions.\\

\noindent \textbf{No regret dynamics and repeated games:}
\noindent Agent $i$ with $s_i,a_i$ is the player. At round $t$, she adopts an opinion $x_i(t) \in [0,1]$ and then the adversary selects a function $C_i^t(x)=(1-a_i)(x-y_t)^2 + a_i(x-s_i)^2$, where $y^t\in [0,1]$.
Let $A$ a no-regret in the above OCO setting. Clearly, each agent $i$ is willinf to adopt as $x_i(t)$ the suggestion of $A$, ensuring that average cost is similar
to the average cost of the best fixed opinion. As a result if all agents adopt the no-regret algorithm $A$, for a fixed instance $I$ of the game the following a dynamical process
is defined.
\begin{algorithm}
  \caption{no-regret dynamics}
  \label{alg:fno_regret_dynamics}
      Let an instance $I=(G,s,a)$ of the opinion formation game and $x_1(0),\ldots,x_n(0)$ the initial opinions.\\
      At round $t\geq 1$, each agent $i$:
  \begin{algorithmic}[1]
    \State Adopts an opinion $x_i(t)=A_t(C_i^1,\ldots,C_i^{t-1})$
    \State Meets uniformly at random one of her friends $j \in N_i$
    \State Suffers cost $C^t_i(x_i(t))$, where $C^t_i(x) = (1-a_i)(x-x_j(t))^2 + a_i(x-s_i)^2$
    %\State Updates opinion
    %\begin{equation}\label{eq:fictitious_play}
    %  x_i(t) =
     % \argmin_{x \in [0,1]}
     % \sum_{\tau=1}^tC^{\tau}_i(x,x_{\pi_i(\tau)}(\tau-1))
    %\end{equation}
\end{algorithmic}
\end{algorithm}

\noindent We denote the as $x_A(t)$ the opinion vector when the algorithm $A$ is selected. In case $A_t(C_i^1,\ldots,C_i^{t-1})= \text{argmin}_{x \in [0,1]}\sum_{\tau=1}^{t-1}C_i^t(x)$ 
we obtain the \emph{fictitious play} defined in \ref{alg:fictitious_play} and for simplicity we denote it as $x(t)$. As we have already menioned, in Section 2 we
bound the convergence rate of the \emph{fictitious play}, $\Exp{\norm{\infty}{x(t)-x^*}}$ and in Section 3, we prove that it is no-regret algorithm for the specific OCO setting. The latter
does not hold in more general OCO settings. In section 4, we investigate whether there exists an algorithm $A$ with significantly better asymptotic rate of convergence. 

%For example if the algorithm $A_t(C_i^1,\ldots,C_i^{t-1})= \text{argmin}_{x \in [0,1]}\sum_{\tau=1}^{t-1}C_i^t(x)$, we obtain the dynamics defined in \ref{alg:fictitious_play}. 
%As we have already mentioned, the above algorithm admits no-regret in the above OCO setting (see Section 3). The above dynamics converges to the unique equilibrium point $x^*$. 
%\noindent In the repeated version of the opinion formation game with random payoffs, things are clear. Each agent $i$, at round $t$ must select a strategy $x_i(t) \in [0,1]$
%and then suffers a cost $C_i(x(t))$. Assume that there exists and algorithm $A$ admits no-regret when $K=[0,1]$ and $F$ is the set of functions $f(x) = (1-a_i)(x-y_t)^2 + a_i(x-s_i)^2$
%for all $c \in [0,1]$. Each agent $i$ is willing to adopt the suggestion of $A$, $x_i(t) = A_t(C_i^1,\ldots,C_i^{t-1})$ since it ensures him similar cost as the that
%of the best opinion fixed in hindsight. For such an algorithm $A$ we define the following dynamical system.



%\begin{definition}\label{d:regret}
%  $A$ for OCO is a sequence of functions, $\{A_t\}_{t=1}^{\infty}$ where
%  $A_t:\mcal{F}^{t-1} \mapsto K$.\\
%  %\[x_t = A_t(f_1,\ldots,f_{t-1})\].
%  The regret of the algorithm $A$ as round $T$ is
%  \[R_A(T)=\sup_{\{f_1,\ldots ,f_T\}\subseteq F^T}\lp(\sum_{t=1}^T f_t(x^t) -
%    \min_{x\in K}\sum_{t=1}^T f_t(x)\rp),\]
%  where $x^t = A^t(f_1,\ldots, f_{t-1})$.
%  An algorithm $A$ is no-regret if $R_A(T)=o(T)$.
%\end{definition}
%In our problem agent $i$ selects at round $t$ an
%opinion $x_i(t) \in [0,1]$ and then suffers a cost $C^t_i(x_i(t))$.
%Obviously in our case $K=[0,1]$ and
%\(
%  \mcal{F} =
%  \{
%  x \mapsto a_i (x - s_i)^2 + (1-a_i) (x - c)^2:
%  \ a_i, s_i, c \in [0,1]
%  \}
%\).

%\begin{definition}[no-regret Dynamics] \label{d:noregret_dynamics}
 % Let $A$ be a no-regret algorithm for $\mcal{F}$, $K$.
 % For an instance $I = (G, s, a)$ of the opinion formation game
 % given in Definition~\ref{d:random_payoff_game} we define
 % the following no-regret dynamics.
 % \begin{itemize}
 %   \item Initially, each agent $i$ has an opinion $x_1(0),\ldots,x_n(0)$.

 %   \item At round $t \geq 1$, each agent $i$:
 %     \begin{itemize}
%	\item Meets uniformly at random one of her friends $j \in N_i$.
 %       \item  Suffers cost
  %        \(
   %         C^t_i(x_i(t-1),x_{j}(t-1))=(1-a_i)(x_i(t-1)
    %        -x_j(t-1))^2 + a_i(x_i(t-1)-s_i)^2
%          \).
 %       \item Updates her opinion using $A$,
 %         $x_i(t) = A_t(C^1_i, \ldots C^{t-1}_i)$.
 %     \end{itemize}
 % \end{itemize}

%\end{definition}

\subsection{Estimating Bernoulli distributions}

Assume that we have access to the random variables $Y_1,\ldots,Y_t$, that are
mutually independent and each
$Y_i$ is distributed according to a Bernoulli distribution with $p \in [0,1]$,
$X_i\sim B(p)$.  Initially the parameter $p$ is unkown. The task is to observe
the realization of $y_1,\ldots,y_t$ and output an estimate
$\hat{p}=\theta_t(y_1,\ldots,y_t)$ according to a function $\hat{\theta_t}:
\{0,1\}^t\mapsto [0,1]$, that is close to the unkown parameter $p$.

\begin{definition}
  An estimator $\hat{\theta}$ is a collections of function,
  $\{\hat{\theta_t}\}_{t=1}^{\infty}$, where
  $\hat{\theta_t}: \{0,1\}^t\mapsto [0,1]$.
\end{definition}

A well known example of such an estimator is the
\emph{sample mean}, where $\hat{\theta_t}=\frac{\sum_{i=1}^tY_i}{t}$.
\noindent Obviously for any efficient estimator, the estimate $\hat{p}$
converges to the unkown parameter $p$ as $t$ grows. The following definition is
the standard metric for the efficiency of an estimator $\hat{\theta}$.

\begin{definition}
  For an estimator $\hat{\theta}
  =\{\hat{\theta_t}\}_{t=1}^\infty$ we define its  error rate $R_p(t) =
  E_p[|\hat{\theta_t}(Y_1,\ldots,Y_t) - p|]$ \[\text{where
    }E_p[|\hat{\theta_t}(Y_1,\ldots,Y_t) - p|]= \sum_{(y_1,\ldots,y^t)\in
      \{0,1\}^t}|\hat{\theta_t}(y^1,\ldots,y^t) -p| \cdot
    p^{\sum_{i=1}^ty^i}(1-p)^{t-\sum_{i=1}^ty^i}\]
\end{definition}

The quantity $E_p[|\hat{\theta_t}(Y_1,\ldots,Y_t) - p|]$ is the expected distance
of the estimated value $\hat{\theta_t}$ from the parameter $p$, when the
distribution of the samples is $B(p)$. To simplify notation we also denote it
as $E_p[|\hat{\theta_t} - p|]$. The error rate $R_p(t)$ quantifies the rate of
convergence of the estimated value $\hat{p} =\theta_t(Y_1,\ldots,Y_t)$ to the
real parameter $p$. \\ Since $p$ is unkown, any meaningful estimator $\hat{p}$
must guarantee that for all $p \in [0,1]$, $\lim_{t \rightarrow \infty}
R_p(t)=0$. For example the \emph{sample mean} has error rate $R_p(t) \leq
\frac{1}{2\sqrt{t}}$ for ant $p \in [0,1]$ and clearly statisfies the above
requirement. In section 4 we investigate the following question,

\begin{question}
  Is there an estimator $\hat{\theta}$ such that for all $p \in
  Q[0,1]$, $\lim_{t \rightarrow \infty}t R_p(t)=0$?
\end{question}

%\noindent Notice that $E_p[|\hat{\theta_t}-p|]$ does not serve as a good
%metric since an effiencient estimator $\hat{\theta}$ must garantee good
%estimates for any parameter $p$. Consider an estimator $\hat{\theta}$ that
%outputs $1/2$ for all $t$. In this case, $E_{p}[|\hat{\theta_t}- p|]=|p-1/2|$
%meaning that this estimator performs optimally when $p=1/2$, but extremely bad
%in any other case. As a result the standard metric in the statistics
%literature is the following.

%\begin{definition} For any estimator $\hat{\theta_t}$ we define its risk with
%$t$ samples as, $$R^t(\hat{\theta})=\text{max}_{p \in
%[0,1]}E_p[|\hat{\theta_t} - p|]$$ \end{definition}Now our goal is clear, we
%need to design estimators $\hat{\theta_t}$ such that the risk
%$R^t(\hat{\theta_t})$ is as small as possible. For example one can easily
%prove that the risk rate of the \emph{sample mean estimator} ($\hat{\theta_t}=
%\frac{\sum_{i=1}^t X_i}{t}$) $R^t(\hat{\theta_t}) \leq \frac{1}{\sqrt{2t}}$.
%Obviously the next question is whether we can find another (probably more
%complex estimator) that acheives risk smaller than that of
%$\frac{1}{\sqrt{2t}}$.

%\begin{theorem}\label{thm: risk_optimality} Any estimator $\hat{\theta}$ has
%risk $R^t(\hat{\theta}) \geq \frac{1}{16\sqrt{t}}$ \end{theorem} The above
%theorem can be easily derived by standard techniques (e.g. Le Cam or Fano
%method) that have been develloped in the statistics literature to provide
%lower bounds on the risk of estimators. Especially, proving this lower bound
%for the Bernoulli case is requires a simple applications of these methods. The
%above theorem also tells us that the risk of the \emph{sample mean estimator}
%is up to constant factors optimal. Are we done?

%\begin{theorem} There exists an estimator $\hat{\theta}$ such that for all
%$p\in[0,1]$, $$E_p[|\hat{\theta_t} - p|] = \bigOh{\frac{1}{\sqrt{t}}}$$
%\end{theorem} Can this rate be imporoved? For example, is there a an estimator
%$\hat{\theta}$ such that for all $p\in [0,1]$, $E_p[|\hat{\theta_t} - p|] =
%\bigOh{\frac{1}{t}}$?\\

%\noindent It seems that the latter question can be answered negatively using
%Theorem~\ref{thm: risk_optimality}. Unfortunately, this is not the case. The
%difference is subtle but very important. The definition of the risk $R^t =
%\text{max}_{p \in [0,1]}E_p[|\hat{\theta_t}-p|]$, allows the maximum to be
%attained in a $p$ that depends on $t$ (e.g. $p=1/2+1/t$). As a result,
%Theorem~\ref{thm: risk_optimality} does not provide us with information about
%the above question.


%\begin{theorem} For any estimator $\hat{\theta}$ there exists a fixed
%$p\in[0,1]$ such that $$E_p[|\hat{\theta_t} - p|] = \Omega(\frac{1}{t})$$
%\end{theorem}

%\noindent To the best of our knowledge this question is not answered in the
%statistics literature. However, the same question has been asked for more the
%estimator of more general and complex distribution. These results ensure the
%above statement for estimators that satisfy certain propetries (unbaised
%estimators, locally regular, e.t.c.). In this section we provide a theorem
%that holds for any estimator and indicates that estimators that don't satisfy
%the above statement are very unlikely to exists.

%\begin{theorem} For any Bernoulli estimator $\hat{\theta}$, for all $[a,b]
%\subseteq [0,1]$, $$ \lim_{t \to \infty}t \int_{a}^{b}E_p[|\hat{\theta_t}
%-p|]dp = +\infty$$ \end{theorem}
