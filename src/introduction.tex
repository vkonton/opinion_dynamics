\section{Introduction}
% \subsection{Friedkin-Johsen Model}

\subsection{Friedkin-Johsen Model and Opinion Formation Games}
In the Friedkin-Johsen model (FJ-model) an undirected graph $G(V,E)$ with $n$
nodes, is assumed, where $V$ denotes the agents and $E$ the social relations
between them. Each agent $i$ poses an internal opinion $s_i \in [0,1]$ and a self
confidence coefficient $\alpha_i \in[0,1]$. At each round $t \geq 1$,
agent $i$ updates her opinion as follows:

\begin{equation}\label{eq:fj_dynamics}
  x_i(t) = (1-\alpha_i) \frac{\sum_{j \in N_i}x_j(t-1)}{|N_i|} + \alpha_i s_i,
\end{equation}
where $N_i$ is the set of her neighbors. The simplicity of the update rule
(\ref{eq:fj_dynamics}) makes the FJ-model plausible, because in real social
networks it is very unlikely that agents change their opinions according
to complex rules.
% \subsection{Opinion Formation Games}
Based on the FJ-model, in \cite{BKO11}, they propose an \emph{
  opinion formation game} in which the strategy that each agent
$i$ plays, is the opinion $x_i \in [0,1]$
that she publicly expresses incurring her a cost

\begin{equation}\label{eq:kleinberg_cost}
  C_i(x_i,x_{-i})=(1-a_i)\sum_{j \in N_i}(x_i -x_j)^2 + a_i(x_i-s_i)^2.
\end{equation}

In \cite{BK011} they prove that it always admits a unique
Nash Equilibrium $x^*$ and has Price of Anarchy $9/8$ if $G$ is
undirected and $O(n)$ in the directed case.  We denote an instance of this
game as $I=(G,s,a)$.
The FJ-model can be viewed as the best-response dynamics of the
of the opinion formation game.
% j
%
% \begin{equation}
%   C_i(x_i(t-1),x_{-i}(t-1))=(1-a_i)\sum_{j \in N_i}(x_i(t-1) -x_j(t-1))^2 +
%   a_i(x_i(t-1)-s_i)^2
% \end{equation}
More precisely, each agent $i$ has an opinion $x_1(0),\ldots,x_n(0)$ and
at round $t \geq 1$, suffers the cost (\ref{eq:best_response})
and updates her opinion so as to minimize her individual cost

\begin{equation}\label{eq:best_response}
  x_i(t) =
  \argmin_{x \in [0,1]}
  (1-a_i) \sum_{j \in N_i} (x - x_j(t-1))^2 + a_i (x -s_i)^2
\end{equation}

In \cite{GS14} it is proved that for any instance $I=(G,s,a)$ the opinion
vector $x(t)=(x_1(t),\ldots,x_n(t))$ converges to the unique equilibrium point
$x^*$. Moreover, this is achieved by an update rule that is very simple
but more importantly it is \emph{rational}, in the sense
that each agent $i$ adopts this rule in order to minimize her individual cost.

\subsection{Opinion Formation Games with Random Payoffs}
In the game defined by (\ref{eq:kleinberg_cost}),
agent's $i$ cost $C_i(x_i,x_{-i})$ is a deterministic function of the
opinion vector $x$. Many recent works (see e.g. \cite{Zhou17}) study games
with random payoffs, that is agent's $i$ cost ($C_i(x_i,x_{-i})$) is a
random variable. The random payoff setting can be much more realistic,
since randomness may naturally occur because of incomplete information, noise
or other stochastic factors. Motivated by this
line of research we introduce a random payoff variant of the opinion
formation game (\ref{eq:kleinberg_cost}).

\begin{definition}\label{d:random_payoff_game}
  Let $I=(G,s,a)$ an instance of the opinion formation game
  and $x$ the opinion vector. Each agent $i$,
  \begin{itemize}
    \item picks uniformly at random one of her neighbors $j \in N_i$
    \item suffers cost
      $C_i(x_i,x_{-i}) = (1-a_i)\sum_{j \in N_i}(x_i-x_j)^2 + a_i(x_i-s_i)^2$
  \end{itemize}
\end{definition}

This game is more compatible to a realistic settings since
in real world networks (e.g. Facebook, Twitter e.t.c.), each agent may have
several hundreds of friends. As a result is far more reasonable to assume that
every day, each agent meets a random small subset of her acquaintances and
suffers a cost based on how much she disagrees with them.
Since the expected cost of our random payoff variant equals
the cost of game $(\ref{eq:kleinberg_cost})$, $ \Exp{C(x_i, x_{-i})} =
(1-a_i)\sum_{j \in N_i}(x_i-x_j)^2 + a_i(x_i-s_i)^2$,
the results about the existence of a unique equilibrium,
and the PoA bounds also hold in our variant.
However, the best-response dynamics in the random payoff game
does not converge to the equilibrium.  In this work, we investigate
whether there exists natural dynamics that leads the system to the
equilibrium point $x^*$.

\subsection{Our Results and Techniques}
% \begin{itemize}
%   \item Initially, each agent $i$ has an opinion $x_1(0),\ldots,x_n(0)$.

%   \item At round $t \geq 1$, each agent $i$ meets uniformly at random one of
%     her friends $j \in N_i$, suffers cost
%     \(
%       C^t_i(x_i(t-1),x_{j}(t-1))=(1-a_i)(x_i(t-1)
%       -x_j(t-1))^2 + a_i(x_i(t-1)-s_i)^2
%     \)
%     and updates her opinion
%     \begin{equation}\label{eq:fictitious_play}
%       x_i(t) =
%       \argmin_{x \in [0,1]}
%       \sum_{\tau=1}^tC^{\tau}_i(x,x_{\pi_i(\tau)}(\tau-1))
%       % =
%       % (1-a_i)\frac{\sum_{\tau=1}^t x_{\pi_i(\tau)}(t-1)}{t} + a_is_i,
%     \end{equation}

%     where $\pi_i(\tau)$ denotes the index of the index of the neighbor that $i$
%     selected at round $\tau$.

% \end{itemize}
In this work we propose Algorithm~\ref{alg:fictitious_play} as a dynamics
for the random payoff game of Definition~\ref{d:random_payoff_game}.
Note that the dynamics described in Algorithm~\ref{alg:fictitious_play} is the
\emph{fictitious play} in the game defined by the instance
$I=(G,s,a)$. Generally speaking \emph{fictitious play} does not guarantee
convergence to the equilibrium.
\begin{algorithm}
  \caption{Fictitious Play}
  \label{alg:fictitious_play}
  \begin{algorithmic}[1]
    \State Initially, each agent $i$ has an opinion $x_1(0),\ldots,x_n(0)$.
    \State At round $t \geq 1$, each agent $i$ meets uniformly at random one of
    her friends $j \in N_i$
    \State Suffers cost
    \(
      C^t_i(x_i(t-1),x_{j}(t-1))=(1-a_i)(x_i(t-1)
      -x_j(t-1))^2 + a_i(x_i(t-1)-s_i)^2
    \)
    \State Updates opinion
    \begin{equation}\label{eq:fictitious_play}
      x_i(t) =
      \argmin_{x \in [0,1]}
      \sum_{\tau=1}^tC^{\tau}_i(x,x_{\pi_i(\tau)}(\tau-1))
    \end{equation}
\end{algorithmic}
\end{algorithm}
In Section~\ref{s:fictitious_convergence} we show that, for our game with
random payoffs, fictitious play converges to equilibrium, with the following
rate.
\begin{theorem}
  Let $I = (G,s, \alpha)$ be any instance of the opinion formation
  game of Definition~\ref{d:random_payoff_game} with equilibrium
  $x^* \in [0,1]^n$.  The opinion vector $x^t$ produced by
  Algorithm~\ref{alg:fictitious_play} after $t$ rounds satisfies
  \[
    \Expnew{}{\norm{\infty}{x^t - x^*}} \leq
    C \sqrt{\log n}\frac{(\log t)^{2}}{t^{\min(1/2,\rho)}},
  \]
  where $\rho = \min_{i \in V} a_i$ and $C$ is a universal constant.
\end{theorem}
The update rule (\ref{eq:fictitious_play}) guarantees convergence
while vastly reducing the information exchange between the agents
at each round.
Namely, each agent learns the opinion of only one of her neighbors
whereas in the classical FJ-model (\ref{eq:fj_dynamics}) each agent
learns the opinions of all of her neighbors. In terms of
total communication needed to get within distance $\eps$ of the
equilibrium, the update rule (\ref{eq:fictitious_play}) needs
$O(|V| \log |V|)$ communication while (\ref{eq:fj_dynamics}) needs
$O(|E|)$. Of course for this difference to be significant we need
each agent to have at least $O(\log |V|)$ neighbors.  A large social
network like Facebook has approximately $2$ billion users and each user
has usually more that $100$ friends which far more than $\log(2\ 10^9)$.

Apart from converging to the equilibrium, our update rule
(\ref{eq:fictitious_play}) is also a rational behavioral
assumption since it ensures \emph{no-regret} for the agents.
Having no-regret means that the average cost for each agent $i$
after $T$ rounds is close to the average cost that she would
suffer by expressing any fixed opinion. This is a very important
feature of our update rule because even players that selfishly
will to minimize their incurred cost, could choose to play according
to it. In Section~\ref{s:fictitious_noregret} we show the following
theorem.
\begin{theorem}\label{t:fictitious_noregret}

  For every instance $I=(G,s,a)$, for every agent $i$
  $$\sum_{\tau=1}^tC_i^\tau(x_i(\tau)) \leq \text{min}_{x \in
    [0,1]}\sum_{\tau=1}^tC_i^\tau(x) + \bigOh{\log t}$$

\end{theorem}

Even though our update rule (\ref{eq:fictitious_play}) has the above
desired properties it only achieves convergence rate of
$\widetilde{O}(1/\sqrt{t})$ for a fixed instance $I$ with
$\rho>1/2$.  For such an instance the original FJ-model outperforms
our update rule since it achieves convergence rate $O(1/2^{t})$.
We investigate whether this gap is due to an inherent characteristic
of the random payoff setting i.e. learning the opinion of just one random
neighbor vs learning all of them or we could find another natural dynamics
that converges exponentially fast. Can the agents select another no-regret
algorithm that guarantees an asymptotically faster convergence rate?
In Section~4 we investigate the following question

\begin{question}
  Is there a no-regret algorithm that the agents can choose such that
  for any instance $I = (G, s, a)$, $\Exp{\norm{\infty}{x_t - x^*}} = O(1/t)$ ?
\end{question}

To answer this question we first show that the existence of a no-regret
algorithm (that achieves this convergence rate to the equilibrium) implies
the existence of an algorithm that uses i.i.d samples from a Bernoulli random
variable $B(p)$ to estimate its success probability $p$ with the same asymptotic
error rate. We use sample complexity lower bound techniques from statistics
to show that such a no-regret algorithm is unlikely to exist.
We show that updating the opinion according to no-regret algorithms leads
to update rules that are totally ignorant of the specific graph structure $G$.
Perhaps surprisingly, in Section~\ref{s:graph_aware} we present an update rule
that takes into account only the maximum degree of the graph and for any
fixed instance $I$ achieves convergence rate $O(1/2^{\sqrt{t}})$.


% In section 4, we investigate where a better convergence rate,
% $E[||x(t)-x^*||_{\infty}]$ can be acheived if agent selected another no-regret
% algorithm. More precisely, we investigate the following question, \emph{Is
%   there a no-regret algorithm such that for every instance }
% $I,~E[||x(t)-x^*||_{\infty}] \in \bigOh{\frac{1}{t}}?$  We reduce this question
% to the following question in statistical estimation, \emph{Is there a Bernoulli
%   estimator }$\hat{\theta}$\emph{, such that for all }$q \in [0,1]$, $\lim_{t
%   \rightarrow \infty} t E_q[|\hat{\theta^t} -q|]=0?$  To the best of our
% knowledge this question is not answered in statistics literature. However, we
% use standard tecnhiques for lower bounds in the statistical estimation to prove
% the following theorem.

% \begin{theorem}
%   For any Bernoulli estimator
%   $\hat{\theta}$, for all $[a,b] \subseteq [0,1]:~$ $\lim_{t \rightarrow
%     \infty}t \int_{a}^bE_p[|\hat{\theta^t} -p|]= +\infty$
% \end{theorem}
% This result indicates that the second question and consequently the
% first is very unlike to hold, especially for any reasonable no-regret
% algorithm. We believe that there exists no such algorithm and we leave this
% proof as an open problem.\\

% Finally in section 5, we present an interesting side result. The reason that
% there does not exists a no-regret algorithm, that ensures faster convergence
% rate, is the algorithm's ignorance to the instance $I$ that defines the
% dynamics. More formally, a no-regret algorithm does needs the values $s_i,a_i$
% and $Y_1,\ldots,Y_t$ to determine $x_i(t)$. We find interesting that we design
% a distributed algorithm that uses $s_i,a_i,Y_1,\ldots,Y_t$ and additionally the
% $d_\text{max}$ of $G$ that achieves for every instance $I$ convergence rate,
% $E||x(t)-x^*||_{\infty} \in \bigOh{2^{-\frac{\sqrt{t}}{d^3}}}$.

\subsection{Related Work}
