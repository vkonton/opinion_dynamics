We study opinion formation games based on the famous model proposed by Friedkin
and Johsen.  In today's huge social networks the assumption that in each round
agents update their opinions by taking into account the opinions of \emph{all}
their friends could be unrealistic. Therefore, we assume that in each round
each agent gets to meet with only one random friend of hers. Since it is more
likely to meet some friends than others we assume that agent $i$ meets agent
$j$ with probability $p_{ij}$.  In this imperfect information setting, we are
interested in simple and natural variants of the FJ model that converge to the
same equilibrium point $x^*$.  Specifically, we define an opinion formation
game, where at round $t$, agent $i$ with intrinsic opinion $s_i\in[0,1]$ and
expressed opinion $x_i(t) \in[0,1]$ meets with probability $p_{ij}$ neighbor
$j$ with opinion $x_j(t)$ and suffers a disagreement cost that is a convex
combination of $(x_i(t) - s_i)^2$ and $(x_i(t) - x_j(t))^2$.  We show that the
agents can adopt an intuitive and simple update rule that ensures
\emph{no regret} to the experienced disagreement cost and at the same time the
produced opinion vector converges to the equilibrium $x^*$ within error $\eps$
after roughly $O(\mrm{poly}(\log n /\eps))$ rounds, where $n$ is the number of
agents.  In our imperfect information setting, we prove that, for dynamics that
ensure no regret for the agents, $\Omega(1/\eps)$ rounds are also necessary.
Finally, we present an update rule that requires only
$\widetilde{O}(\log^2(1/\eps))$ rounds to acheive error $\eps$,
resembling the convergence of the original FJ model.  Slow convergence is not
a generic property of our imperfect information setting but rather a
characteristic of no regret dynamics.
