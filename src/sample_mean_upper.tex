\section{Constant Memory with full window}

In this work we investigate an variant of the above process. Namely we assume that at each time $t$, each agent $i$ picks uniformly at random one of her neighbors and updates her opinion as follows: $$x_i(t) = (1-\alpha_i)\frac{\sum_{\tau=1}^tx_{\pi_i(\tau)}(\tau-1)}{t} + \alpha_i s_i$$ $\pi_i(\tau)$ denotes neighbor that has selected at $\tau$ and $x_{\pi_i(\tau)}(\tau-1)$ her opinion that given time. In this word we prove that the above process converges arbitrarily close with arbitrarily high probability to the same equilibrium point $x^*$.
We denote with $d$ the maximum degree of $G(V,E)$ ($d=\text{max}_{i \in V}d_i$) and $x^t \in [0,1]^n$ the opinion vector of the agents at $t$.

We start by stating the standard Hoeffding bound

\begin{lemma}[Hoeffding's Inequality]\label{l:hoeffding}
  Let $X = (X_1 +\cdots + X_t) /t$, $X_i \in [0,1]$.
  Then,
  \[
    \Prob{|X - \Exp{X} | > \lambda} < 2 e^{-2 t \lambda^2}.
  \]
\end{lemma}

\begin{lemma}
With probability at least $1-p$, $\|x^t - x^*\|_{\infty} \leq e(t)$, where $e(t)$ satisfies the following recursive relation
$$e(t) = \delta(t) + (1-\alpha)\frac{\sum_{\tau=0}^{t-1}e(\tau)}{t} \text{ and } e(0)=\|x^0 - x^*\|_{\infty}$$
and $\delta(t)=\sqrt{\frac{\log(\frac{n}{p})}{t}}$.
\end{lemma}
\begin{proof}
Using Lemma~\ref{l:hoeffding} we have
\[
  \Prob
  {
    \lp|
    \frac{\sum_{\tau = 1}^{t} x^*_{\pi_i(\tau)}}{t}
    - \frac{\sum_{j \in N_i} x^*_j}{|N_i|} \rp|
    > \sqrt{
      \frac{\log(n/p)}{t}
    }
  } < \frac{2 \me^{-2} p}{n} < \frac{p}{n}
\]
Therefore, by the union bound we have
\[
  \Prob
  {
    \max_{i \in V}
    \lp|
    \frac{\sum_{\tau = 1}^{t} x^*_{\pi_i(\tau)}}{t}
    - \frac{\sum_{j \in N_i} x^*_j}{|N_i|} \rp|
    > \sqrt{
      \frac{\log(n/p)}{t} }
  } < p
\]
We assume that $\norm{\infty}{x^{\tau}-x^*} \leq e(\tau)$ for all $\tau \leq t$.
To simplify notation, let $\delta(t) =
\sqrt{ \log(n/p)/t} $.
Now with probability $p$ we have that for all agents it holds
\begin{align*}
x_i(t)
&= (1-\alpha_i)\frac{\sum_{\tau=1}^{t}x_{\pi_i(\tau)}(\tau-1)}{t} + \alpha_i s_i\\
&\leq (1-\alpha_i)\frac{\sum_{\tau=1}^{t}x^*_{\pi_i(\tau)} + \sum_{\tau=1}^{t} e(\tau-1)}{t} + \alpha_i s_i\\
&\leq (1-\alpha_i)\frac{\sum_{\tau=1}^{t}x^*_{\pi_i(\tau)} + \sum_{\tau=0}^{t-1} e(\tau)}{t} + \alpha_i s_i\\
&\leq  (1-\alpha_i)\lp(\frac{\sum_{j \in N_i}x^*_{j}}{|N_i|} + \delta(t) + \frac{\sum_{\tau=0}^{t-1} e(\tau)}{t} \rp) + \alpha_i s_i\\
&\leq x_i^* + (1-\alpha) \lp(\delta(t) + \frac{\sum_{\tau=0}^{t-1} e(\tau)}{t} \rp)
\end{align*}
Equivalently, we can prove that $x_i(t+1) \geq x_i^* - \delta(t) - (1-\alpha)\frac{\sum_{\tau=1}^t e(\tau)}{t}$.
As a result $||x_i(t)-x^*||_{\infty} \leq e(t)$.
\end{proof}

\begin{corollary}
The function $e(t)$ satisfies the following recursive relation
$$e(t+1)-e(t) + \alpha \frac{e(t)}{t+1}=\delta(t+1)-\delta(t) + \frac{\delta(t)}{t+1}$$
\end{corollary}
In order to bound the convergence time of the system, we just need to bound the convergence rate of the function $e(t)$. One can easily prove that $e(t)$ tends to $0$, however finding the explicit formula for $e(t)$ is a quite hard task. The following lemma provides us with a simpler upper bound for the convergence rate of our process.

\begin{lemma}
$$e(t) \leq \frac{e(0)}{t^\alpha} + \frac{O(\sqrt{\log(\frac{nd}{p})})}{t^\alpha}\sum_{\tau=1}^t\tau^\alpha\frac{\sqrt{\log \tau}}{\tau^{3/2}}$$
\end{lemma}

\begin{proof}
At first since $\frac{\delta(t)}{t}$ is a decreasing function, $e(t)\leq (1-\frac{\alpha}{t}) + g(t)$, where $g(t)=\frac{\delta(t)}{t}$.
\begin{align*}
 e(t)
 &\leq (1-\frac{\alpha}{t})e(t-1) + g(t)\\
 &\leq (1-\frac{\alpha}{t})(1-\frac{\alpha}{t-1})e(t-2) + (1-\frac{\alpha}{t})g(t-1) + g(t)\\
 &\leq (1-\frac{\alpha}{t})\cdots (1-\alpha)e(0) + \sum_{\tau=1}^t g(\tau)\Pi_{i=\tau+1}^t(1-\frac{\alpha}{i})\\
 &\leq \frac{e(0)}{t^\alpha} + \sum_{\tau=1}^tg(\tau)e^{-\alpha\sum_{i=\tau+1}^t\frac{1}{i}}\\
 &\leq \frac{e(0)}{t^\alpha} + \sum_{\tau=1}^tg(\tau)e^{-\alpha(H_t-H_{\tau})}\\
 &\leq \frac{e(0)}{t^\alpha} + e^{-\alpha H_t}\sum_{\tau=1}^tg(\tau)e^{\alpha H_{\tau}}\\
 &\leq \frac{e(0)}{t^\alpha} + \frac{O(\sqrt{\log(\frac{nd}{p})})}{t^\alpha}\sum_{\tau=1}^t\tau^\alpha\frac{\sqrt{\log \tau}}{\tau^{3/2}}\\
\end{align*}

\end{proof}

\begin{lemma}
\[
e(t) =
     \begin{cases}
       O(\sqrt{\log(\frac{nd}{p})}\frac{(\log t)^{3/2}}{t^\alpha}) &\quad\text{if } \alpha \leq 1/2\\
       O(\sqrt{\log(\frac{nd}{p})}\frac{(\log t)^{3/2}}{t^{1/2}}) &\quad\text{if } \alpha > 1/2\\
       \end{cases}
\]

\end{lemma}
\begin{proof}
We observe that $\sum_{\tau=1}^t\tau^\alpha\frac{\sqrt{\log \tau}}{\tau^{3/2}} \leq \int_{\tau=1}^t \tau^\alpha\frac{\sqrt{\log \tau}}{\tau^{3/2}}d\tau$ since $\tau^\alpha\frac{\sqrt{\log \tau}}{\tau^{3/2}}$\\
If $\alpha\leq 1/2$ then $\int_{\tau=1}^t \tau^\alpha\frac{\sqrt{\log \tau}}{\tau^{3/2}}d\tau \leq \int_{\tau=1}^t \tau^{1/2}\frac{\sqrt{\log \tau}}{\tau^{3/2}}d\tau = O((\log t)^{3/2}))$\\

\noindent If $\alpha> 1/2$ then\\

\begin{eqnarray*}
\int_{\tau=1}^t \tau^\alpha\frac{\sqrt{\log \tau}}{\tau^{3/2}}d\tau &=& \int_{\tau=1}^t \tau^{\alpha-1/2}\frac{\sqrt{\log \tau}}{\tau}d\tau\\
&=& \frac{2}{3} \int_{\tau=1}^t \tau^{\alpha-1/2}((\log \tau)^{3/2})'d\tau\\
&=& \frac{2}{3}(\log t)^{3/2} - (\alpha-1/2)\frac{2}{3} \int_{\tau=1}^t \tau^{\alpha-3/2}(\log \tau)^{3/2}d\tau\\
&\leq& O(\log t)^{3/2})
\end{eqnarray*}

%$\int_{\tau=1}^t \tau^\alpha\frac{\sqrt{\log \tau}}{\tau^{3/2}}d\tau \leq \int_{\tau=1}^t \tau^{1/2}\frac{\sqrt{\log \tau}}{\tau^{3/2}}d\tau = O((\log t)^{3/2}))$

\end{proof}
