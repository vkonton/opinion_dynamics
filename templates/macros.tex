% MACROS.

%display style in FRAC
\newcommand\ddfrac[2]{\frac{\displaystyle #1}{\displaystyle #2}}


\newcommand{\lp}{\left}
\newcommand{\rp}{\right}
\newcommand{\mrm}[1]{\mathrm{#1}}
\newcommand{\mbb}[1]{\mathbb{#1}}
\newcommand{\mcal}[1]{\mathcal{#1}}
\newcommand{\mfrak}[1]{\mathfrak{#1}}
\DeclareMathOperator*{\argmin}{argmin}
\DeclareMathOperator*{\argmax}{argmax}
\DeclareMathOperator*{\maj}{maj}

%Tikz crap.
\newcommand*{\equal}{=}
\newcommand*{\comma}{,}

% Asymptotic Notation.
\newcommand{\bigOh}[1]{O\lp(#1\rp)}
\newcommand{\smalloh}[1]{o\lp(#1\rp)}
\newcommand{\bigOmega}[1]{\Omega\lp(#1\rp)}
\newcommand{\smallomega}[1]{\omega\lp(#1\rp)}
\newcommand{\poly}{\mrm{poly}}

% Standard sets
\newcommand{\R}{\bm{\mrm{R}}}
\newcommand{\N}{\bm{\mrm{N}}}
\newcommand{\Q}{\bm{\mrm{Q}}}
\newcommand{\Z}{\bm{\mrm{Z}}}

% Constants.
\newcommand{\me}{\mathrm{e}}

% Analysis.
\def\d{\mrm{d}}
\newcommand\Ball[2]{\mcal{B}\lp(#1, #2 \rp)}
\newcommand\norm[2]{\| #2 \|_{#1}}

% Functions.
\newcommand{\tr}{\mathrm{tr}}
\newcommand{\1}[1]{\bm{1}{\left[#1\right]}}
\newcommand{\Real}[1]{\mathrm{Re}\lp(#1\rp)}
\newcommand{\Imag}[1]{\mathrm{Im}\lp(#1\rp)}
\DeclarePairedDelimiter\ceil{\lceil}{\rceil}
\DeclarePairedDelimiter\floor{\lfloor}{\rfloor}

% Distances & Probability.
\newcommand{\Normal}[2]{\mcal{N}\lp(#1,#2\rp)}
\newcommand{\DNormal}[2]{\mrm{DN}\lp(#1,#2\rp)}
\newcommand{\GBD}{\mrm{GBD}}
\newcommand{\PBD}{\mrm{PBD}}
\newcommand{\Bin}{\mrm{Bin}}
\newcommand{\Pois}{\mrm{Pois}}
\newcommand{\Distr}{\mathcal{L}}
\newcommand{\Unif}{\mrm{U}}

% \newcommand{\GBD}[1][]{
%   \def\ArgI{{#1}}%
%   \GBDRelay
% }
% \newcommand{\GBDRelay}[1]{
%   \mrm{GBD}\lp( \ArgI, #2 \rp)
% }

\newcommand{\NormalPDF}[3]
{
  \frac{1}{\sqrt{2 \pi} #2} \me^{-\frac{(#3 - #1)^2}{2 {#2}^2}}
}
\newcommand{\erf}[1]{\mrm{erf}\lp(#1\rp)}
\newcommand{\erfc}[1]{\mrm{erfc}\lp(#1\rp)}

\newcommand{\Prob}[1]{\bm{\mrm{P}}\lp[#1\rp]}

\newcommand{\Exp}[1][]{
  \def\ArgI{#1}%
  \ExpRelay
}
\newcommand{\ExpRelay}[1]{
  \bm{\mrm{E}}_{\ArgI} \lp[ #1 \rp]
}

\newcommand{\Expnew}[2]{
  \bm{\mrm{E}}_{#1} \lp[ #2 \rp]
}


\newcommand{\Var}[1]{\bm{\mrm{V}}\lp[#1\rp]}
\newcommand{\Cov}[2]{\mrm{Cov}\lp[#1, #2\rp]}
\newcommand{\Kurt}[1]{\mrm{Kurt}\lp[#1 \rp]}

\newcommand{\fdiv}[3]{D_{#1} \lp(#2\|#3\rp)}
\newcommand{\dkl}[2]{D_{\mrm{kl}} \lp(#1\|#2\rp) }
\newcommand{\dtv}[2]{d_{\mrm{tv}} \lp(#1, #2\rp) }
\newcommand{\dhel}[2]{d_{\mrm{hel}} \lp(#1, #2\rp) }
\newcommand{\dkol}[2]{d_{\mrm{kol}} \lp(#1, #2\rp) }

% minimax risk
\newcommand{\minimax}[1]{\mfrak{M}_{#1}}

\newcommand{\h}[1]{\hat{#1}}

% PBD powers specific macros.
\def\eps{\varepsilon}
\def\p{\hat{p}}
\def\q{\hat{q}}

\def\reals{\bm{\mrm{R}}}
\def\nats{\mathbb{N}}
\def\D{\mathcal{D}}

\def\P{\mathcal{P}}
\def\p{\hat{p}}
\def\q{\hat{q}}
\newcommand{\err}[1]{\mbox{\rm err}\left(#1\right)}


% other.

\newcommand{\sline}[1]{\textcolor{#1}{\linethickness{.5mm}\line(1,0){15}} }

% Ident in algorithmic.
\newlength\myindent
\setlength\myindent{2em}
\newcommand\bindent{%
  \begingroup
  \setlength{\itemindent}{\myindent}
  \addtolength{\algorithmicindent}{\myindent}
}
\newcommand\eindent{\endgroup}


%LLNCS repeat theorem.
\newcommand{\repeattheorem}[1]{%
  \begingroup
  \renewcommand{\thetheorem}{\ref{#1}}%
  \expandafter\expandafter\expandafter\theorem
  \csname reptheorem@#1\endcsname
  \endtheorem
  \endgroup
}

\newcommand{\repeatlemma}[1]{%
  \begingroup
  \renewcommand{\thelemma}{\ref{#1}}%
  \expandafter\expandafter\expandafter\lemma
  \csname replemma@#1\endcsname
  \endlemma
  \endgroup
}

\NewEnviron{replemma}[1]{%
  \global\expandafter\xdef\csname replemma@#1\endcsname{%
    \unexpanded\expandafter{\BODY}%
  }%
  \expandafter\lemma\BODY\unskip\label{#1}\endlemma
}


\NewEnviron{reptheorem}[1]{%
  \global\expandafter\xdef\csname reptheorem@#1\endcsname{%
    \unexpanded\expandafter{\BODY}%
  }%
  \expandafter\theorem\BODY\unskip\label{#1}\endtheorem
}


\makeatletter
\newcommand*{\inlineequation}[2][]{%
  \begingroup
    % Put \refstepcounter at the beginning, because
    % package `hyperref' sets the anchor here.
    \refstepcounter{equation}%
    \ifx\\#1\\%
    \else
      \label{#1}%
    \fi
    % prevent line breaks inside equation
    \relpenalty=10000 %
    \binoppenalty=10000 %
    \ensuremath{%
      % \displaystyle % larger fractions, ...
      #2%
    }%
    \qquad \@eqnnum
  \endgroup
}
\makeatother
