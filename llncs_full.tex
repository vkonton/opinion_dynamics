% This is samplepaper.tex, a sample chapter demonstrating the
% LLNCS macro package for Springer Computer Science proceedings;
% Version 2.20 of 2017/10/04
%
\documentclass[runningheads]{templates/llncs/llncs}

\usepackage[margin=1in]{geometry}
\usepackage[T1]{fontenc}
\usepackage{microtype}

\usepackage{kpfonts}
 % \usepackage{mathpazo}
% \usepackage{minionpro}
% \usepackage[boldsans]{ccfonts}
% \usepackage{beton}
% \usepackage{concrete}
%  \usepackage{sectsty}
%  \allsectionsfont{\mdseries}
%\usepackage{beton}
%\renewcommand{\bfdefault}{sbc}
%\usepackage{kurier}
% \usepackage{eulervm}

% \usepackage{ntheorem}


% allows for indexgeneration
\usepackage{makeidx}

\usepackage{amsfonts}
\usepackage{amsmath}
\usepackage{mathtools}

\usepackage{hyperref}
\hypersetup{
  colorlinks = true,
  urlcolor = {blue},
  citecolor = {blue}
}


%fonts.
%\usepackage{fontspec}

%\usepackage{newpxtext}
%\usepackage[euler-digits]{eulervm}

%%\setmainfont{Linux Libertine}

\usepackage{textcomp}
\usepackage{csquotes}

\usepackage{booktabs}
\usepackage[scale=2]{ccicons}
\usepackage{graphicx}

\usepackage{pgfplots}
\usepgfplotslibrary{dateplot}
\usepackage{color}

\usepackage{listings, color}
\definecolor{green}{rgb}{0,0.5,0}
\definecolor{gray}{rgb}{0.4,0.4,0.4}
\definecolor{lblue}{rgb}{0.9,0.9,1}
\definecolor{red}{rgb}{0.9,0,0}
\definecolor{blue}{rgb}{0.6,0,0.6}

\DeclareSymbolFont{matha}{OML}{txmi}{m}{it}% txfonts
\DeclareMathSymbol{\varv}{\mathord}{matha}{118}


% bold vectors
\usepackage{bm}

% mathbb numbers
\usepackage{dsfont}

% mathlarger
\usepackage{relsize}

\usepackage{algorithm}
\usepackage[noend]{algpseudocode}
\makeatletter
\def\BState{\State\hskip-\ALG@thistlm}
\makeatother

\lstset{
  backgroundcolor=\color{lblue},
  basicstyle=\scriptsize\ttfamily,
  breaklines=true,
  columns=fullflexible,
  commentstyle=\color{gray},
  keywordstyle=\color{blue},
  stringstyle=\color{red},
  identifierstyle=\color{black},
  language=haskell,
  %numbers=left,
  %numberstyle=\scriptsize,
  %numbersep=5pt,
  showstringspaces=false
}

\usepackage{subcaption}

 \definecolor{1}{rgb}{0.368417, 0.506779, 0.709798}
 \definecolor{2}{rgb}{0.880722, 0.611041, 0.142051}
 \definecolor{3}{rgb}{0.560181, 0.691569, 0.194885}
 \definecolor{4}{rgb}{0.922526, 0.385626, 0.209179}
 \definecolor{5}{rgb}{0.528488, 0.470624, 0.701351}
 \definecolor{6}{rgb}{0.772079, 0.431554, 0.102387}
 \definecolor{7}{rgb}{0.363898, 0.618501, 0.782349}
 \definecolor{8}{rgb}{1, 0.75, 0}
 \definecolor{9}{rgb}{0.647624, 0.37816, 0.614037}
 \definecolor{10}{rgb}{0.571589, 0.586483, 0.}
 \definecolor{11}{rgb}{0.915, 0.3325, 0.2125}
 \definecolor{12}{rgb}{0.40082222609352647, 0.5220066643438841, 0.85}
 \definecolor{13}{rgb}{0.9728288904374106, 0.621644452187053, 0.07336199581899142}
 \definecolor{14}{rgb}{0.736782672705901, 0.358, 0.5030266573755369}
 \definecolor{15}{rgb}{0.28026441037696703, 0.715, 0.4292089322474965}

% MACROS.

%display style in FRAC
\newcommand\ddfrac[2]{\frac{\displaystyle #1}{\displaystyle #2}}


\newcommand{\lp}{\left}
\newcommand{\rp}{\right}
\newcommand{\mrm}[1]{\mathrm{#1}}
\newcommand{\mbb}[1]{\mathbb{#1}}
\newcommand{\mcal}[1]{\mathcal{#1}}
\newcommand{\mfrak}[1]{\mathfrak{#1}}
\DeclareMathOperator*{\argmin}{argmin}
\DeclareMathOperator*{\argmax}{argmax}
\DeclareMathOperator*{\maj}{maj}

%Tikz crap.
\newcommand*{\equal}{=}
\newcommand*{\comma}{,}

% Asymptotic Notation.
\newcommand{\bigOh}[1]{O\lp(#1\rp)}
\newcommand{\smalloh}[1]{o\lp(#1\rp)}
\newcommand{\bigOmega}[1]{\Omega\lp(#1\rp)}
\newcommand{\smallomega}[1]{\omega\lp(#1\rp)}
\newcommand{\poly}{\mrm{poly}}

% Standard sets
\newcommand{\R}{\bm{\mrm{R}}}
\newcommand{\N}{\bm{\mrm{N}}}
\newcommand{\Q}{\bm{\mrm{Q}}}
\newcommand{\Z}{\bm{\mrm{Z}}}

% Constants.
\newcommand{\me}{\mathrm{e}}

% Analysis.
\def\d{\mrm{d}}
\newcommand\Ball[2]{\mcal{B}\lp(#1, #2 \rp)}
\newcommand\norm[2]{\| #2 \|_{#1}}

% Functions.
\newcommand{\tr}{\mathrm{tr}}
\newcommand{\1}[1]{\bm{1}{\left[#1\right]}}
\newcommand{\Real}[1]{\mathrm{Re}\lp(#1\rp)}
\newcommand{\Imag}[1]{\mathrm{Im}\lp(#1\rp)}
\DeclarePairedDelimiter\ceil{\lceil}{\rceil}
\DeclarePairedDelimiter\floor{\lfloor}{\rfloor}

% Distances & Probability.
\newcommand{\Normal}[2]{\mcal{N}\lp(#1,#2\rp)}
\newcommand{\DNormal}[2]{\mrm{DN}\lp(#1,#2\rp)}
\newcommand{\GBD}{\mrm{GBD}}
\newcommand{\PBD}{\mrm{PBD}}
\newcommand{\Bin}{\mrm{Bin}}
\newcommand{\Pois}{\mrm{Pois}}
\newcommand{\Distr}{\mathcal{L}}
\newcommand{\Unif}{\mrm{U}}

% \newcommand{\GBD}[1][]{
%   \def\ArgI{{#1}}%
%   \GBDRelay
% }
% \newcommand{\GBDRelay}[1]{
%   \mrm{GBD}\lp( \ArgI, #2 \rp)
% }

\newcommand{\NormalPDF}[3]
{
  \frac{1}{\sqrt{2 \pi} #2} \me^{-\frac{(#3 - #1)^2}{2 {#2}^2}}
}
\newcommand{\erf}[1]{\mrm{erf}\lp(#1\rp)}
\newcommand{\erfc}[1]{\mrm{erfc}\lp(#1\rp)}

\newcommand{\Prob}[1]{\bm{\mrm{P}}\lp[#1\rp]}

\newcommand{\Exp}[1][]{
  \def\ArgI{#1}%
  \ExpRelay
}
\newcommand{\ExpRelay}[1]{
  \bm{\mrm{E}}_{\ArgI} \lp[ #1 \rp]
}

\newcommand{\Expnew}[2]{
  \bm{\mrm{E}}_{#1} \lp[ #2 \rp]
}


\newcommand{\Var}[1]{\bm{\mrm{V}}\lp[#1\rp]}
\newcommand{\Cov}[2]{\mrm{Cov}\lp[#1, #2\rp]}
\newcommand{\Kurt}[1]{\mrm{Kurt}\lp[#1 \rp]}

\newcommand{\fdiv}[3]{D_{#1} \lp(#2\|#3\rp)}
\newcommand{\dkl}[2]{D_{\mrm{kl}} \lp(#1\|#2\rp) }
\newcommand{\dtv}[2]{d_{\mrm{tv}} \lp(#1, #2\rp) }
\newcommand{\dhel}[2]{d_{\mrm{hel}} \lp(#1, #2\rp) }
\newcommand{\dkol}[2]{d_{\mrm{kol}} \lp(#1, #2\rp) }

% minimax risk
\newcommand{\minimax}[1]{\mfrak{M}_{#1}}

\newcommand{\h}[1]{\hat{#1}}

% PBD powers specific macros.
\def\eps{\varepsilon}
\def\p{\hat{p}}
\def\q{\hat{q}}

\def\reals{\bm{\mrm{R}}}
\def\nats{\mathbb{N}}
\def\D{\mathcal{D}}

\def\P{\mathcal{P}}
\def\p{\hat{p}}
\def\q{\hat{q}}
\newcommand{\err}[1]{\mbox{\rm err}\left(#1\right)}


% other.

\newcommand{\sline}[1]{\textcolor{#1}{\linethickness{.5mm}\line(1,0){15}} }

% Ident in algorithmic.
\newlength\myindent
\setlength\myindent{2em}
\newcommand\bindent{%
  \begingroup
  \setlength{\itemindent}{\myindent}
  \addtolength{\algorithmicindent}{\myindent}
}
\newcommand\eindent{\endgroup}


% If you use the hyperref package, please uncomment the following line
% to display URLs in blue roman font according to Springer's eBook style:
\renewcommand\UrlFont{\color{blue}\rmfamily}

\begin{document}
%
\title{Contribution Title\thanks{Supported by MOP}}
%
%\titlerunning{Abbreviated paper title}
% If the paper title is too long for the running head, you can set
% an abbreviated paper title here
%
\author{First Author\inst{1}\and
  Second Author\inst{2,3} \and
  Third Author\inst{3}
}
%
\authorrunning{F. Author et al.}
% First names are abbreviated in the running head.
% If there are more than two authors, 'et al.' is used.
%
\institute{Princeton University, Princeton NJ 08544, USA \and
  Springer Heidelberg, Tiergartenstr. 17, 69121 Heidelberg, Germany
  \email{lncs@springer.com}\\
  \url{http://www.springer.com/gp/computer-science/lncs} \and
  ABC Institute, Rupert-Karls-University Heidelberg, Heidelberg, Germany\\
  \email{\{abc,lncs\}@uni-heidelberg.de}}
%
\maketitle              % typeset the header of the contribution
%
\begin{abstract}
  %We study opinion formation games based on the famous model proposed by Friedkin
%and Johsen.  In today's huge social networks the assumption that in each round
%agents update their opinions by taking into account the opinions of
%\emph{all} their friends could be unrealistic. Therefore, we assume that in
%each round each agent gets to meet with only one random friend of
%hers.  Since it is more likely to meet some friends than others we assume
%that agent $i$ meets agent $j$ with probability $p_{ij}$.
%The original setting of the FJ model and our limited information
%variant share the same equilibrium $x^*$.  Therefore, the interpretation
%of the equilibrium $x^*$ as an estimate of the opinions of the agents in the long
%run is still valid in our setting that resembles large social networks.
% we use a \emph{limited information} opinion
% formation game, where at round $t$, agent $i$ with intrinsic opinion
% $s_i\in[0,1]$ and expressed opinion $x_i(t) \in[0,1]$ meets with probability
% $p_{ij}$ neighbor $j$ with opinion $x_j(t)$ and suffers a disagreement cost
% that is a convex combination of $(x_i(t) - s_i)^2$ and $(x_i(t) - x_j(t))^2$.


%For a dynamics in the above setting to be considered as natural it must be
%simple, converge to the equilibrium $x^*$, and perhaps most importantly, it
%must be a rational choice for selfish agents.  In this work we show that
%an intuitive game play, is a natural dynamics for the above game.
%We prove that, after $O(1/\eps^2)$ rounds, the opinion vector
%is within error $\eps$  of the equilibrium. Moreover, to show that selfish
%agents would choose to update their opinions according to our update rule,
%we use a limited information opinion formation game, and show that it
%guarrantees no regret to the agents.


%The classical Friedkin-Johsen dynamics converges to the equilibrium within
%error $\eps$ after only $O(\log(1/\eps))$ rounds whereas, in our imperfect
%information setting, our update rule needs $\widetilde{O}(1/\eps^2)$ rounds.
%We ask whether there exists a different natural dynamics for our problem
%with better rate of convergence.  We answer this question in the negative
%by showing that update rules that guarrantee no regret to the agents cannot
%converge with less than $\mrm{poly}(1/\eps)$ rounds.
We study opinion formation games based on the famous model proposed by Friedkin
and Johsen.  In today's huge social networks the assumption that in each round
agents update their opinions by taking into account the opinions of
\emph{all} their friends could be unrealistic. Therefore, we assume that
in each round each agent gets to meet with only one random friend of
hers. Since it is more likely to meet some friends than others we assume
that agent $i$ meets agent $j$ with probability $p_{ij}$.
In this imperfect information setting, we are interested in simple and
natural variants of the FJ model that converge to the same equilibrium point $x^*$.
Specifically, we define an opinion formation game, where at round $t$,
agent $i$ with intrinsic opinion $s_i\in[0,1]$ and expressed opinion $x_i(t)
\in[0,1]$ meets with probability $p_{ij}$ neighbor $j$ with opinion $x_j(t)$
and suffers a disagreement cost that is a convex combination of
$(x_i(t) - s_i)^2$ and $(x_i(t) - x_j(t))^2$.
We show that the agents can adopt an intuitive and simple update
rule that ensures \emph{no-regret} to the experienced disagreement cost
and at the same time the produced opinion vector converges
to the equilibrium $x^*$. More precisely we prove that, after
$\mrm{poly}(1/\eps)$ rounds,
the opinion vector is within error $\eps$  of the equilibrium.

The classical Friedkin-Johsen dynamics converges to the equilibrium within
error $\eps$ after only $O(\log(1/\eps))$ rounds whereas, in our imperfect
information setting, our update rule needs $\mrm{poly}(1/\eps)$ rounds.
We ask whether there exists a different no-regret dynamics for our problem
with better rate of convergence.  We answer this question in the negative
by showing that dynamics based on no-regret algorithms cannot converge with
less than $\Omega(1/\eps)$ rounds. Finally, we present an update rule that
requires only $\widetilde{O}(\log^2(1/\eps))$ rounds to acheive dinstance
$\eps$, revealing that slow convergence is not a generic property of our
imperfect information setting.


  \keywords{First keyword  \and Second keyword \and Another keyword.}
\end{abstract}

\setcounter{page}{0}
\thispagestyle{empty}
\newpage

\section{Introduction}
% \subsection{Friedkin-Johsen Model}

\subsection{Friedkin-Johsen Model and Opinion Formation Games}
In the Friedkin-Johsen model (FJ-model) an undirected graph $G(V,E)$ with $n$
nodes, is assumed, where $V$ denotes the agents and $E$ the social relations
between them. Each agent $i$ poses an internal opinion $s_i \in [0,1]$ and a self
confidence coefficient $\alpha_i \in[0,1]$. At each round $t \geq 1$,
agent $i$ updates her opinion as follows:

\begin{equation}\label{eq:fj_dynamics}
  x_i(t) = (1-\alpha_i) \frac{\sum_{j \in N_i}x_j(t-1)}{|N_i|} + \alpha_i s_i,
\end{equation}

where $N_i$ is the set of her neighbors. The simplicity of the update rule
(\ref{eq:fj_dynamics}) makes the FJ-model plausible, because in real social
networks it is very unlikely that agents change their opinions according
to complex rules.
% \subsection{Opinion Formation Games}
Based on the FJ-model, in \cite{BKO11}, they propose an \emph{
  opinion formation game} in which the strategy that each agent
$i$ plays, is the opinion $x_i \in [0,1]$
that she publicly expresses incurring her a cost

\begin{equation}\label{eq:kleinberg_cost}
  C_i(x_i,x_{-i})=(1-a_i)\sum_{j \in N_i}(x_i -x_j)^2 + a_i(x_i-s_i)^2.
\end{equation}

In \cite{BK011} they prove that it always admits a unique
Nash Equilibrium $x^*$ and has Price of Anarchy $9/8$ if $G$ is
undirected and $O(n)$ in the directed case.  We denote an instance of this
game as $I=(G,s,a)$.
The FJ-model can be viewed as the best-response dynamics of the
of the opinion formation game.
% j
%
% \begin{equation}
%   C_i(x_i(t-1),x_{-i}(t-1))=(1-a_i)\sum_{j \in N_i}(x_i(t-1) -x_j(t-1))^2 +
%   a_i(x_i(t-1)-s_i)^2
% \end{equation}
More precisely, each agent $i$ has an opinion $x_1(0),\ldots,x_n(0)$ and
at round $t \geq 1$, suffers the cost (\ref{eq:best_response})
and updates her opinion so as to minimize her individual cost

\begin{equation}\label{eq:best_response}
  x_i(t) =
  \argmin_{x \in [0,1]}
  (1-a_i) \sum_{j \in N_i} (x - x_j(t-1))^2 + a_i (x -s_i)^2
\end{equation}

In \cite{GS14} it is proved that for any instance $I=(G,s,a)$ the opinion
vector $x(t)=(x_1(t),\ldots,x_n(t))$ converges to the unique equilibrium point
$x^*$. Moreover, this is achieved by an update rule that is very simple
but more importantly it is \emph{rational}, in the sense
that each agent $i$ adopts this rule in order to minimize her individual cost.

\subsection{Opinion Formation Games with Random Payoffs}
In the game defined by (\ref{eq:kleinberg_cost}),
agent's $i$ cost $C_i(x_i,x_{-i})$ is a deterministic function of the
opinion vector $x$. Many recent works (see e.g. \cite{Zhou17}) study games
with random payoffs, that is agent's $i$ cost ($C_i(x_i,x_{-i})$) is a
random variable. The random payoff setting can be much more realistic,
since randomness may naturally occur because of incomplete information, noise
or other stochastic factors. Motivated by this
line of research we introduce a random payoff variant of the opinion
formation game (\ref{eq:kleinberg_cost}).

\begin{definition}\label{d:random_payoff_game}
  Let $I=(G,s,a)$ an instance of the opinion formation game
  and $x$ the opinion vector. Each agent $i$,
  \begin{itemize}
    \item picks uniformly at random one of her neighbors $j \in N_i$
    \item suffers cost
      $C_i(x_i,x_{-i}) = (1-a_i)\sum_{j \in N_i}(x_i-x_j)^2 + a_i(x_i-s_i)^2$
  \end{itemize}

\end{definition}

This game is more compatible to a realistic settings since
in real world networks (e.g. Facebook, Twitter e.t.c.), each agent may have
several hundreds of friends. As a result is far more reasonable to assume that
every day, each agent meets a random small subset of her acquaintances and
suffers a cost based on how much she disagrees with them.
Since the expected cost of our random payoff variant equals
the cost of game $(\ref{eq:kleinberg_cost})$, $ \Exp{C(x_i, x_{-i})} =
(1-a_i)\sum_{j \in N_i}(x_i-x_j)^2 + a_i(x_i-s_i)^2$,
the results about the existence of a unique equilibrium,
and the PoA bounds also hold in our variant.
However, the best-response dynamics in the random payoff game
does not converge to the equilibrium.  In this work, we investigate
whether there exists natural dynamics that leads the system to the
equilibrium point $x^*$.

\subsection{Our Results and Techniques}
In this work we propose the following dynamics for the
random payoff game of Definition~\ref{d:random_payoff_game}.

\begin{itemize}
  \item Initially, each agent $i$ has an opinion $x_1(0),\ldots,x_n(0)$.

  \item At round $t \geq 1$, each agent $i$ meets uniformly at random one of
    her friends $j \in N_i$, suffers cost
    \(
      C^t_i(x_i(t-1),x_{j}(t-1))=(1-a_i)(x_i(t-1)
      -x_j(t-1))^2 + a_i(x_i(t-1)-s_i)^2
    \)
    and updates her opinion
    \begin{equation}\label{eq:fictitious_play}
      x_i(t) =
      \argmin_{x \in [0,1]}
      \sum_{\tau=1}^tC^{\tau}_i(x,x_{\pi_i(\tau)}(\tau-1))
      % =
      % (1-a_i)\frac{\sum_{\tau=1}^t x_{\pi_i(\tau)}(t-1)}{t} + a_is_i,
    \end{equation}

    where $\pi_i(\tau)$ denotes the index of the index of the neighbor that $i$
    selected at round $\tau$.

\end{itemize}
The above dynamics is the \emph{fictitious play} in the game defined
by the instance $I=(G,s,a)$. Generally speaking \emph{fictitious play} does
guarantees convergence to the equilibrium.  We show that for our game with
random payoffs fictitious play converges to equilibrium, giving the following
bound for its convergence rate
\begin{theorem}
  For every instance $I=(G(V,E),s,a)$, if $x(t)$ is the opinion vector
  generated by the update rule  (\ref{eq:fictitious_play}) then

  \[
    \Expnew{}{\norm{\infty}{x^t - x^*}} \leq
      C \sqrt{\log |V|}\frac{(\log t)^{2}}{t^{\min(1/2,\, \rho)}},
  \]
  where $\rho = \min_{i \in V} a_i$ and $C$ is a universal constant.
\end{theorem}
The update rule (\ref{eq:fictitious_play}) guarantees convergence
while vastly reducing the information exchange between the agents
at each round.
Namely, each agent learns the opinion of only one of her neighbors
whereas in the classical FJ-model (\ref{eq:fj_dynamics}) each agent
learns the opinions of all of her neighbors. In terms of
total communication needed to get within distance $\eps$ of the
equilibrium, the update rule (\ref{eq:fictitious_play}) needs
$O(|V| \log |V|)$ communication while (\ref{eq:fj_dynamics}) needs
$O(|E|)$. Of course for this difference to be significant we need
each agent to have at least $O(\log |V|)$ neighbors.  A large social
network like Facebook has approximately $2$ billion users and each user
has usually more that $100$ friends which far more than $\log(2\ 10^9)$.

Apart from converging to the equilibrium, our update rule
(\ref{eq:fictitious_play}) is also a rational behavioral
assumption since it ensures \emph{no regret} for the agents.
Having no-regret means that the average cost for each agent $i$
after $T$ rounds is close to the average cost that she would
suffer by expressing any fixed opinion. This is a very important
feature of our update rule because even players that selfishly
will to minimize their incurred cost, could choose to play according
to it.

\begin{theorem}

  For every instance $I=(G,s,a)$, for every agent $i$
  $$\sum_{\tau=1}^tC_i^\tau(x_i(\tau)) \leq \text{min}_{x \in
    [0,1]}\sum_{\tau=1}^tC_i^\tau(x) + \bigOh{\log t}$$

\end{theorem}

Even though our update rule (\ref{eq:fictitious_play}) has the above
desired properties it only achieves convergence rate of
$\widetilde{O}(1/\sqrt{t})$ for a fixed instance $I$ with
$\rho>1/2$.  For such an instance the original FJ-model outperforms
our update rule since it achieves convergence rate $O(1/2^{t})$.
We investigate whether this gap is due to an inherent characteristic
of the random payoff setting i.e. learning the opinion of just one random
neighbor vs learning all of them or we could find another natural dynamics
that converges exponentially fast. Can the agents select another no regret
algorithm that guarantees an asymptotically faster convergence rate?
In Section~4 we investigate the following question

\begin{question}
  Is there a no regret algorithm that the agents can choose such that
  for any instance $I = (G, s, a)$, $\Exp{\norm{\infty}{x_t - x^*}} = O(1/t)$ ?
\end{question}

To answer this question we first show that the existence of a no regret
algorithm (that achieves this convergence rate to the equilibrium) implies
the existence of an algorithm that uses i.i.d samples from a Bernoulli random
variable $B(p)$ to estimate its success probability $p$ with the same asymptotic
error rate. We use sample complexity lower bound techniques from statistics
to show that such a no regret algorithm is unlikely to exist.
We show that updating the opinion according to no regret algorithms leads
to update rules that are totally ignorant of the specific graph structure $G$.
Perhaps surprisingly, in Section~5 we present an update rule that takes
into account only the maximum degree of the graph and for any fixed instance $I$
achieves convergence rate $O(1/2^{\sqrt{t}})$.


% In section 4, we investigate where a better convergence rate,
% $E[||x(t)-x^*||_{\infty}]$ can be acheived if agent selected another no regret
% algorithm. More precisely, we investigate the following question, \emph{Is
%   there a no-regret algorithm such that for every instance }
% $I,~E[||x(t)-x^*||_{\infty}] \in \bigOh{\frac{1}{t}}?$  We reduce this question
% to the following question in statistical estimation, \emph{Is there a Bernoulli
%   estimator }$\hat{\theta}$\emph{, such that for all }$q \in [0,1]$, $\lim_{t
%   \rightarrow \infty} t E_q[|\hat{\theta^t} -q|]=0?$  To the best of our
% knowledge this question is not answered in statistics literature. However, we
% use standard tecnhiques for lower bounds in the statistical estimation to prove
% the following theorem.

% \begin{theorem}
%   For any Bernoulli estimator
%   $\hat{\theta}$, for all $[a,b] \subseteq [0,1]:~$ $\lim_{t \rightarrow
%     \infty}t \int_{a}^bE_p[|\hat{\theta^t} -p|]= +\infty$
% \end{theorem}
% This result indicates that the second question and consequently the
% first is very unlike to hold, especially for any reasonable no-regret
% algorithm. We believe that there exists no such algorithm and we leave this
% proof as an open problem.\\

% Finally in section 5, we present an interesting side result. The reason that
% there does not exists a no-regret algorithm, that ensures faster convergence
% rate, is the algorithm's ignorance to the instance $I$ that defines the
% dynamics. More formally, a no-regret algorithm does needs the values $s_i,a_i$
% and $Y_1,\ldots,Y_t$ to determine $x_i(t)$. We find interesting that we design
% a distributed algorithm that uses $s_i,a_i,Y_1,\ldots,Y_t$ and additionally the
% $d_\text{max}$ of $G$ that achieves for every instance $I$ convergence rate,
% $E||x(t)-x^*||_{\infty} \in \bigOh{2^{-\frac{\sqrt{t}}{d^3}}}$.

\subsection{Related Work}

% \section{Full memory with limited information exchange}
In each round, every agent learns a neighbour's opinion and updates the appropriate cell in memory. Because an agent learns only one opinion in each round, he uses outdated information about his other neighbours in order to compute his opinion. We first introduce some convinient notation.
\begin{definition}
$\pi_{ij}(t)$ is the last time that $i$ learned $j$'s opinion until time $t$.
\end{definition}

Obviously, if at time $t$ an agent $i$ learned $k$'s opinion, then $\pi_{ik}(t) = t$. The rule, according to which agent $i$ updates his opinion at time $t$ is the following(for simplicity, assume that $\alpha_i = \frac{1}{d+1}$ and that the graph is $d$-regular):$$x_{i}(t+1)=(1-\alpha_i)\frac{\sum_{j \in N_i}x_j(\pi_{ij}(t))}{d_i}+\alpha_i s_i$$
Notice that in contrast with the traditional model, instead of $x_j(t)$ we have $x_j(\pi_{ij}(t))$. That is because the information about $j$ is probably outdated.
Denote with $x^*$ the solution vector of the system. We would like to prove that such a process converges to the solution of the linear system involving the laplacian matrix of the graph. In order to analyse the algorithm, we will divide the time into "epochs". Each epoch has a length $D \ge d$, so the first epoch is from time 1 to time $D$, the second from $D+1$ to $2D$ e.t.c. Time $0$ does not belong at any epoch. The numbering of epochs begins from $1$. We will also assume that during each epoch, each agent picks every one of his neighbours for update at least once. Later we are going to find a suitable epoch length D, such that this holds with high probability. In other words, for a specific time in epoch $i$ every agent has information which has "bounded" outdateness, since every coordinate was updated at least once during the $i-1$ epoch. We are going to use this fact to prove the following lemma.

\begin{lemma}
Denote with $x^*$ the solution vector of the system. For every time $t$, which belongs to epoch $T$, it holds:
$$ ||x(t)-x^*||_{\infty} \leq \left( 1-\alpha_i\right)^T||x(0)-x^*||_{\infty}$$
\end{lemma}

\begin{proof}
We are going to use induction in time. The base case is for time $t=1$. We are going to prove that: $$ ||x(1)-x^*||_{\infty} \leq (1-\alpha_i)||x(0)-x^*||_{\infty}$$
We assume that everybody has the same initial vector $x(0)$ stored in memory, so for each $i$ holds:
\begin{equation}\label{eq:time1}
x_i(1) = (1-\alpha_i)\frac{\sum_{j \in N_i}x_j(0)}{d}+\alpha_i s_i 
\end{equation}
Since $x^*$ is the solution of the system, we have:
\begin{equation}\label{eq:sol}
x_i^* = (1-\alpha_i)\frac{\sum_{j \in N_i}x_j^*}{d}+\alpha_i s_i 
\end{equation}
Subtracting ~(\ref{eq:sol}) from ~(\ref{eq:time1}) we get:
\begin{eqnarray*}
|x_i(1) - x_i^*| &=& |(1-\alpha_i)\frac{\sum_{j \in N_i}(x_j(0)-x_j^*)}{d}|\\
&\leq& (1-\alpha_i)\frac{\sum_{j \in N_i}|x_j(0)-x_j^*|}{d} \\
&\leq& (1-\alpha_i) ||x(0)-x^*||_{\infty}
\end{eqnarray*}
$$ \Rightarrow ||x(1)-x^*||_{\infty} \leq (1-\alpha_i)||x(0)-x^*||_{\infty}$$
So the base case is verified. Let the inductive hypothesis hold for all times until $t$. Suppose that time $t+1$ belongs to epoch $T$. We fix an agent $i$. Then, if $\pi_{ij}(t)$ belongs to the previous epoch, by the induction hypothesis it holds:
$$ |x_j(\pi_{ij}(t)) - x_j^*| \leq \left( 1-\alpha_i\right)^{T-1}||x(0)-x^*||_{\infty}$$
 On the other hand, if $\pi_{ij}(t)$ belongs to the current epoch $T$, by the induction hypothesis we have:
 $$ |x_j(\pi_{ij}) - x_j^*| \leq \left( 1-\alpha_i\right)^{T}||x(0)-x^*||_{\infty} \leq \left( 1-\alpha_i\right)^{T-1}||x(0)-x^*||_{\infty}$$
 So, for every neighbour of agent $i$, the value that $i$ stores for this neighbour is "close" to the optimal. Now, using again equation $(1)$, we have:
 \begin{eqnarray*}
|x_i(t+1) - x_i^*| &=& |(1-\alpha_i)\frac{\sum_{j \in N_i}(x_j(\pi_{ij}(t))-x_j^*)}{d}|\\
&\leq& (1-\alpha_i)\frac{\sum_{j \in N_i}|x_j(\pi_{ij}(t))-x_j^*|}{d}\\
&\leq& (1-\alpha_i)\frac{\sum_{j \in N_i}||x(\pi_{ij}(t))-x^*||_\infty}{d}\\
&\leq& \frac{\sum_{j \in N_i}\left(1-\alpha_i\right)^{T-1}||x(0)-x^*||_{\infty}}{d+1}\\
&=& (1-\alpha_i)\left(1-\alpha_i\right)^{T-1}||x(0)-x^*||_{\infty}\\
&=& \left(1-\alpha_i\right)^{T}||x(0)-x^*||_{\infty}
 \end{eqnarray*}

\end{proof}

\begin{corollary}
If the algorithm runs for $O(D \log \frac{1}{\epsilon})$ iterations, it gets $\epsilon$-close to the solution $x^*$.
\end{corollary}

We now turn our attention to the problem of computing the appropriate length $D$ of epochs. A useful fact concerning the coupons collector problem is the following.

\begin{lemma}
Suppose that the collector picks $n\ln n + cn$ coupons, where $n$ is the number of distinct coupons. Then:
$$
\Prob{\text{collector hasn't seen all coupons}} \leq \frac{1}{\me^c}
$$
\end{lemma}

\begin{theorem}
After $t$ rounds, with probability at least $1 - p$: $\norm{\infty}{x^{t}-x^*} \leq (1-a)^{\frac{t}{2d\ln( \frac{dt}{p})} }$
\end{theorem}
\begin{proof}
In our setting, coupon $i$ corresponds to the selection of neighbour $i$. 
Each node is a collector and wants to gather all $d_i$ coupons during each epoch. 
Suppose $d = \max_i d_i$ is the maximum degree of the graph. Then, if we set $n = d$ and $c = \ln (\frac{dt}{p})$ , 
using the previous lemma we get that a node hasn't seen at least one neigbour after $cd + d\ln d$ samples with probability at most $\frac{p}{dt}$. 
This means that if we set $D = cd + d\ln d = d \ln \frac{dt}{p} + d \ln d \geq 2d\ln\frac{dt}{p} $ when $p$ is small enough, 
then the probability that a specific agent at a specific epoch hasn't collected all neighbouring opinions at least once is at most $\frac{p}{dt}$. 
By a simple union bound argument, we get that all agents have seen all their neighbours during all epochs with probability at least $1 - p$. 
We observe that at time $t$ approximately $\frac{t}{D}$ epochs have passed. 
Therefore, using the previous result about the convergence rate, we get that with probability at least $1 - p$:
$$ \norm{\infty}{x^{t}-x^*} \leq (1-a)^{\frac{t}{D}} \leq (1-a)^{\frac{t}{2d\ln( \frac{dt}{p})} }$$

\end{proof}
We are now going to translate this result to one that involves the expected value of the error.
\begin{theorem}If we set $u(t) = \norm{\infty}{x^t-x^*}$, the error after $t$ rounds, then:
$$
\Exp{u(t)} \leq 2(1-a)^{\frac{\sqrt{t}}{4d\ln(dt)\ln\lp( \frac{1}{1-a}\rp)}}
$$
\end{theorem}
\begin{proof}
Using the result of the previous theorem, we obtain:
$$ \Prob{u(t) > (1-a)^{\frac{t}{2d\ln( \frac{dt}{p})} }} \leq p $$
for every probability $p \in [0,1]$. Also, since all the parameters of the problem lie in $[0,1]$, we have
$$\Exp{u(t)|u(t) > r} \leq 1$$
Now, by the conditional expectations identity, we get:
\begin{align*}
\Exp{u(t)} &= \Exp{u(t)|u(t) > r}\Prob{u(t) > r} +\Exp{u(t)|u(t) \leq r}\Prob{u(t) \leq r}\\
&\leq p + r
\end{align*}
where $r = (1-a)^{\frac{t}{2d\ln( \frac{dt}{p})} }$. If we set $p = (1-a)^{\frac{\sqrt{t}}{2d\ln dt}}$, then:
$$
\Exp{u(t)} \leq (1-a)^{\frac{\sqrt{t}}{2d\ln dt}} + (1-a)^{\frac{t}{2d\ln( \frac{dt}{p})} }
$$
We now evaluate $r$ for our choice of probability $p$:
\begin{align*}
r
&= (1-a)^{\frac{t}{2d\ln\lp( \frac{dt}{p}\rp)} }\\
&= (1-a)^{\frac{t}{2d\ln\lp( \frac{dt}{(1-a)^{\frac{\sqrt{t}}{2d\ln dt}}}\rp)} }\\
&= (1-a)^{\frac{t}{2d\ln dt + 2d\frac{\sqrt{t}}{2d\ln dt}\ln\lp( \frac{1}{(1-a)}\rp) }}\\
&\leq (1-a)^{\frac{t}{4d\ln(dt)\ln\lp( \frac{1}{1-a}\rp) \sqrt{t}}}\\
&= (1-a)^{\frac{\sqrt{t}}{4d\ln(dt)\ln\lp( \frac{1}{1-a}\rp)}}
\end{align*}

Using the previous calculation, we obtain:
$$ \Exp{u(t)} \leq (1-a)^{\frac{\sqrt{t}}{2d\ln dt}} + (1-a)^{\frac{\sqrt{t}}{4d\ln(dt)\ln\lp( \frac{1}{1-a}\rp)}} \leq 2(1-a)^{\frac{\sqrt{t}}{4d\ln(dt)\ln\lp( \frac{1}{1-a}\rp)}}$$
\end{proof}
Therefore, this strategy achieves subexponential convergence rate in expectation.

\section{Constant Memory with full window}

In this work we investigate an variant of the above process.
We denote by $V$ the set of agents, and $N_{i} \subset V$ the set of neighbors
of agent $i$.
Let $x(t) \in [0,1]^n$ be the opinion vector at round $t$.
We denote by $x_i(t)$ the opinion of agent $i$ at round $t$.
At round $t+1$, $x(t+1)$ is constructed in the following way.
At first, each agent $i$ gets to see
the opinion $x_j(t)$, where $j \in N_i$ is picked uniformly at random
from the set of her neighbors.  Let $W_i^t$ be the random variable
corresponding to the selected neighbor. Now each agent $i \in V$
updates her opinion as follows
\[
  x_i(t+1) =
  (1-\alpha_i)\frac{\sum_{\tau=1}^{t+1} x_{W_i^\tau}(\tau-1)}{t+1}
  + \alpha_i s_i.
\]
We remark that each agent $i$ get to see \emph{only} the opinion
$x_{W_i^t}$ and not the label $W_i^t$ of her neighbor.

Let $x^* \in [0,1]^n$ be the unique equilibrium point
of the given instance $I$. We prove that
the above stochastic process has the following convergence
rate to $x^*$.
\begin{theorem}

\[
\Exp{\norm{\infty}{x^t - x^*}} =
    \begin{cases}
      \bigOh{\sqrt{\log n}\frac{(\log t)^{2}}{t^\alpha}}
      &\quad\text{if } \alpha \leq 1/2\\
      \bigOh{\sqrt{\log n}\frac{(\log t)^{2}}{t^{1/2}}}
      &\quad\text{if } \alpha > 1/2\\
    \end{cases}
  \]
\end{theorem}
\begin{proof}
  Setting $p = 1/\sqrt{t}$ in Lemma~\ref{l:error_bound} yields
  the result.
\end{proof}

We start by stating the standard Hoeffding bound

\begin{lemma}[Hoeffding's Inequality]\label{l:hoeffding}
  Let $X = (X_1 +\cdots + X_t) /t$, $X_i \in [0,1]$.
  Then,
  \[
    \Prob{|X - \Exp{X} | > \lambda} < 2 e^{-2 t \lambda^2}.
  \]
\end{lemma}

\begin{lemma}\label{l:error_recursion}
  With probability at least $1-p$, $\|x^t - x^*\|_{\infty} \leq e(t)$,
  where $e(t)$ satisfies the following recursive relation
  $$e(t) =
  \delta(t) + (1-\alpha)\frac{\sum_{\tau=0}^{t-1}e(\tau)}{t}
  \text{ and } e(0)=\|x^0 - x^*\|_{\infty}, $$
  where $\delta(t) = \sqrt{ \frac{\log(\pi^2 n t^2/(6 p))}{t}}$.
\end{lemma}
\begin{corollary}\label{c:error_recursion}
  The function $e(t)$ satisfies the following recursive relation
  $$e(t+1)-e(t) + \alpha \frac{e(t)}{t+1}=\delta(t+1)-\delta(t) + \frac{\delta(t)}{t+1}$$
\end{corollary}
\begin{proof}
  As we have already mentioned for any instance $I$ there exists
  a unique equilibrium vector $x^*$. Since
  $W_i^\tau \sim \Unif(N_i)$ we have that $
  \Exp{x^*_{W_i^\tau}} = \frac{\sum_{j \in N_i} x^*_j}{|N_i|}
  $.
  Since $W_i^\tau$ are independent random variables, we can use
  Hoeffding's inequality (Lemma~\ref{l:hoeffding}) to get
  \[
    \Prob
    {
      \lp|
      \frac{\sum_{\tau = 1}^{t} x^*_{W_i^\tau}}{t}
      - \frac{\sum_{j \in N_i} x^*_j}{|N_i|} \rp|
      > \delta(t)
    }
    < \frac{6}{\pi^2} \frac{p}{ n t^2},
  \]
  where $\delta(t) = \sqrt{ \frac{\log(\pi^2 n t^2/(6 p))}{t}}$.
  Therefore, by the union bound
  \begin{align*}
    &\Prob
    {
      \text{for all } t \geq 1 :
      \max_{i \in V}
      \lp|
      \frac{\sum_{\tau = 1}^{t} x^*_{W_i^{\tau}}}{t}
      - \frac{\sum_{j \in N_i} x^*_j}{|N_i|} \rp|
      > \delta(t)} \leq \\
    &\sum_{t =1 }^{\infty} \Prob
    {
      \max_{i \in V}
      \lp|
      \frac{\sum_{\tau = 1}^{t} x^*_{W_i^{\tau}}}{t}
      - \frac{\sum_{j \in N_i} x^*_j}{|N_i|} \rp|
      > \delta(t)} \leq
    \\
    &\sum_{i=1}^{\infty} \frac{6}{\pi^2} \frac{1}{t^2} \sum_{i=1}^n \frac{p}{n} =
    p
  \end{align*}

  As a result with probability $1-p$ we have that for all $t$
  \begin{equation}\label{eq:error_per_round}
    \lp|
    \frac{\sum_{\tau=1}^t x^*_{W_i^\tau}}{t} -
    \frac{\sum_{j \in N_i} x^*_j}{|N_i|}
    \rp| \leq \delta(t)
  \end{equation}
  We will our claim by induction.
  We assume that $\norm{\infty}{x^{\tau}-x^*} \leq e(\tau)$ for all
  $\tau \leq t-1$. Then
  \begin{align}
    x_i(t)
    &=
    (1-\alpha_i)\frac{\sum_{\tau=1}^{t}x_{W_i^{\tau}}(\tau-1)}{t}
    + \alpha_i s_i \nonumber \\
    &\leq
    (1-\alpha_i)\frac{\sum_{\tau=1}^{t}x^*_{W_i^{\tau}} +
      \sum_{\tau=1}^{t} e(\tau-1)}{t} + \alpha_i s_i \label{step:induction_step}\\
    &\leq
    (1-\alpha_i)
    \lp(
    \frac{\sum_{\tau=1}^{t}x^*_{W_i^{\tau}}}{t}+
    \frac{\sum_{\tau=0}^{t-1} e(\tau)}{t}
    \rp)
    + \alpha_i s_i \nonumber\\
    &\leq
    (1-\alpha_i)\lp(\frac{\sum_{j \in N_i}x^*_{j}}{|N_i|} +
    \delta(t) + \frac{\sum_{\tau=0}^{t-1} e(\tau)}{t} \rp) +
    \alpha_i s_i \label{step:error} \\
    &\leq
    x_i^* + \delta(t) + (1-\alpha)
    \lp(
    \frac{\sum_{\tau=0}^{t-1} e(\tau)}{t}
    \rp)
    \nonumber
  \end{align}
  We get (\ref{step:induction_step}) from the induction step and
  (\ref{step:error}) from inequality~(\ref{eq:error_per_round}).
  Similarly, we can prove that
  $x_i(t) \geq x_i^* - \delta(t) - (1-\alpha)
  \frac{\sum_{\tau=1}^t e(\tau)}{t}$.
  As a result $||x_i(t)-x^*||_{\infty} \leq e(t)$.
\end{proof}

In order to bound the convergence time of the system, we just need to bound
the convergence rate of the function $e(t)$.
The following lemma provides us with a simple upper bound for the
convergence rate of our process.

% \begin{lemma}\label{l:error_bound}
%   $$e(t) \leq \frac{e(0)}{t^\alpha} +
%   \frac{\bigOh{\sqrt{\log(\frac{nd}{p})}}}{t^\alpha}
%   \sum_{\tau=1}^t\tau^\alpha\frac{\sqrt{\log \tau}}{\tau^{3/2}}$$
% \end{lemma}

\begin{lemma}\label{l:error_bound}
  Let $e(t)$ be a function satisfying the recursion of
  Corollary~\ref{c:error_recursion}. Then with probability
  at least $1-p$ we have that
  \[
    e(t) =
    \begin{cases}
      \bigOh{\sqrt{\log(\frac{n}{p})}\frac{(\log t)^{3/2}}{t^\alpha}}
      &\quad\text{if } \alpha \leq 1/2\\
      \bigOh{\sqrt{\log(\frac{n}{p})}\frac{(\log t)^{3/2}}{t^{1/2}}}
      &\quad\text{if } \alpha > 1/2\\
    \end{cases}
  \]
\end{lemma}

\begin{proof}
  At first, since $\frac{\delta(t)}{t}$ is a decreasing function
  for $p \leq 1/4$, we have that
  $e(t)\leq (1-\frac{\alpha}{t}) + g(t)$, where $g(t)=\frac{\delta(t)}{t}$.
  \begin{align*}
    e(t)
    &\leq
    (1-\frac{\alpha}{t})e(t-1) + g(t)\\
    &\leq
    (1-\frac{\alpha}{t})(1-\frac{\alpha}{t-1})e(t-2)
    + (1-\frac{\alpha}{t})g(t-1) + g(t)\\
    &\leq
    (1-\frac{\alpha}{t})\cdots (1-\alpha)e(0)
    + \sum_{\tau=1}^t g(\tau)\prod_{i=\tau+1}^t(1-\frac{\alpha}{i})\\
    &\leq
    \frac{e(0)}{t^\alpha}
    + \sum_{\tau=1}^tg(\tau)e^{-\alpha\sum_{i=\tau+1}^t\frac{1}{i}}\\
    &\leq
    \frac{e(0)}{t^\alpha}
    + \sum_{\tau=1}^tg(\tau)e^{-\alpha(H_t-H_{\tau})}\\
    &\leq
    \frac{e(0)}{t^\alpha}
    + e^{-\alpha H_t}\sum_{\tau=1}^tg(\tau)e^{\alpha H_{\tau}}\\
    &\leq \frac{e(0)}{t^\alpha}
    + \frac{\bigOh{\sqrt{\log(\frac{n}{p})}}}{t^\alpha}
    \sum_{\tau=1}^t\tau^\alpha\frac{\sqrt{\log \tau}}{\tau^{3/2}}\\
  \end{align*}

  We observe that
  \[
    \sum_{\tau=1}^t\tau^\alpha
    \frac{\sqrt{\log \tau}}{\tau^{3/2}}
    \leq
    \int_{\tau=1}^t
    \tau^\alpha\frac{\sqrt{\log \tau}}{\tau^{3/2}}d\tau,
  \]
  since $\tau^\alpha\frac{\sqrt{\log \tau}}{\tau^{3/2}}$
  \begin{itemize}
    \item If
  $\alpha\leq 1/2$ then
  \[
    \int_{\tau=1}^t \tau^\alpha\frac{\sqrt{\log \tau}}{\tau^{3/2}}d\tau
    \leq
    \int_{\tau=1}^t
    \tau^{1/2}\frac{\sqrt{\log \tau}}{\tau^{3/2}}d\tau = O((\log t)^{3/2}))
  \]
\item If $\alpha> 1/2$ then\\
  \begin{align*}
    \int_{\tau=1}^t
    \tau^\alpha\frac{\sqrt{\log \tau}}{\tau^{3/2}}d\tau
    &=
    \int_{\tau=1}^t \tau^{\alpha-1/2}\frac{\sqrt{\log \tau}}{\tau}d\tau \\
    &=
    \frac{2}{3} \int_{\tau=1}^t
    \tau^{\alpha-1/2}((\log \tau)^{3/2})'d\tau \\
    &=
    \frac{2}{3}(\log t)^{3/2} - (\alpha-1/2)\frac{2}{3}
    \int_{\tau=1}^t \tau^{\alpha-3/2}(\log \tau)^{3/2}d\tau \\
    &=
    \bigOh{(\log t)^{3/2}}
  \end{align*}

\end{itemize}

\end{proof}

% \section{Lower bound for unbiased estimators}

We first state a well known result from point estimation:
\begin{lemma}[Cramer-Rao bound]Let $\hat{\theta}$ be an estimator for the parameter $\theta$ of a distribution $P_\theta$, where $\theta$ is a continuous parameter. Suppose that the estimator is unbiased, that is:$E[\hat{\theta}] = 0$, for all distributions $P_\theta$. Under suitable regularity conditions, which are met by bernoulli distributions, it holds:$$Var[\hat{\theta}] \geq  \frac{1}{nI_\theta}$$, where $I_\theta$ is the Fischer information of distribution $P_\theta$.
\end{lemma}

In the case of Bernoulli random variables, we can easily show that for a Bernoulli with probability $\theta$, $I_\theta = \frac{1}{\theta(1-\theta)} $. Applying the above result we obtain $$ var[\hat{\theta}] \geq \frac{\theta(1-\theta) }{n}\geq \frac{1}{4n} $$ for every unbiased estimator $\hat{\theta}$. We can use this fact to show that a specific class of distributed protocols that solve our problem has "large" variance. Let $P$ be a protocol that when restricted to the star topology acts like an unbiased estimator for the weighted mean value of the neighbours, which can be an arbitrary real number in $[0,1]$. We will show that for all star topologies the variance of the solution is $\geq \frac{(1-a)^2}{4n}$ after $n$ samples. Suppose that for a specific star topology the preceding claim doesn't hold. Then, we notice that if $\hat{\theta}$ is the output of the protocol is this topology, then $\frac{(\hat{\theta}-as)}{1-a}$ is an unbiased estimator of the mean value of the neighbours, and this holds for every possible value of the mean. By our hypothesis, we have:
$$Var[\frac{\hat{\theta}-as}{1-a}] = \frac{1}{(1-a)^2}Var[\hat{\theta}] < \frac{1}{4n}$$, a contradiction, since our constructed estimator is unbiased. 

\section{No Regret}
We consider the following online convex optimization problem.
At each time step $t$, the player $i$ selects a real number $x^t$ and a
function $f^t(x)$ arrives. The player then suffers $f^t(x^t)$ cost.
The functions $f^t(x)$ have the following form:

$$f^t(x) = \alpha(x-s_i)^2 + (1-\alpha)(x-a_t)^2$$
where $s_i,\alpha \in [0,1]$ and are independent of $t$ and $a_t \in [0,1]$.
In other words, the function $f^t$ is uniquely determined by the number $a_t$.

We show that for this class of functions \emph{fictitious play} admits no regret.
\begin{theorem}\label{t:no_regret}
  Let $x^t = \arg min_{x \in [0,1]} \sum_{\tau = 1}^{t-1} f^\tau (x)$ then
  $$
  \sum_{t = 1}^{T} f^t(x^t) \leq min_{x \in [0,1]} \sum_{t=1}^T f^t(x) + O(log T)
  $$
\end{theorem}
\begin{proof}

Theorem~\ref{t:no_regret} easily follows since:
\begin{align*}
  \sum_{t=1}^T f^t(x^t)
  &\leq
  \sum_{t=1}^T f^t(y^t) + \sum_{t=1}^T 2\frac{1-\alpha}{t} +
  \sum_{t=1}^T \frac{(1-\alpha)^2}{t^2}\\
  &\leq
  min_{x \in [0,1]} \sum_{t=1}^T f^t(x) +
  2(1-\alpha)(logT + 1) + (1-\alpha)\frac{\pi^2}{6}\\
  &\leq
  min_{x \in [0,1]} \sum_{t=1}^T f^t(x) + O(logT)
\end{align*}
\end{proof}

At first we prove that the sequence
$y^t = arg min_{x \in [0,1]} \sum_{\tau = 1}^{t}f^\tau (x)$ admits no regret.
\begin{lemma} Let $y^t = argmin_{x \in [0,1]} \sum_{\tau = 1}^t f^\tau (x)$ then
  $$ \sum_{t=1}^T f^t(y^t) \leq min_{x \in [0,1]} \sum_{t=1}^T f^t(x)$$
\end{lemma}
\begin{proof}By definition of $y^t$,
  $min_{x \in [0,1]} \sum_{t=1}^T f^t(x) = \sum_{t=1}^T f^t(y^T)$, so
  \begin{align*}
    \sum_{t=1}^T f^t(y^t) - min_{x \in [0,1]} \sum_{t=1}^T f^t(x) &=
    \sum_{t=1}^T f^t(y^t) - \sum_{t=1}^T f^t(y^T)\\
    &= \sum_{t=1}^{T-1} f^t(y^t) - \sum_{t=1}^{T-1} f^t(y^T)\\
    &\leq \sum_{t=1}^{T-1} f^t(y^t) - \sum_{t=1}^{T-1} f^t(y^{T-1})\\
    &= \sum_{t=1}^{T-2} f^t(y^t) - \sum_{t=1}^{T-2} f^t(y^{T-1})
  \end{align*}
  Continuing in the same way, we get
  $\sum_{t=1}^T f^t(y^t) \leq min_{x \in [0,1]} \sum_{t=1}^T f^t(x)$.
\end{proof}

Now we can derive some intuition for the reason that \emph{fictitious play} admits
no regret. Since the cost incurred by the sequence $y^t$ is at most that of the best
fixed strategy, we can compare the cost incurred by $x^t$ with that of $y^t$.
However, for each $t$ the numbers $x^t$ and $y^t$ are quite close and as a result
the difference in their cost must be quite small.
\begin{lemma}
  For all $t$, $\lp|x^t - y^t \rp| \leq \frac{1-\alpha}{t}$
\end{lemma}
\begin{proof}
  By definition $x^t = \alpha s_i + (1-\alpha)\frac{\sum_{\tau = 1}^{t-1} a_\tau}{t-1}$
  and $ y^t = \alpha s_i + (1-\alpha)\frac{\sum_{\tau = 1}^t a_\tau}{t}$.
  \begin{align*}
    \lp|x^t - y^t\rp|
    &=
    (1-\alpha)\lp|\frac{\sum_{\tau = 1}^{t-1}a_\tau}{t-1}
    - \frac{\sum_{\tau = 1}^t a_\tau}{t}\rp|\\
    &=
    (1-\alpha)\lp|\frac{\sum_{\tau = 1}^{t-1}a_\tau -(t-1)a_t}{t(t-1)}\rp|\\
    &\leq
    \frac{1-\alpha}{t}
  \end{align*}
  The last inequality follows from the fact that $a_\tau \in [0,1]$.
\end{proof}

\begin{lemma}
  For all $t$, $f^t(x^t) \leq f^t(y^t) + 2\frac{1-\alpha}{t} + \frac{(1-a)^2}{t^2}$.
\end{lemma}
\begin{proof}
  \begin{align*}
    f^t(x^t)
    &=
    \alpha(x^t - s_i)^2 + (1 - \alpha)(x^t - a_t)^2 \\
    &\leq
    \alpha(y^t - s_i)^2 + 2\alpha\lp|y^t -
    s_i\rp|\lp|x^t - y^t\rp| + \alpha \lp|x^t - y^t\rp|^2 \\
    &\quad + (1-\alpha)(y_t - a_t)^2 +
    2(1-\alpha)\lp|y^t - a_t\rp|\lp|x^t-y^t\rp| + (1 - \alpha)\lp|x^t - y^t\rp|^2\\
    &\leq
    f^t(y^t) + 2\lp|x^t - y^t\rp| + \lp|y^t - x^t\rp|^2\\
    &\leq
    f^t(y^t) + 2\frac{1-\alpha}{t} + \frac{(1-\alpha)^2}{t^2}
  \end{align*}
\end{proof}

\section{Lower Bound}
\begin{theorem}\label{t:lower_bound}
Suppose $\hat{\theta}^t$ is an estimator for the probability $p$ of a Bernoulli random variable that takes $t$ samples. Then, for every interval $[a,b]$ which is contained in $[0,1]$ and for every $c>0$, it holds that:
$$ \lim_{t\to\infty} t^{1+c}\int_{a}^b \Exp{\norm{\infty}{\hat{\theta}^t-p}}dp = \infty$$
\end{theorem}
In order to prove this, we need the following result:
\begin{lemma}[Fano's Inequality]\label{l:fano}
Suppose we have an estimator $\hat{\theta}^t$ for the probability of a Bernoulli distribution. 
The unknown distribution belongs to a family of distributions $\mathcal{P}$ with the property that the probabilities of every two distributions in this family differ by at least $2\delta$, where $\delta$ is a positive real number.
Then, for every subset ${P_1, P_2, ..., P_n}$ of distributions from this family, where $p_i$ is the probability of distribution $P_1$, the following inequality holds:
$$\frac{1}{n} \sum_{i=1}^n \Exp{\lp|\hat{\theta}^t-p_i\rp|} \geq \delta \lp(1-\frac{I(X;V)+\log 2}{\log n}\rp)$$
\end{lemma}

We are now going to use Theroem~(\ref{l:fano}) to prove the following lemma, which provides a lower bound for the mean of the expectation of errors of the estimator.
\begin{lemma}\label{l:fano_application}
Suppose we choose $t$ Bernoulli distributions $\{P_i\}_{i=0}^{t-1}$ with the following probabilities: $$p_i = a+\frac{b-a}{2t} + i \frac{b-a}{t} , i=0, ..., t-1$$. Then:
$$\frac{1}{t} \sum_{i=1}^t \Exp{\lp|\hat{\theta}^t-p_i\rp|} \geq \frac{b-a}{2t} \lp(c_1 - \frac{c_2}{t}\rp)$$
where $c_1, c_2 >0$ and $c_1 > \frac{1}{2}$.
\end{lemma}
\begin{proof}
Without loss of generality, we can assume that $t$ is a multiple of 5. For simplicity, we set $E_i = \Exp{\lp|\hat{\theta}^t-p_i\rp|}$.
By the choice of $p_i$, we have that $$\lp|p_i - p_j\rp| \geq \frac{b-a}{t}$$, for every distinct $i,j$ in the family.
 This means that for the family of $t$ distributions that we picked, the property of Lemma~(\ref{l:fano}) is satisfied for $\delta = \frac{b-a}{2t}$. Thus, by dividing the $t$ distributions into groups of 5 and applying Lemma~(\ref{l:fano}) to each one, we obtain:
 \begin{align}\label{eq:fano}
 \frac{\sum{k =i}^{i+4}}{5} &\geq \frac{b-a}{2t}\lp(1-\frac{I(X;V)+\log 2}{\log n}\rp) \nonumber\\
 &= \frac{b-a}{2t}\lp(1- \frac{\log 2}{\log 5} - \frac{I(X;V)}{\log5}\rp) \nonumber\\
 &= \frac{b-a}{2t}\lp(c_1 - \frac{I(X;V)}{\log5}\rp)
 \end{align}
 where $c_1 = 1-\frac{\log2}{\log5} > \frac{1}{2}$. 
 We are now going to upper bound the mutual information $I(X;V)$ for every group of 5 distributions. We denote by $P_i^t$ the distribution on vectors with $t$ coordinates,
  where each coordinate is sampled independently from $P_i$. Let $U= \{i,i+1,i+2,i+3,i+4\}$ be a family of 5 distributions.
 Using a well known inequality REFERENCE NEEDED HERE:
\begin{align*}
I(X;V) &\leq \frac{1}{5^2}\sum_{\substack{i,j \in U \\i\neq j}} D_{kl}\lp(P_i^t,P_j^t\rp)\\
&\leq D_{kl}\lp(P_i^t, P_{i+4}^t\rp)\\
&= t D_{kl}\lp(P_i, P_{i+4}\rp)\\
&\leq t \frac{\lp(p_i - p_{i+4}\rp)^2}{p_i(1-p_i)}\\
&\leq t\lp(\frac{1}{a(1-a)} + \frac{1}{b(1-b)}\rp) \lp(p_i - p_{i+4}\rp)^2\\
&= t\lp(\frac{1}{a(1-a)} + \frac{1}{b(1-b)}\rp)\lp(\frac{b-a}{2t} + i\frac{b-a}{t} - \frac{b-a}{t} -(i+4)\frac{b-a}{t}\rp)^2\\
&= t\lp(\frac{1}{a(1-a)} + \frac{1}{b(1-b)}\rp) \frac{16(b-a)^2}{t^2}\\
&= \frac{16(b-a)^2}{t} \lp(\frac{1}{a(1-a)} + \frac{1}{b(1-b)}\rp) 
\end{align*}
So, if we set $$c_2 = 16\frac{(b-a)^2}{\log5}\lp(\frac{1}{a(1-a)} + \frac{1}{b(1-b)}\rp)$$ 
then by equation~(\ref{eq:fano}) we obtain:
\begin{equation}\label{eq:fano5}
\sum_{k =i}^{i+4} \geq \frac{5(b-a)}{2t}\lp(c_1 - \frac{c_2}{t}\rp)
\end{equation}
Inequality~(\ref{eq:fano5}) holds for every group of 5 distributions. By summing up for all groups:
\begin{align*}
\sum_{i=0}^{t-1} E_i &\geq t \frac{b-a}{2t}\lp(c_1 - \frac{c_2}{t}\rp)\\
&= \frac{b-a}{2}\lp(c_1 - \frac{c_2}{t}\rp)
\end{align*}
which is what we wanted to prove. 


\end{proof}
Next, we state a lemma regarding an upper bound on the expectation of the error in each $p_i$.
\begin{lemma}\label{l:cauchy_schwarz}
For every $p\in [-(b-a)/{2t} + p_i , p_i + (b-a)/{2t}]$ :
$$\Exp{\lp|\hat{\theta}^t-p_i\rp|}_{p_i} \leq c_a^b\Exp{\lp|\hat{\theta}^t-p\rp|}_p + \lp|p - p_i\rp|
$$
where $c_a^b$ is a constant that depends on $a,b$.
 
\end{lemma}
Now, we find an upper bound for the quantity $\sum_{i=0}^{t-1} E_i$.
Our goal is to find an upper bound where the coefficient of $(b-a)/t$ is less than $1/4$.

\begin{lemma}\label{l:upper}
$$
\frac{1}{t} \sum_{i=0}^{t-1} E_i \leq \frac{c_a^b}{b-a} \int_a^b \Exp{\lp|\hat{\theta}^t-p\rp|} dp + \frac{b-a}{4t}
$$
\end{lemma}
\begin{proof}
By Lemma~(\ref{l:cauchy_schwarz}) we get that for every $p\in [-(b-a)/{2t} + p_i , p_i + (b-a)/{2t}]$:
$$E_i \leq  c_a^b\Exp{\lp|\hat{\theta}^t-p\rp|}_p + \lp|p_i - p\rp|$$
We integrate both sides of the equation with respect to p and get:
$$
\int_{p_i - \frac{b-a}{2t}}^{p_i + \frac{b-a}{2t}} E_i dp \leq c_a^b \int_{p_i - \frac{b-a}{2t}}^{p_i + \frac{b-a}{2t}}\Exp{\lp|\hat{\theta}^t-p\rp|}_p dp + \int_{p_i - \frac{b-a}{2t}}^{p_i + \frac{b-a}{2t}}\lp|p_i - p\rp| dp
$$
Hence:
\begin{align*}
\frac{b-a}{t} E_i &\leq c_a^b\int_{p_i - \frac{b-a}{2t}}^{p_i+ \frac{b-a}{2t}}\Exp{\lp|\hat{\theta}^t-p\rp|}_p dp +  2\int_{p_i - \frac{b-a}{2t}}^{p_i}\lp|p_i-p\rp|dp\\
&= c_a^b\int_{p_i - \frac{b-a}{2t}}^{p_i + \frac{b-a}{2t}}\Exp{\lp|\hat{\theta}^t-p\rp|}_p dp + \frac{1}{4} \frac{\lp(b-a\rp)^2}{t^2}
\end{align*}

Therefore:
\begin{equation}\label{eq:one_side}
\frac{E_i}{t} \leq \frac{c_a^b}{b-a} \int_{p_i - \frac{b-a}{2t}}^{p_i + \frac{b-a}{2t}}\Exp{\lp|\hat{\theta}^t-p\rp|}_p dp + \frac{b-a}{4t^2}
\end{equation}
By summing up Equation~(\ref{eq:one_side}) for all $t$ intervals, we get:
\begin{align*}
\frac{\sum_{i=0}^{t-1} E_i}{t} &\leq \frac{c_a^b}{b-a} \sum_{i=0}^{t-1}
\int_{p_i - \frac{b-a}{2t}}^{p_i+ \frac{b-a}{2t}}\Exp{\lp|\hat{\theta}^t-p\rp|}_p dp + \frac{b-a}{4t}\\
&=  \frac{c_a^b}{b-a} \int_a^b \Exp{\lp|\hat{\theta}^t-p\rp|}_p dp + \frac{b-a}{4t}
\end{align*}



\end{proof}
Now, by combining Lemmas~(\ref{l:fano_application}) and ~(\ref{l:upper}) we get:
$$
\frac{c_a^b}{b-a} \int_a^b \Exp{\lp|\hat{\theta}^t-p\rp|}_p dp \geq \frac{b-a}{2t}\lp(\lp(c_1 - \frac{1}{2}\rp)- \frac{c_2}{t}\rp)
$$
By multiplying by $t^{1+c}$ we get:
\begin{equation}\label{eq:final}
\frac{c_a^b}{b-a}t^{1+c} \int_a^b \Exp{\lp|\hat{\theta}^t-p\rp|}_p dp \geq \frac{b-a}{2}t^c\lp(\lp(c_1 - \frac{1}{2}\rp)- \frac{c_2}{t}\rp)
\end{equation}
The coefficient of $t^c$ in the right hand side of~(\ref{eq:final}) is positive, so by sendig $t \to \infty$ we get:
$$ \lim_{t\to\infty} t^{1+c}\int_{a}^b \Exp{\norm{\infty}{\hat{\theta}^t-p}}dp = \infty$$ 



\section{Lower bound}
\begin{theorem}\label{t:lower_bound_lecam}
Let $P$ be a protocol run by each agent that produces vector output $x^t$ after $t$ iterations. 
Suppose that for every graph $G$ and for every choice of parameters $s_i,\alpha_i \in [0,1]$ the error of the output is:
$$
\Exp{\norm{\infty}{x^t - x^*}} = f(n,t)
$$
where $n$ is the number of nodes in the graph. Then, 
$$
\lim_{t \to \infty} \sqrt{t} f(t,t) > 0
$$
\end{theorem}
\begin{proof}

Let $G$ be a graph that consists of 2 star topologies. That is, we have 2 center nodes $v_1, v_2$. Each of the two center nodes has exactly $t$ neighbours, which have degree 1. We will call these nodes the leaves of the graph. This graph has $2t+2$ nodes. We set $\alpha_i = 1$ for all the leaves and $\alpha_i = 1/2$ for the two centers. For exactly $t/2 + \sqrt{t}/2$ leaves of $v_1$ we set $s_i = 1$, for the rest $s_i = 0$. Similarly, for exactly $t/2 - \sqrt{t}/2$ leaves of $v_2$ we set $s_i = 1$, for the rest $s_i = 0$. By direct calculation, we get that the equilibrium point for this graph is the following:
$$
x_i^* = 
\begin{cases}
s_i &\quad\text{if i is a leaf node}\\
\frac{1}{4}+\frac{1}{4\sqrt{t}} &\quad\text{if } i = v_1\\
\frac{1}{4}-\frac{1}{4\sqrt{t}} &\quad\text{if } i = v_2
\end{cases}
$$
Since the graph has $2t+2$ nodes:
\begin{equation}\label{eq:f(2t,t)}
\Exp{\norm{\infty}{x^t - x^*}} = f(2t+2,t)
\end{equation}
On the other hand, since $\norm{\infty}{x^t - x^*} \geq \max\{\lp|x_{v_1} - x_{v_1}^*\rp|, \lp|x_{v_2} - x_{v_2}^*\rp|\}$ we get:
\begin{equation}\label{eq:mean}
\Exp{\norm{\infty}{x^t - x^*}} \geq \frac{1}{2} \Exp{\lp|x_{v_1}^t - x_{v_1}^*\rp|} + \frac{1}{2} \Exp{\lp|x_{v_2}^t - x_{v_2}^*\rp|}
\end{equation} 
Suppose we have a star graph with center node $c$. At each time $t$, the oracle returns to the center node either the value of a leaf with $s_i = 1$ (e.g.$f(1,1,t)$) with probability :
$$
p = \frac{\text{\# leaves with } s_i = 1}{\text{\# leaves}}
$$
or the value of a leaf with $s_i = 0$ (e.g.$f(1,0,t)$) with probability $1-p$. As a result, knowing the 0-1 choice of the oracle at time step $t$, one can compute $x_c^t$, 
meaning that $\hat{\theta}^t = 2x_c^t$ is an estimator for a Bernoulli random variable with probability $p$. This obviously holds for every choice of $s_i \in \{0,1\}$. 
This means that in our graph, $\hat{\theta}_1 = 2x_{v_1}^t$ is an estimator for the Bernoulli with probability $p_1 = \frac{1}{2}+\frac{1}{2\sqrt{t}}$ 
and $\hat{\theta}_2 = 2x_{v_2}^t$ is an estimator for the Bernoulli with probability $p_2 = \frac{1}{2}-\frac{1}{2\sqrt{t}}$.
To be more precise, since the protocol is executed in the same way in all nodes, $\hat{\theta}_1$ and $\hat{\theta}_2$ are the ouputs of one estimator $\hat{\theta}$, depending on the distribution 
from which this estimator gets its input. 
By combining inequalities~(\ref{eq:f(2t,t)}) and~(\ref{eq:mean}) we get:
\begin{equation}\label{eq:lecam}
f(2t+2,t) \geq \frac{1}{4} E_{p_1}\{\lp|\hat{\theta} - p_1\rp|\} + \frac{1}{4} E_{p_2}\{\lp|\hat{\theta} - p_2\rp|\}
\end{equation}
We will now try to bound the right hand side of equation~(\ref{eq:lecam}).
We first notice that:
\begin{align*}
E_{p_1}{\lp|\hat{\theta} - p_1\rp|} &\geq \delta \Exp{\mathds{1}\{\lp|\hat{\theta} - p_1\rp| > \delta\}}\\
&= \delta \Prob{\lp|\hat{\theta} - p_1\rp| > \delta}
\end{align*}
The same holds for $p_2$. It suffices now to provide a lower bound for this probability. 
In order to do this, we use a standard reduction from estimation problems to testing problems(see lecam). 
In a testing problem, there exist two Bernoulli distributions $P_1$ and $P_2$. First, nature selects with equal probability one of them and let $V \in \{1,2\}$ denote this random choice. 
Then, we draw a random vector $(Y_1,\ldots, Y_t)$ from the $t$-fold product distribution $P_V^t$. 
The goal is to decide whether $V=1$ or $V=2$. More precisely, we have to select a suitable function $\psi:\{0,1\}^t\mapsto \{1,2\}$ that minimizes $\Prob{\psi(Y_1,\ldots,Y_t) \neq V}$ 
First, we state a Lemma that provides a lower bound for this probability.
\begin{lemma}\label{l:lecam_lemma}
For every function $\psi:\{0,1\}^t\mapsto \{1,2\}$:
$$
\Prob{\psi(Y_1,\ldots,Y_t) \neq V} \geq \frac{1}{2} \lp(1 - D_{TV}(P_1^t, P_2^t)\rp)
$$
where $t$ is the number of samples.
\end{lemma}
We have:
\begin{align*}
\frac{1}{4} E_{p_1}\{\lp|\hat{\theta} - p_1\rp|\} + \frac{1}{4} E_{p_2}\{\lp|\hat{\theta} - p_2\rp|\} 
&= \frac{1}{4} E\{\lp|\hat{\theta} - p_1\rp|| V = 1\} + \frac{1}{4} E\{\lp|\hat{\theta} - p_2\rp|| V = 2\}\\
&\geq \frac{1}{4\sqrt{t}}\Prob{\hat{\theta} > \frac{1}{2}|V = 1} + \frac{1}{4\sqrt{t}}\Prob{\hat{\theta} < \frac{1}{2}|V = 2} 
\end{align*}
We define :
$$
\psi = 
\begin{cases}
1 \quad \hat{\theta} < \frac{1}{2}\\
2 \quad \hat{\theta} > \frac{1}{2}
\end{cases}
$$
As a result, 
\begin{align*}
\frac{1}{4\sqrt{t}}\Prob{\hat{\theta} > \frac{1}{2}|V = 1} + \frac{1}{4\sqrt{t}}\Prob{\hat{\theta} < \frac{1}{2}|V = 2} 
&\geq \frac{1}{2\sqrt{t}}\lp(\Prob{\hat{\theta} > \frac{1}{2},V = 1} + \Prob{\hat{\theta} > \frac{1}{2},V = 2}\rp)\\
&= \frac{1}{2\sqrt{t}}\lp(\Prob{\psi = 2,V = 1} + \Prob{\psi = 1,V = 2}\rp) \\
&= \frac{1}{2\sqrt{t}}\lp(\Prob{\psi \neq V}\rp)\\
&\geq \frac{1}{4\sqrt{t}}(1 - D_{TV}(P_1^t,P_2^t))\\
&\geq \frac{1}{24\sqrt{t}}
\end{align*}
So 
$$
\lim_{t \to \infty} \sqrt{t} f(2t+2,t) > 0
$$




\end{proof}
\subsection{Estimating Bernoulli Distributions}
The problem definition is the following:
\begin{itemize}
\item There exists an unkown parameter $p \in [0,1]$
\item We receive i.i.d. observations $X_i$ , where $X_i\sim B(p)$ 
\item We output an estimate value $\hat{p}=\hat{\theta_t}(X_1,\ldots,X_t)$ according to some function $\hat{\theta_t}:~ \{0,1\}^t\mapsto [0,1]$.
 \end{itemize}
The goal is to appropriately select $\hat{\theta_t}$ such that $\hat{p}$ and $p$ are relatively close. For example a very known selection is $$\hat{\theta_t}(X1,\ldots,X_t)= \frac{\sum_{i=1}^t X_i}{t}$$ which is the sample mean estimator

\begin{definition}
An estimator $\hat{\theta}$ is a colection of functions $\{\hat{\theta_t}\}_{t=1}^{\infty}$, where $\hat{\theta_t}: \{0,1\}^t\mapsto [0,1]$.
\end{definition} As the number of samples grows we would expect that an effiencient estimator $\hat{\theta}$ produces more and more accurate values for $p$. The following definition provides us the metric to evaluate the quality of an estimator $\hat{\theta}$. 

\begin{definition}
For any estimator $\hat{\theta_t}$ we define 
$$E_p[|\hat{\theta_t}(X_1,\ldots,X_t)| - p|]= \sum_{(x_1,\ldots,x^t)\in \{0,1\}^t}|\hat{\theta_t}(x^1,\ldots,x^t) -p| \cdot p^{\sum_{i=1}^tx^i}(1-p)^{t-\sum_{i=1}^tx^i}$$ 
\end{definition}
The quantity $E_p[|\hat{\theta_t}(X_1,\ldots,X_t)| - p|]$ is the expected distance of the estimated value $\hat{\theta_t}$ from the parameter $p$, when the distribution of the samples is $B(p)$. To simplify notation we also denote it as $E_p[|\hat{\theta_t} - p|]$.\\

\noindent Notice that $E_p[|\hat{\theta_t}-p|]$ does not serve as a good metric since an effiencient estimator $\hat{\theta}$ must garantee good estimates for any parameter $p$. Consider an estimator $\hat{\theta}$ that outputs $1/2$ for all $t$. In this case, $E_{p}[|\hat{\theta_t}- p|]=|p-1/2|$ meaning that this estimator performs optimally when $p=1/2$, but extremely bad in any other case. As a result the standard metric in the statistics literature is the following.

\begin{definition}
For any estimator $\hat{\theta_t}$ we define its risk with $t$ samples as, $$R^t(\hat{\theta})=\text{max}_{p \in [0,1]}E_p[|\hat{\theta_t} - p|]$$
\end{definition}Now our goal is clear, we need to design estimators $\hat{\theta_t}$ such that the risk $R^t(\hat{\theta_t})$ is as small as possible. For example one can easily prove that the risk rate of the \emph{sample mean estimator} ($\hat{\theta_t}= \frac{\sum_{i=1}^t X_i}{t}$) $R^t(\hat{\theta_t}) \leq \frac{1}{\sqrt{2t}}$. Obviously the next question is whether we can find another (probably more complex estimator) that acheives risk smaller than that of $\frac{1}{\sqrt{2t}}$.   

\begin{theorem}\label{thm: risk_optimality}
Any estimator $\hat{\theta}$ has risk $R^t(\hat{\theta}) \geq \frac{1}{16\sqrt{t}}$
\end{theorem}
The above theorem can be easily derived by standard techniques (e.g. Le Cam or Fano method) that have been develloped in the statistics literature to provide lower bounds on the risk of estimators. Especially, proving this lower bound for the Bernoulli case is requires a simple applications of these methods. The above theorem also tells us that the risk of the \emph{sample mean estimator} is up to constant factors optimal. Are we done?

\begin{theorem}
There exists an estimator $\hat{\theta}$ such that for all $p\in[0,1]$, $$E_p[|\hat{\theta_t} - p|] = \bigOh{\frac{1}{\sqrt{t}}}$$ 
\end{theorem} Can this rate be imporoved? For example, is there a an estimator $\hat{\theta}$ such that for all $p\in [0,1]$, $E_p[|\hat{\theta_t} - p|] = \bigOh{\frac{1}{t}}$?\\

\noindent It seems that the latter question can be answered negatively using Theorem~\ref{thm: risk_optimality}. Unfortunately, this is not the case. The difference is subtle but very important. The definition of the risk $R^t = \text{max}_{p \in [0,1]}E_p[|\hat{\theta_t}-p|]$, allows the maximum to be attained in a $p$ that depends on $t$ (e.g. $p=1/2+1/t$). As a result, Theorem~\ref{thm: risk_optimality} does not provide us with information about the above question.


\begin{theorem}
For any estimator $\hat{\theta}$ there exists a fixed $p\in[0,1]$ such that $$E_p[|\hat{\theta_t} - p|] = \Omega(\frac{1}{t})$$
\end{theorem}

\noindent To the best of our knowledge this question is not answered in the statistics literature. However, the same question has been asked for more the estimator of more general and complex distribution. These results ensure the above statement for estimators that satisfy certain propetries (unbaised estimators, locally regular, e.t.c.). In this section we provide a theorem that holds for any estimator and indicates that estimators that don't satisfy the above statement are very unlikely to exists.

\begin{theorem}
For any Bernoulli estimator $\hat{\theta}$, for all $[a,b] \subseteq [0,1]$,
$$ \lim_{t \to \infty}t \int_{a}^{b}E_p[|\hat{\theta_t} -p|]dp = +\infty$$
\end{theorem} In case that statement .. is not true than there exists an estimator $\hat{\theta}$ that simultaneous satisfies the following two properties:
\begin{enumerate}
 \item for all $p\in[0,1]$, $\lim_{t \to \infty}t E_p[|\hat{\theta_t} - p|]=0$
 \item for all $[a,b] \in [0,1]$, $\lim_{t \to \infty}t \int_a^b E_p[|\hat{\theta_t} - p|]=+\infty$
\end{enumerate}

We conjecture that there does not exists a function that satisfies both of these properties. Even in case that this is not true, such a function has a very degenerate form making it very difficult to be the risk function of an estimator.  

\subsection{Final Lower Bound}
\begin{theorem}
Let a graph oblivious protocol $P$ such that for any instance $I$, $E[||x^t - x^*||_{\infty}] \leq f(I)g(t)$, where $f: I\mapsto R_+$ and $g: N \mapsto R_+$. Then $\text{lim}_{t \to \infty}tf(t)\neq 0$.
\end{theorem}
\begin{proof}
We use the protocol $P$ to construct an estimator $\hat{\theta}$ and then we use the theorem ... For any $k,n \in N$ with $k \leq n$ we construct the following instance.
\begin{itemize}
 \item A star graph with $n+1$ nodes
 \item $k$ leaves have $a_i=1$ and $s_i=1$
 \item $n-k$ leaves have $a_i=1$ and $s_i=0$
 \item the central node has $a_{n+1}=1/2$ and $s_{n+1}=0$
\item directed edges from the central node to all the leaves.
 \end{itemize} Observe that for all $i \neq n+1$, $x^*_i=s_i$ and $x^*_{n+1}=\frac{k}{2n}$. For all $i\neq n+1$, $x_i^t=P(1,1,t)$ if $s_i=1$ and $x_i^t=P(0,1,t)$ if $s_i=0$. Notice that these are deterministic functions with respect to $t$ and are independent of the values of $k,n$. Now, $x_{n+1}^t=P(P(y_1,1,1),P(y_2,1,2),\dots,P(y_t,1,t),1/2,0,t)$, where $x_i \in {0,1}$. As a result, $x_{n+1}^t = g(y_1,\ldots,y_t)$, where $g: \{0,1\}^t \mapsto [0,1]$. As a result, the estimator $\hat{\theta}$ with $\hat{\theta_t} =\frac{x_c^t}{2}$ is valid estimator. Now let $p=k/n$ we have that \begin{eqnarray*}
  E_p[||x^t-x^*||] &\geq& E_p[|x_{n+1}^t-\frac{p}{2}|]\\
  &=& E_p[|\frac{\hat{\theta_t}}{2}-\frac{p}{2}|]\\
  &=& \frac{1}{2}E_p[|\hat{\theta_t}-p|]\\
 \end{eqnarray*} Finally, for all $p \in[0,1]~$  $E_p[|\hat{\theta_t}-p|] \leq 2f(I_{k,n})g(t)= f'(p)g(t)$. By theorem .., $\text{lim}_{t \to \infty}tf(t)\neq 0$.
 
 
 

 
 
\end{proof}

%\section{LeCam's lower bound}
Suppose $P = B(p), Q = B(q)$ are two Bernoulli distributions. 
We are first going to bound $D_{TV}(P^t, Q^t)$ by the hellinger distance of the two distributions.
\begin{lemma}\label{l:totalv_hell}
Suppose $P,Q$ are two arbitrary distributions. Then:
$$
D_{TV}(P,Q) \leq D_{hel}(P,Q) \sqrt{1-\frac{D_{hel}(P,Q)^2}{4}}
$$
\end{lemma}
Thus, in order to find a lower bound for $D_{TV}(P,Q)$ , it suffices 
to prove upper and lower bound for $D_{hel}(P,Q)$. This is done in the following lemmas. 
In the following we denote by $D = D_{hel}(P, Q)^2$.
\begin{lemma}\label{l:hell_n}
$$
D_{hel}(P^t, Q^t)^2 = 2 - 2\lp(1 - \frac{1}{2}D^2\rp)^t
$$
\end{lemma}

\begin{lemma}\label{l:upper_lower}
$$
2 - 2e^{-\frac{Dt}{2}} \leq D_{hel}(P^t,Q^t)^2 \leq
2 - 2e^{-Dt}
$$
\end{lemma}
\begin{proof}
Let's first prove the lower bound. We have:
\begin{equation}\label{eq:exp1}
\lp(1 - \frac{D}{2}\rp)^t \leq e^{-\frac{Dt}{2}}
\end{equation}
So, by Lemma~(\ref{l:hell_n}) we get:
\begin{align*}
d_hel(P^t,Q^t)^2 &= 2 - 2\lp(1 - \frac{D}{2}\rp)^t \\
&\geq 2 - 2e^{-\frac{Dt}{2}}
\end{align*}
For the upper bound, notice that for any $x\in[0,0.7]$ :
\begin{equation}\label{eq:exp2}
1 -x \geq e^{-2x}
\end{equation}
As we shall see later, if $p,q$ are close enough, $D$ will be small, so we can assum that $D/2 \in [0,0.7]$. So, by using inequality~(\ref{eq:exp2}) and Lemma~(\ref{l:hell_n})
we have:
\begin{align*}
d_{hel}(P^t,Q^t)^2 &= 2 - 2\lp(1 - \frac{D}{2}\rp)^t \\
&\leq 2 - 2 e^{-Dt}
\end{align*}
\end{proof}
\begin{lemma}\label{l:dtv_lower}
$$
D_{TV}(P^t,Q^t) \leq 1 - \frac{e^{-Dt}}{2}
$$
\end{lemma}
\begin{proof}
By Lemmata~(\ref{l:totalv_hell}) and ~(\ref{l:upper_lower}) we get:
\begin{align*}
D_{TV}(P^t,Q^t) &\leq D_{hel}(P^t,Q^t) \sqrt{1-\frac{D_{hel}(P^t,Q^t)^2}{4}}\\
&\leq \sqrt{2 - 2e^{-Dt}} \sqrt{1 - \frac{2 - 2e^{\frac{Dt}{t}}}{4}}\\
&= \sqrt{2}\sqrt{1 - e^{-Dt}} \sqrt{\frac{1}{2} - \frac{e^{-\frac{Dt}{2}}}{2}}\\
&= \sqrt{1 - e^{-Dt}} \sqrt{1 - e^{-\frac{Dt}{2}}}\\
&\leq 1 - \frac{e^{-Dt}}{2}
\end{align*}
since $e^{-\frac{Dt}{2}} \leq e^{-Dt}$. 
\end{proof}
The only thing that remains is to compute $D = D_{hel}(P,Q)$. We proceed to do this in the following lemma:
\begin{lemma}\label{l:hell}
If $p<q$ then
$$
D_{hel}(P,Q)^2 \leq \frac{1}{4}\lp(\frac{1}{p} + \frac{1}{1-q}\rp)(p-q)^2
$$
\end{lemma}
\begin{proof}
Since $P$ and $Q$ are Bernoulli, we have:
$$
D_{hel}(P,Q)^2 = \lp(\sqrt{p}- \sqrt{q}\rp)^2 + \lp(\sqrt{1-p}- \sqrt{1-q}\rp)^2
$$
We will bound the first term. We define the function $f:[p,q]\to \reals$:
$$
f(x) = \sqrt{x} - \sqrt{q}
$$
where $q$ is assumed to be constant. We have $f(p) = \sqrt{p} - \sqrt{q}$ and $ f(q) = 0$. $f$ is clearly continuously differentiable in $[p,q] $ with $f'(x) = \frac{1}{2\sqrt{x}}$,  
so by the mean value theorem:
$$
f(q)-f(p) = f'(\xi) (q-p)
$$
for some $\xi \in [p,q]$. We notice that $\lp|f'(\xi)\rp| \leq \frac{1}{\sqrt{p}}$. So
$$
\lp|f(q)-f(p)\rp| \leq  \frac{1}{\sqrt{p}}(q-p)
$$
\end{proof}

% \section{Lower bound: Graph Oblivious Protocols}

\subsection{Our Results for Graph Oblivious}
In the previous section, we have seen that in the unlabelled comunication model, a simple update rule acheives an error rate $E[||x^t-x^*||_{\infty}] \leq \sqrt{\log n}\frac{(\log t)^2}{\sqrt{t}}$ when $a\geq 1/2$. This rate is good enough from the perspective of the size of the graph, but very inefficient from the perspective of the steps needed to acheive a certain accuracy. More precisely, in order to acheive accuracy $\epsilon>0$, $\bigOh{\frac{\log n}{\epsilon^2}}$ steps are needed. On the other hand, in the labelled case another simple update rule needs $\bigOh{d^2\log \frac{1}{\epsilon}}$. The terms $\log n$ and $d^2$ are comparable, however $\log \frac{1}{\eps}$ and $\frac{1}{\eps^2}$ are of different order of magnitude. In this section, we investigate whether this gap between these two cases can be closed or is impact of a generic defieciency of the unlabelled communication model.   

\begin{definition}\label{def:graph_oblivious}
A graph oblivious update rule $F$, is a sequence of function $\{f_t\}_{t=1}^{\infty}$ where $f_t: [0,1]^{t+2} \mapsto [0,1]$.
\end{definition}

\noindent These restricted class of update rules is very natural since it is very reasonable to assume that the update rule is not aware of the graph structure of network and must work for any structure. Formally, this is imposed in Definition~\ref{def:graph_oblivious} since $F=\{f_t\}_{t=1}^{\infty}$ is selected prior to knowledge of $G$. In the next definition, we define the error of an update rule $F$ for an instance $I$.  
\begin{definition}\label{Graph Oblivious Updates Rules}
Let the graph oblivious update rule $F = \{f_t\}_{t=1}^{\infty}$. For an instance $I=(G,s,a)$ we define the following process. At round $t$:
\begin{itemize}
 \item Each agent $i$ has a state $x_i(t) \in [0,1]$
 \item For each agent $i$, the oracle $O_G$ randomly picks an agent $j \in N_i$ and returns the value $x_j(t-1)$ to $i$. Let $O_i^t$ denotes this value.
 \item Each agent $i$ updates her value as follows: $x_i(t) = f_t(O_i^1,\ldots,O_i^t,s_i,a_i)$
\end{itemize}
The error of the update rule $F$, for the instance $I$ is denoted by $E_I[||x(t)-x^*||_\infty]$, where $x^*$ is the equilibrium point of $I$.
\end{definition}

\noindent For example in the update rule $\ref{eq:unlabeled_update_rule}$, $f_t(O_1,\ldots,O_t,s_i,a_i) = (1-a_i)\frac{\sum_{j=1}^tO_j}{t} + a_is_i$ and we have shown that for any instance $I$ with $\alpha>1/2$, $E_I[||x(t)-x^*||_{\infty}] \in \bigOh{\frac{(\log t)^2}{\sqrt{t}}}$. We investigate whether the exists a graph oblivious update rule that for gatantees a better asympotic rate for all instances $I$. We conjecture that the following statement holds.

\begin{theorem}
For any graph oblivious update rule, $F=\{f_t\}_{t=1}^\infty$ there exists an instance $I$ such that 
$$E_I[||x_t-x^*||_\infty] \in \Omega(\frac{1}{t})$$
\end{theorem}
\noindent Although we were not able to prove the above claim, in the next section we provide a theorem that indicates that the existence of such a graph oblivious update rule is highly unlikely. In section .., we prove that such an update rule must satisfy simultaneous two techinical properties that excludes any reasonable update rule. We beleive that our difficulty is of tecnhical nature and we leave the proof of the above statement as an open problem.  
\subsection{Our results for No regret}
We have an agent that acts at the following simple setting. At every time $t$ the agent chooses $x_t \in [0,1]$. 
Then, he receives a function $f_t:[0,1]\to [0,1]$. The cost incurred by 
the agent for round $t$ is $f_t(x_t)$. 
The goal of the player is to pick $x_t$ in a way that minimizes the total cost for all rounds.
We emphasize the fact that the agent has to choose $x_t$ before he sees $f_t$, otherwise the problem becomes trivial. 
Formally:
\begin{definition}
Let $K$ be the set of all functions $f:[0,1]\to \R$. Then, a protocol is a sequence of functions $F_t:K^{t-1} \to \R$, where $x_t = F_t(f_1, \ldots,f_{t-1})$
\end{definition} 
In order to measure the performance of the agent, we define the following quantity:
\begin{definition}
If an agent runs the protocol for $T$ rounds, we define the regret of the protocol as:
\[
\sup_{\{f_1,\ldots ,f_T\}\subseteq K}\lp\{\sum_{t=1}^T f_t(x^t) - \min_{x\in[0,1]}\lp\{\sum_{t=1}^T f_t(x)\rp\}\rp\}
\]
\end{definition}
This quantity should ideally be a sublinear function of $T$, since this means that on average the cost incurred by the algorithm is close to 
the best decision in hindsight.
A protocol $P$ is said to be \emph{no regret} if the regret is $o(T)$.
In the next definition, we define the error of an update rule $F$ for an instance $I$.  
\begin{definition}\label{No regret Update Rules}
Let the no regret update rule $F = \{F_t\}_{t=1}^{\infty}$. For an instance $I=(G,s,a)$ we define the following process. At round $t$:
\begin{itemize}
 \item Each agent $i$ has a state $x_i(t) \in [0,1]$
 \item Each agent $i$ updates her value as follows: $x_i(t) = F_t(f_1,\ldots,f_{t-1})$
 \item For each agent $i$, the oracle $O_G$ randomly picks an agent $j \in N_i$ and let $O_i^t = j$. Then, the oracle returns the following function:
\[
f_t(x) = \alpha\lp(x - s_i\rp)^2 + (1 - \alpha)\lp(x - x_j(t-1)\rp)^2
\]

\end{itemize}
The error of the update rule $F$, for the instance $I$ is denoted by $E_I[||x(t)-x^*||_\infty]$, where $x^*$ is the equilibrium point of $I$.
\end{definition}
For example, if we have the protocol: 
\[
F_t(f_1, \ldots,f_{t-1}) = \argmin{x\in[0,1]}\sum_{i=1}^t f_i(x)
\]
then it is easy to see that 
\[
x_i(t) = (1-a_i)\frac{\sum_{j=1}^{t-1}x_{O_i^j}(t)}{t} + a_is_i
\]
which is just the weighted sample mean. 

\subsection{Estimating Bernoulli Distributions}
The problem definition is the following:
\begin{itemize}
\item There exists an unkown parameter $p \in [0,1]$
\item We receive i.i.d. observations $X_i$ , where $X_i\sim B(p)$ 
\item We output an estimate value $\hat{p}=\hat{\theta_t}(X_1,\ldots,X_t)$ according to some function $\hat{\theta_t}:~ \{0,1\}^t\mapsto [0,1]$.
 \end{itemize}
The goal is to appropriately select $\hat{\theta_t}$ such that $\hat{p}$ and $p$ are relatively close. For example a very known selection is $$\hat{\theta_t}(X1,\ldots,X_t)= \frac{\sum_{i=1}^t X_i}{t}$$ which is the sample mean estimator

\begin{definition}
An estimator $\hat{\theta}$ is a colection of functions $\{\hat{\theta_t}\}_{t=1}^{\infty}$, where $\hat{\theta_t}: \{0,1\}^t\mapsto [0,1]$.
\end{definition} As the number of samples grows we would expect that an effiencient estimator $\hat{\theta}$ produces more and more accurate values for $p$. The following definition provides us the metric to evaluate the quality of an estimator $\hat{\theta}$. 

\begin{definition}
For any estimator $\hat{\theta_t}$ we define 
$$E_p[|\hat{\theta_t}(X_1,\ldots,X_t)| - p|]= \sum_{(x_1,\ldots,x^t)\in \{0,1\}^t}|\hat{\theta_t}(x^1,\ldots,x^t) -p| \cdot p^{\sum_{i=1}^tx^i}(1-p)^{t-\sum_{i=1}^tx^i}$$ 
\end{definition}
The quantity $E_p[|\hat{\theta_t}(X_1,\ldots,X_t)| - p|]$ is the expected distance of the estimated value $\hat{\theta_t}$ from the parameter $p$, when the distribution of the samples is $B(p)$. To simplify notation we also denote it as $E_p[|\hat{\theta_t} - p|]$.\\

\noindent Notice that $E_p[|\hat{\theta_t}-p|]$ does not serve as a good metric since an effiencient estimator $\hat{\theta}$ must garantee good estimates for any parameter $p$. Consider an estimator $\hat{\theta}$ that outputs $1/2$ for all $t$. In this case, $E_{p}[|\hat{\theta_t}- p|]=|p-1/2|$ meaning that this estimator performs optimally when $p=1/2$, but extremely bad in any other case. As a result the standard metric in the statistics literature is the following.

\begin{definition}
For any estimator $\hat{\theta_t}$ we define its risk with $t$ samples as, $$R^t(\hat{\theta})=\text{max}_{p \in [0,1]}E_p[|\hat{\theta_t} - p|]$$
\end{definition}Now our goal is clear, we need to design estimators $\hat{\theta_t}$ such that the risk $R^t(\hat{\theta_t})$ is as small as possible. For example one can easily prove that the risk rate of the \emph{sample mean estimator} ($\hat{\theta_t}= \frac{\sum_{i=1}^t X_i}{t}$) $R^t(\hat{\theta_t}) \leq \frac{1}{\sqrt{2t}}$. Obviously the next question is whether we can find another (probably more complex estimator) that acheives risk smaller than that of $\frac{1}{\sqrt{2t}}$.   

\begin{theorem}\label{thm: risk_optimality}
Any estimator $\hat{\theta}$ has risk $R^t(\hat{\theta}) \geq \frac{1}{16\sqrt{t}}$
\end{theorem}
The above theorem can be easily derived by standard techniques (e.g. Le Cam or Fano method) that have been develloped in the statistics literature to provide lower bounds on the risk of estimators. Especially, proving this lower bound for the Bernoulli case is requires a simple applications of these methods. The above theorem also tells us that the risk of the \emph{sample mean estimator} is up to constant factors optimal. Are we done?

\begin{theorem}
There exists an estimator $\hat{\theta}$ such that for all $p\in[0,1]$, $$E_p[|\hat{\theta_t} - p|] = \bigOh{\frac{1}{\sqrt{t}}}$$ 
\end{theorem} Can this rate be imporoved? For example, is there a an estimator $\hat{\theta}$ such that for all $p\in [0,1]$, $E_p[|\hat{\theta_t} - p|] = \bigOh{\frac{1}{t}}$?\\

\noindent It seems that the latter question can be answered negatively using Theorem~\ref{thm: risk_optimality}. Unfortunately, this is not the case. The difference is subtle but very important. The definition of the risk $R^t = \text{max}_{p \in [0,1]}E_p[|\hat{\theta_t}-p|]$, allows the maximum to be attained in a $p$ that depends on $t$ (e.g. $p=1/2+1/t$). As a result, Theorem~\ref{thm: risk_optimality} does not provide us with information about the above question.


\begin{theorem}
For any estimator $\hat{\theta}$ there exists a fixed $p\in[0,1]$ such that $$E_p[|\hat{\theta_t} - p|] = \Omega(\frac{1}{t})$$
\end{theorem}

\noindent To the best of our knowledge this question is not answered in the statistics literature. However, the same question has been asked for more the estimator of more general and complex distribution. These results ensure the above statement for estimators that satisfy certain propetries (unbaised estimators, locally regular, e.t.c.). In this section we provide a theorem that holds for any estimator and indicates that estimators that don't satisfy the above statement are very unlikely to exists.

\begin{theorem}
For any Bernoulli estimator $\hat{\theta}$, for all $[a,b] \subseteq [0,1]$,
$$ \lim_{t \to \infty}t \int_{a}^{b}E_p[|\hat{\theta_t} -p|]dp = +\infty$$
\end{theorem} In case that statement .. is not true than there exists an estimator $\hat{\theta}$ that simultaneous satisfies the following two properties:
\begin{enumerate}
 \item for all $p\in[0,1]$, $\lim_{t \to \infty}t E_p[|\hat{\theta_t} - p|]=0$
 \item for all $[a,b] \in [0,1]$, $\lim_{t \to \infty}t \int_a^b E_p[|\hat{\theta_t} - p|]=+\infty$
\end{enumerate}

We conjecture that there does not exists a function that satisfies both of these properties. Even in case that this is not true, such a function has a very degenerate form making it very difficult to be the risk function of an estimator.  

\subsection{Reductions form update rules to Bernoulli Estimators}

\begin{lemma}
For any no regret protocol $P$, there exists a Bernulli estimator $\hat{\theta}$ such that\\
for all $q \in Q([0,1])$, there exists an instance $I_q$ such that $$E_q[|\hat{\theta_t} - q|] \leq 2E_{I_q}[||x^t-x^*||_{\infty}]$$
\end{lemma}

\begin{proof}
At round $t$, the graph oblivious protocol $P$ has selected the functions $F_t(f_1,\ldots,f_{t-1})$. An estimator $\hat{\theta}$ receives the samples $Y_1,\ldots,Y_t$ and outputs the value $\hat{\theta_t}(Y_1,\ldots,Y_t) \in [0,1]$. By Lemma~(\ref{l:help}) the opinions of neighbours are functions of $Y_1, \ldots, Y_{t-2}$. 
We define:
\[
g_t(Y_1,\ldots,Y_t) = g_{Y_t}^t(Y_1,\ldots,Y_t) 
\]
We consider the estimator $\hat{\theta_t}=\frac{1}{2}F_t(g_1(Y_1),\ldots,g_t(Y_1,\ldots, Y_t))$ and observe that $\hat{\theta_t}$ is a function $\{0,1\}^t \mapsto [0,1]$.\\
Now for any $q \in Q([0,1])$, construct an appropriate instance $I_q$ such that $E_q[|\hat{\theta_t} - q|] \leq 2E_{I_q}[||x^t-x^*||_{\infty}]$.\\
Since $q=\frac{k}{n}$ some $k,n \in \nats$, the following instance $I_q$ with $n+1$ agents:
\begin{itemize}
 \item A central agent with $s_c=0$ and $a_c=1/2$.
 \item Directed edges from the central agent to all the other agents.
 \item $k$ agents with $s_i=0$ and $a_i=1$
 \item $n-k$ agents with $s_i=1$ and $a_i=1$
 \end{itemize}
Notice that in this instance for all $i\neq c$, $x^*_i=s_i$ and $x^*_c=\frac{q}{2}$. We show that $E_q[|\hat{\theta_t}-q|] \leq 2E_{I_q}[||x^t-x^*||_{\infty}]$.\\
At round $t$, if the oracle returns to the center agent the value $g_t(1,1)$ of a $1$-agent, then $Y_t=1$ otherwise $Y_t=0$. As a result, $\Prob{Y_t=1}=q$ and 
\begin{align*}
 E_{I_q}[||x^t-x^*||_{\infty}] &\geq E_{I_q}[|x^t_c-x^*_c|]\\
 &= E_{q}[|\frac{\hat{\theta_t}}{2}-\frac{q}{2}|]
\end{align*}

\end{proof}

\begin{lemma}\label{l:help}
Suppose the no regret protocol $F = \{F_i\}_{i=1}^t $ runs in the topology defined in the previous lemma. We define the sequence $b_1, b_2, \ldots , b_t$ such that 
$b_i = s$, where $s$ is the $s_i$ of the neighbour of center that was selected at time $i$. Then , for every time $i$ there are functions $g_0^i, g_1^i$ such that for all neighbours $j$ with $s_i = 0$ :
$x_j(i) = g_0^i(b_1,\ldots, b_{i-2})$ and for all neighbours $j$ with $s_i = 1$ :
$x_j(i) = g_1^i(b_1,\ldots, b_{i-2})$.
\end{lemma}
\begin{proof}
We are going to prove the statement inductively for $i$. If $i=1$, then each agent hasn't seen any function yet and should make a choice for $x(1)$.
 By the definition of $F_1$, all the agents select the same value, so $g_0^1 = g_1^1 = F_1$. We verified the base case.

Suppose the statement holds for all times $i \leq k$. In particular, it holds for $i = k$, so there exist $g_0^k, g_1^k$ 
that are functions of $b_1,\ldots, b_{k-2}$ that satisfy the proposition. Since all neighbouring values up to time $k$ are functions of $b_1,\ldots, b_{k-2}$
and since the functions that the center node receives are determined by the opinions of the neighbours and the choice at round $k-1$,
we get that $x_c(k)$ is a function of $b_1,\ldots, b_{k-1}$. At the end of round $k$, each agent learns one more function
based on the opinion of one of its neighbours. Observe that every neigbour of the center has only the center as neighbour. As a result, all nodes
with $s_i = 0$ receive the same function $\phi_1$ and all nodes with $s_i = 1$ receive the same function $\phi_2$. We notice that $\phi_1$ and $\phi_2$ are uniquely determined by $x_c(k)$ and thus of $b_1,\ldots, b_{k-1}$\\
Now, at time $k+1$ first each agent evaluates the new opinion, based on the new function he received. Since the function that each node received 
is determined by his $s_i$ and $x_c(k)$, it follows that all nodes that have the same $s_i$ have the same value, which is a function of $b_1,\ldots, b_{k-1}$. The inductive step is now complete. 
\end{proof}

\section{Graph Aware Protocol - Frequencies}
Let $P$ be a discrete distribution over $[d]$, and let
$S_1, \ldots, S_t$ be $t$ i.i.d samples drawn from $P$, i.e.
$(S_1, \ldots, S_t) \sim P^t$.
The empirical distribution $\hat{P}_t$ is the following estimator of
the density of $P$.
\begin{equation}\label{eq:empirical_distribution}
  \hat{P}_N(A) = \frac{\sum_{i=1}^N \1{S_i \in A}}{N},
\end{equation}
where $A \subseteq [d]$. In words, $\hat{P}_N$ simply counts
how many times the value $i$ appeared in the samples $S_1, \ldots S_N$.
We will use the following version of the classical Vapnik and
Chervonenkis inequality.
\begin{lemma}\label{l:vc_inequality}
  Let $\mcal{A}$ be a collection of subsets of $\{0,\ldots, d\}$ and
  Let $S_{\mcal{A}}(N)$ be the Vapnik-Chervonenkis shatter coefficient, defined
  by
  \[
    S_{\mcal{A}(N)} = \max_{x_1,\ldots, x_t \in [d]}
    \lp| \{ \{x_1, \ldots, x_t\} \cap A : A \in \mcal{A} \} \rp|.
  \]
  Then
  \[
    \Expnew{P^N}{\max_{A \in \mcal{A}} \lp| \hat{P}_t(A) - P(A) \rp|}
    \leq 2 \sqrt{\frac{\log{2 S_{\mcal{A}}(N)}}{N}}
  \]
\end{lemma}

An interesting question is whether $\mrm{poly}(1/\eps)$ rounds
are necessary when protocols have access to the unlabeled oracle.
We show that this is not true by showing a protocol
that needs $\ln^2(1/\eps)$ rounds to be within error $\eps$ from
the solution $x^*$. Our protocol is graph-aware in the sense that
it is allowed to depend on the graph $G$. Therefore, we show that
without any constraints on the protocols it is hard to prove strong
lower bounds for our problem. The problem in this case is that
all agents could agree to stop updating their public opinions for
enough rounds so that everybody learns, with high probability, exactly
the average of the opinions of their neighbors. Having the exact
average they can perform a step of the original update of the FJ-model
which results in a similar and fast convergence rate.
Of course if the opinions of the neighbors are all different we can still
use an argument similar to that of the protocol of section COUPONS COLLECTOR.
Since some neighbors may share opinions we have to come up with a different
solution for this problem. Let $L \leq d$ be the number of different
opinions and $k_i$ be the number of neighbors who have opinion $\xi_i$.
Then the average of the opinions of the neighbors is
$(\sum_{j = 1}^L k_i \xi_i)/d_i$. Instead of using the average like
we did in section SAMPLE MEAN, we will instead find \emph{exactly}
the frequencies $k_i$ and then compute their average exactly with high
probability.
We next describe our protocol. Since it is the same for all agents,
For simplicity we describe it for agent $i$.
\begin{algorithm}
  \caption{Graph Aware Update Rule}
  \label{alg:frequencies}
  \begin{algorithmic}[1]
    \State $x_i(0) \gets s_i$, $t \gets 0$.
    \For{$l = 1, \ldots, \infty$}
    \State $\eps \gets 1/2^l$.
    \For{$t = 1, \ldots,\ln(1/\eps))$}
    \State Keep a map $h$ from $[0,1]$ to $[d_i]$
    and array of counters $A$ of length $d_i$ and an array $B$ of length
    $M_1$.
    \State $M_1 = O(\ln(1/\eps))$,
    $M_2 = O(d^2 \ln(d))$

    \For{$j = 1, \ldots, M_1$}

    \For{$k = 1, \ldots, M_2$}
    \State Call the unlabeled oracle $O_i^u(t)$ to get an opinion $X_i$.
    \If{$X_i$ is not in $h$}
      \State Insert $X_i$ to $h$.
    \Else
      \State $A(h(X_i)) \gets A(h(X_i)) +1$.
    \EndIf
  \State $t \gets t+1$
\EndFor
  \State Divide all entries of $A$ by $M_2$.  \label{alg:line:counters}
  \State Round all entries of $h$ to the closest multiple of $1/d$.
  \State $B(j) = \sum_{i=1}^{d_i} A(i)$.
\EndFor
\State $x_i(t) \gets \maj_{j} B(j)$.
\EndFor
\EndFor
\end{algorithmic}
\end{algorithm}

\begin{theorem}
  The update rule~\ref{alg:frequencies} for $a > 1/2$ achieves
  convergence rate
  \[
    \Exp{\norm{\infty}{x_t- x^*}}
    \leq C \me^{-\sqrt{t}/(d\sqrt{\log{d}})},
\]
where $C$ is a universal constant.
\end{theorem}
\begin{proof}
  According to the update rule~\ref{alg:frequencies}
  all agents fix their opinions $x_i(t)$ for $M_1 \times M_2$ rounds.
  To estimate the sum of the opinions each agent estimates the
  frequencies $k_j/d_i$. Since the neighbors have at most $d_i$
  different opinions we can think of the opinions as natural
  numbers in $[d_i]$.  The oracle returns the opinion
  of a random neighbor and therefore the samples $X_i$ returned by
  the oracle $O_i^u$ are drawn from a discrete distribution
  $P$ supported on $[d_i]$.
  The probability $P(j)$ of the $j$-th opinion is the number of neighbors
  having this opinion $k_j/d_i$.
  To learn the probabilities $P(j)$ using samples from $P$.
  Letting $\mcal{A} = \{\{1\},\{2\},\ldots, \{d_i\}\}$ we have from
  Lemma~\ref{l:vc_inequality} that
  \[
    \Expnew{P^m}{\max_{j \in [d_i]} \lp| \hat{P}_m(j) - P(j) \rp|}
    \leq 2 \sqrt{\frac{\log{2 d_i}}{m}},
  \]
  since $S_{\mcal{A}} \leq d_i$. Therefore, an agent can draw
  $m = 100 d^2 \log(2 d)$ to learn the frequencies $k_j/d_i$
  within expected error $1/(5 d)$.
  Notice now that the array $A$ after line~\ref{alg:line:counters}
  corresponds to the empirical distribution of
  equation~(\ref{eq:empirical_distribution}).
  Notice that if the agents have estimations of the frequencies $k_j/d_i$
  with error smaller than $1/d$ then by rounding them to the closest
  multiple of $1/d_i$ they learn the frequencies exactly.
  By Markov's inequality we have that with probability at least $4/5$ the
  rounded frequencies are exactly correct. By standard Chernoff bounds we
  have that if the agents repeat the above procedure
  $\ln(1/\delta)$ times and
  keep the most frequent of the answers $B(j)$ then they will
  obtain the correct answer with probability at least
  $1-\delta$. From Lemma~\ref{l:generic_convergence} we have
  that to achieve error $\eps$ we need $\log(1/\eps)/(1/(1-\alpha))$ rounds.
  Since we need all nodes to succeed at computing the exact averages for
$\log(1/\eps)/\log(1/(1-\alpha))$ rounds we have from the union bound that for
$\delta < \frac{\eps \ln(1/(1-\alpha))}{n \ln(1/\eps)}$, with
probability at least $1-\eps$ the error is at most $\eps$.
For the expected error after
$T = O(d^2 \log d ( \log(n/\eps) + \log \log(1/\eps) + \log \log (1-a))$ rounds
we have that $\Exp{\norm{\infty}{x_T - x^*}} = O(\eps)$.
\end{proof}

%
% ---- Bibliography ----
%
% BibTeX users should specify bibliography style 'splncs04'.
% References will then be sorted and formatted in the correct style.
%

\bibliographystyle{splncs04}
\bibliography{refs}

\end{document}
