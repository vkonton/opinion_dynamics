\documentclass[a4paper,UKenglish]{templates/lipics-v2018-authors/lipics-v2018}
%This is a template for producing LIPIcs articles.
%See lipics-manual.pdf for further information.
%for A4 paper format use option "a4paper", for US-letter use option "letterpaper"
%for british hyphenation rules use option "UKenglish", for american hyphenation rules use option "USenglish"
% for section-numbered lemmas etc., use "numberwithinsect"
\usepackage[margin=1in]{geometry}
\usepackage[T1]{fontenc}
\usepackage{microtype}

\usepackage{kpfonts}
 % \usepackage{mathpazo}
% \usepackage{minionpro}
% \usepackage[boldsans]{ccfonts}
% \usepackage{beton}
% \usepackage{concrete}
%  \usepackage{sectsty}
%  \allsectionsfont{\mdseries}
%\usepackage{beton}
%\renewcommand{\bfdefault}{sbc}
%\usepackage{kurier}
% \usepackage{eulervm}

% \usepackage{ntheorem}


% allows for indexgeneration
\usepackage{makeidx}

\usepackage{amsfonts}
\usepackage{amsmath}
\usepackage{mathtools}

\usepackage{hyperref}
\hypersetup{
  colorlinks = true,
  urlcolor = {blue},
  citecolor = {blue}
}


%fonts.
%\usepackage{fontspec}

%\usepackage{newpxtext}
%\usepackage[euler-digits]{eulervm}

%%\setmainfont{Linux Libertine}

\usepackage{textcomp}
\usepackage{csquotes}

\usepackage{booktabs}
\usepackage[scale=2]{ccicons}
\usepackage{graphicx}

\usepackage{pgfplots}
\usepgfplotslibrary{dateplot}
\usepackage{color}

\usepackage{listings, color}
\definecolor{green}{rgb}{0,0.5,0}
\definecolor{gray}{rgb}{0.4,0.4,0.4}
\definecolor{lblue}{rgb}{0.9,0.9,1}
\definecolor{red}{rgb}{0.9,0,0}
\definecolor{blue}{rgb}{0.6,0,0.6}

\DeclareSymbolFont{matha}{OML}{txmi}{m}{it}% txfonts
\DeclareMathSymbol{\varv}{\mathord}{matha}{118}


% bold vectors
\usepackage{bm}

% mathbb numbers
\usepackage{dsfont}

% mathlarger
\usepackage{relsize}

\usepackage{algorithm}
\usepackage[noend]{algpseudocode}
\makeatletter
\def\BState{\State\hskip-\ALG@thistlm}
\makeatother

\lstset{
  backgroundcolor=\color{lblue},
  basicstyle=\scriptsize\ttfamily,
  breaklines=true,
  columns=fullflexible,
  commentstyle=\color{gray},
  keywordstyle=\color{blue},
  stringstyle=\color{red},
  identifierstyle=\color{black},
  language=haskell,
  %numbers=left,
  %numberstyle=\scriptsize,
  %numbersep=5pt,
  showstringspaces=false
}

\usepackage{subcaption}

 \definecolor{1}{rgb}{0.368417, 0.506779, 0.709798}
 \definecolor{2}{rgb}{0.880722, 0.611041, 0.142051}
 \definecolor{3}{rgb}{0.560181, 0.691569, 0.194885}
 \definecolor{4}{rgb}{0.922526, 0.385626, 0.209179}
 \definecolor{5}{rgb}{0.528488, 0.470624, 0.701351}
 \definecolor{6}{rgb}{0.772079, 0.431554, 0.102387}
 \definecolor{7}{rgb}{0.363898, 0.618501, 0.782349}
 \definecolor{8}{rgb}{1, 0.75, 0}
 \definecolor{9}{rgb}{0.647624, 0.37816, 0.614037}
 \definecolor{10}{rgb}{0.571589, 0.586483, 0.}
 \definecolor{11}{rgb}{0.915, 0.3325, 0.2125}
 \definecolor{12}{rgb}{0.40082222609352647, 0.5220066643438841, 0.85}
 \definecolor{13}{rgb}{0.9728288904374106, 0.621644452187053, 0.07336199581899142}
 \definecolor{14}{rgb}{0.736782672705901, 0.358, 0.5030266573755369}
 \definecolor{15}{rgb}{0.28026441037696703, 0.715, 0.4292089322474965}

% MACROS.

%display style in FRAC
\newcommand\ddfrac[2]{\frac{\displaystyle #1}{\displaystyle #2}}


\newcommand{\lp}{\left}
\newcommand{\rp}{\right}
\newcommand{\mrm}[1]{\mathrm{#1}}
\newcommand{\mbb}[1]{\mathbb{#1}}
\newcommand{\mcal}[1]{\mathcal{#1}}
\newcommand{\mfrak}[1]{\mathfrak{#1}}
\DeclareMathOperator*{\argmin}{argmin}
\DeclareMathOperator*{\argmax}{argmax}
\DeclareMathOperator*{\maj}{maj}

%Tikz crap.
\newcommand*{\equal}{=}
\newcommand*{\comma}{,}

% Asymptotic Notation.
\newcommand{\bigOh}[1]{O\lp(#1\rp)}
\newcommand{\smalloh}[1]{o\lp(#1\rp)}
\newcommand{\bigOmega}[1]{\Omega\lp(#1\rp)}
\newcommand{\smallomega}[1]{\omega\lp(#1\rp)}
\newcommand{\poly}{\mrm{poly}}

% Standard sets
\newcommand{\R}{\bm{\mrm{R}}}
\newcommand{\N}{\bm{\mrm{N}}}
\newcommand{\Q}{\bm{\mrm{Q}}}
\newcommand{\Z}{\bm{\mrm{Z}}}

% Constants.
\newcommand{\me}{\mathrm{e}}

% Analysis.
\def\d{\mrm{d}}
\newcommand\Ball[2]{\mcal{B}\lp(#1, #2 \rp)}
\newcommand\norm[2]{\| #2 \|_{#1}}

% Functions.
\newcommand{\tr}{\mathrm{tr}}
\newcommand{\1}[1]{\bm{1}{\left[#1\right]}}
\newcommand{\Real}[1]{\mathrm{Re}\lp(#1\rp)}
\newcommand{\Imag}[1]{\mathrm{Im}\lp(#1\rp)}
\DeclarePairedDelimiter\ceil{\lceil}{\rceil}
\DeclarePairedDelimiter\floor{\lfloor}{\rfloor}

% Distances & Probability.
\newcommand{\Normal}[2]{\mcal{N}\lp(#1,#2\rp)}
\newcommand{\DNormal}[2]{\mrm{DN}\lp(#1,#2\rp)}
\newcommand{\GBD}{\mrm{GBD}}
\newcommand{\PBD}{\mrm{PBD}}
\newcommand{\Bin}{\mrm{Bin}}
\newcommand{\Pois}{\mrm{Pois}}
\newcommand{\Distr}{\mathcal{L}}
\newcommand{\Unif}{\mrm{U}}

% \newcommand{\GBD}[1][]{
%   \def\ArgI{{#1}}%
%   \GBDRelay
% }
% \newcommand{\GBDRelay}[1]{
%   \mrm{GBD}\lp( \ArgI, #2 \rp)
% }

\newcommand{\NormalPDF}[3]
{
  \frac{1}{\sqrt{2 \pi} #2} \me^{-\frac{(#3 - #1)^2}{2 {#2}^2}}
}
\newcommand{\erf}[1]{\mrm{erf}\lp(#1\rp)}
\newcommand{\erfc}[1]{\mrm{erfc}\lp(#1\rp)}

\newcommand{\Prob}[1]{\bm{\mrm{P}}\lp[#1\rp]}

\newcommand{\Exp}[1][]{
  \def\ArgI{#1}%
  \ExpRelay
}
\newcommand{\ExpRelay}[1]{
  \bm{\mrm{E}}_{\ArgI} \lp[ #1 \rp]
}

\newcommand{\Expnew}[2]{
  \bm{\mrm{E}}_{#1} \lp[ #2 \rp]
}


\newcommand{\Var}[1]{\bm{\mrm{V}}\lp[#1\rp]}
\newcommand{\Cov}[2]{\mrm{Cov}\lp[#1, #2\rp]}
\newcommand{\Kurt}[1]{\mrm{Kurt}\lp[#1 \rp]}

\newcommand{\fdiv}[3]{D_{#1} \lp(#2\|#3\rp)}
\newcommand{\dkl}[2]{D_{\mrm{kl}} \lp(#1\|#2\rp) }
\newcommand{\dtv}[2]{d_{\mrm{tv}} \lp(#1, #2\rp) }
\newcommand{\dhel}[2]{d_{\mrm{hel}} \lp(#1, #2\rp) }
\newcommand{\dkol}[2]{d_{\mrm{kol}} \lp(#1, #2\rp) }

% minimax risk
\newcommand{\minimax}[1]{\mfrak{M}_{#1}}

\newcommand{\h}[1]{\hat{#1}}

% PBD powers specific macros.
\def\eps{\varepsilon}
\def\p{\hat{p}}
\def\q{\hat{q}}

\def\reals{\bm{\mrm{R}}}
\def\nats{\mathbb{N}}
\def\D{\mathcal{D}}

\def\P{\mathcal{P}}
\def\p{\hat{p}}
\def\q{\hat{q}}
\newcommand{\err}[1]{\mbox{\rm err}\left(#1\right)}


% other.

\newcommand{\sline}[1]{\textcolor{#1}{\linethickness{.5mm}\line(1,0){15}} }

% Ident in algorithmic.
\newlength\myindent
\setlength\myindent{2em}
\newcommand\bindent{%
  \begingroup
  \setlength{\itemindent}{\myindent}
  \addtolength{\algorithmicindent}{\myindent}
}
\newcommand\eindent{\endgroup}

\usepackage{amsmath,amssymb}
\usepackage{microtype}%if unwanted, comment out or use option "draft"

%\graphicspath{{./graphics/}}%helpful if your graphic files are in another directory

\bibliographystyle{plainurl}% the recommnded bibstyle

\title{Dummy title}

\titlerunning{Dummy short title}%optional, please use if title is longer than one line

\author{John Q. Public}{Dummy University Computing Laboratory, [Address], Country}{johnqpublic@dummyuni.org}{https://orcid.org/0000-0002-1825-0097}{[funding]}%mandatory, please use full name; only 1 author per \author macro; first two parameters are mandatory, other parameters can be empty.

\author{Joan R. Public}{Department of Informatics, Dummy College, [Address], Country}{joanrpublic@dummycollege.org}{[orcid]}{[funding]}

\authorrunning{J.\,Q. Public and J.\,R. Public}%mandatory. First: Use abbreviated first/middle names. Second (only in severe cases): Use first author plus 'et al.'

\Copyright{John Q. Public and Joan R. Public}%mandatory, please use full first names. LIPIcs license is "CC-BY";  http://creativecommons.org/licenses/by/3.0/

\subjclass{Dummy classification}% mandatory: Please choose ACM 2012 classifications from https://www.acm.org/publications/class-2012 or https://dl.acm.org/ccs/ccs_flat.cfm . E.g., cite as "General and reference $\rightarrow$ General literature" or \ccsdesc[100]{General and reference~General literature}.

\keywords{Dummy keyword}%mandatory

\category{}%optional, e.g. invited paper

\relatedversion{}%optional, e.g. full version hosted on arXiv, HAL, or other respository/website

\supplement{}%optional, e.g. related research data, source code, ... hosted on a repository like zenodo, figshare, GitHub, ...

\funding{}%optional, to capture a funding statement, which applies to all authors. Please enter author specific funding statements as fifth argument of the \author macro.

\acknowledgements{I want to thank \dots}%optional

%Editor-only macros:: begin (do not touch as author)%%%%%%%%%%%%%%%%%%%%%%%%%%%%%%%%%%
\EventEditors{John Q. Open and Joan R. Access}
\EventNoEds{2}
\EventLongTitle{42nd Conference on Very Important Topics (CVIT 2016)}
\EventShortTitle{CVIT 2016}
\EventAcronym{CVIT}
\EventYear{2016}
\EventDate{December 24--27, 2016}
\EventLocation{Little Whinging, United Kingdom}
\EventLogo{}
\SeriesVolume{42}
\ArticleNo{23}
%\nolinenumbers %uncomment to disable line numbering
%\hideLIPIcs  %uncomment to remove references to LIPIcs series (logo, DOI, ...), e.g. when preparing a pre-final version to be uploaded to arXiv or another public repository
%%%%%%%%%%%%%%%%%%%%%%%%%%%%%%%%%%%%%%%%%%%%%%%%%%%%%%

\begin{document}

\maketitle

\begin{abstract}
Lorem ipsum dolor sit amet, consectetur adipiscing elit. Praesent
	convallis orci arcu, eu mollis dolor. Aliquam eleifend
	suscipit lacinia. Maecenas quam mi, porta ut lacinia sed,
	convallis ac dui. Lorem ipsum dolor sit amet, consectetur
	adipiscing elit. Suspendisse potenti.
\end{abstract}

\section{Introduction}

\subsection{Friedkin-Johsen Model}
In the Friedkin-Johsen model (FJ-model) an undirected graph $G(V,E)$ with $n$
nodes, is assumed, where $V$ denotes the agents and $E$ the social relations between
them. Each agent $i$ poses an internal opinion $s_i \in [0,1]$ and a self confidence coefficient
$\alpha_i \in[0,1]$. At each round $t\geq 1$, agent $i$ updates her opinion
as follows:

\begin{equation}\label{eq:best_response}
  x_i(t) = (1-\alpha_i) \frac{\sum_{j \in N_i}x_j(t-1)}{|N_i|} + \alpha_i s_i,
\end{equation}
where $N_i$ is the set of her neighbors. The simplicity of its update rule makes it plausible, because in
real social networks it is very unlikely that agents change their opinions
according to complex rules. This model has been studies from several perspectives.
Based on the above model, in \cite{BKO11}, they propose a game
in which the strategy that each agent $i$ plays, is the opinion $x_i \in [0,1]$
that she publicly expresses incurring her a cost
\begin{align}\label{eq:kleinberg_cost}
  C_i(x) = (1-a_i) \sum_{j \in N_i} (x_i - x_j)^2 + a_i (x_i - s_i)^2.
\end{align}
They prove that this game admits a unique equilibrium point and that
the price of anarchy is $9/8$ for undirected graphs and $O(n)$ for
directed networks.  In \cite{GS14} they study they show that the update
rule (\ref{eq:best_response}) is the \emph{best response} play for the above
game and they also study its convergence properties to the equilibrium point
$x^* \in [0,1]^n$. Therefore, Friedkin-Johsen model admits several nice properties
as it very simple to execute by the agents, rapidly converges to its equilibrium,
and it is a natural behavior for selfish agents.

Our work is motivated by the fact that the FJ-model requires that
agents interact with all their neighbors at each round. This requirement
could be under discussion in today's large social networks, where each individual
could have several hundreds of friends (neighbors). In such graphs it is quite
unnatural to assume that everybody updates her opinion using the opinions of all
her friends. In this work we propose simple and natural models, similar to
the original FJ-model, that require minimal interaction between agents at each
round with similar convergence properties. More precisely, we assume that
each agent can only interact with only one of her neighbors at each round.
To make our setting precise we define the following oracles.
\begin{itemize}
  \item The \emph{labeled} oracle $O^l_i(t)$: when
    agent $i$ calls the oracle at round $t$ the oracle $O^l_i(t)$ picks one
    of her neighbors $j$ uniformly at random and returns $(j, x_j(t-1))$,
    namely the label and the opinion of her neighbor $j$.

  \item The \emph{unlabeled} oracle $O^u_i(t)$: when
    agent $i$ calls the oracle $O^u_i(t)$ at round $t$ the oracle picks one
    of her neighbors $j$ uniformly at random and returns $x_j(t-1)$.
\end{itemize}
In the \emph{labeled} resp. \emph{unlabeled} setting, at the beginning of each
round $t$, each agent $i$ calls $O^l(i)$ resp. $O^u(i)$.
We motivate the study of the unlabeled setting by the fact that
in a huge social network, natural models would not take into account
the identities of the people that express an opinion. (CAREFUL OPINION LEADERS).
For the labeled setting we propose the following natural update rule.

In the unlabeled setting we consider the following update rule that closely
resembles the original FJ-model
\begin{align}
  x_i(0) &= s_i \nonumber \\
  x_i(t) &=
  (1-\alpha_i)\frac{\sum_{\tau=1}^{t} O^u_i(\tau)}{t} + \alpha_i s_i.
  \label{eq:unlabeled_update_rule}
\end{align}
In the labeled setting each agent $i$ keeps a vector $M_i \in [0,1]^{|N_i|}$
with the opinions of all her neighbors. At each round $t$ she receives the
value $(j, x_j(t-1))$ from $O_i^l(t)$ and sets $M_i(j) = x_j(t-1)$. Initially
all agents have $M_i = 0$. Her update rule is
\begin{align}
  x_i(0) &= s_i \nonumber \\
  x_i(t) &=  (1-a_i) \frac{\sum_{j\in N_i} M_i(j)}{|N_i|} + a_i s_i
  \label{eq:labeled_update_rule}
\end{align}

\subsection{Our Results}
We first show that in the unlabeled setting, using the update rule
(\ref{eq:unlabeled_update_rule}) gives a protocol that converges to the
same equilibrium point as the FJ-model. In the next theorem we provide
the rate of convergence of our model.
\begin{theorem}
  For any instance $I = (G, s, a)$ the update rule \ref{eq:unlabeled_update_rule}
  has the following convergence rate
  \[
    \Exp{\norm{\infty}{x^t - x^*}} \leq
    \begin{cases}
      C_1 \sqrt{\log n}\frac{(\log t)^{2}}{t^\rho}
      &\quad\text{if } \rho \leq 1/2\\
      C_2 \sqrt{\log n}\frac{(\log t)^{2}}{t^{1/2}}
      &\quad\text{if } \rho > 1/2\\
    \end{cases},
  \]
  where $\rho = \min_{i \in V} a_i$ and $C_1, C_2$ are universal constants.
\end{theorem}

For the labeled case we show that we even in our restricted setting we can still
achieve an almost linear rate of convergence as stated in the following theorem

\begin{theorem}
  For any instance $I = (G, s, a)$ the update rule \ref{eq:labeled_update_rule}
  has the following convergence rate
  \[
    \Exp{\norm{\infty}{x^t -x^*}} \leq
    2(1-a)^{\frac{\sqrt{t}}{4d\ln(dt)\ln\lp( \frac{1}{1-a}\rp)}}
  \]
\end{theorem}


% In this limited information environment, we propose simple and natural
% models that share similar convergence properties with the original FJ-model.


\section{Full memory with limited information exchange}
In each round, every agent learns a neighbour's opinion and updates the appropriate cell in memory. Because an agent learns only one opinion in each round, he uses outdated information about his other neighbours in order to compute his opinion. We first introduce some convinient notation.
\begin{definition}
$\pi_{ij}(t)$ is the last time that $i$ learned $j$'s opinion until time $t$.
\end{definition}

Obviously, if at time $t$ an agent $i$ learned $k$'s opinion, then $\pi_{ik}(t) = t$. The rule, according to which agent $i$ updates his opinion at time $t$ is the following(for simplicity, assume that $\alpha_i = \frac{1}{d+1}$ and that the graph is $d$-regular):$$x_{i}(t+1)=(1-\alpha_i)\frac{\sum_{j \in N_i}x_j(\pi_{ij}(t))}{d_i}+\alpha_i s_i$$
Notice that in contrast with the traditional model, instead of $x_j(t)$ we have $x_j(\pi_{ij}(t))$. That is because the information about $j$ is probably outdated.
Denote with $x^*$ the solution vector of the system. We would like to prove that such a process converges to the solution of the linear system involving the laplacian matrix of the graph. In order to analyse the algorithm, we will divide the time into "epochs". Each epoch has a length $D \ge d$, so the first epoch is from time 1 to time $D$, the second from $D+1$ to $2D$ e.t.c. Time $0$ does not belong at any epoch. The numbering of epochs begins from $1$. We will also assume that during each epoch, each agent picks every one of his neighbours for update at least once. Later we are going to find a suitable epoch length D, such that this holds with high probability. In other words, for a specific time in epoch $i$ every agent has information which has "bounded" outdateness, since every coordinate was updated at least once during the $i-1$ epoch. We are going to use this fact to prove the following lemma.

\begin{lemma}
Denote with $x^*$ the solution vector of the system. For every time $t$, which belongs to epoch $T$, it holds:
$$ ||x(t)-x^*||_{\infty} \leq \left( 1-\alpha_i\right)^T||x(0)-x^*||_{\infty}$$
\end{lemma}

\begin{proof}
We are going to use induction in time. The base case is for time $t=1$. We are going to prove that: $$ ||x(1)-x^*||_{\infty} \leq (1-\alpha_i)||x(0)-x^*||_{\infty}$$
We assume that everybody has the same initial vector $x(0)$ stored in memory, so for each $i$ holds:
\begin{equation}\label{eq:time1}
x_i(1) = (1-\alpha_i)\frac{\sum_{j \in N_i}x_j(0)}{d}+\alpha_i s_i 
\end{equation}
Since $x^*$ is the solution of the system, we have:
\begin{equation}\label{eq:sol}
x_i^* = (1-\alpha_i)\frac{\sum_{j \in N_i}x_j^*}{d}+\alpha_i s_i 
\end{equation}
Subtracting ~(\ref{eq:sol}) from ~(\ref{eq:time1}) we get:
\begin{eqnarray*}
|x_i(1) - x_i^*| &=& |(1-\alpha_i)\frac{\sum_{j \in N_i}(x_j(0)-x_j^*)}{d}|\\
&\leq& (1-\alpha_i)\frac{\sum_{j \in N_i}|x_j(0)-x_j^*|}{d} \\
&\leq& (1-\alpha_i) ||x(0)-x^*||_{\infty}
\end{eqnarray*}
$$ \Rightarrow ||x(1)-x^*||_{\infty} \leq (1-\alpha_i)||x(0)-x^*||_{\infty}$$
So the base case is verified. Let the inductive hypothesis hold for all times until $t$. Suppose that time $t+1$ belongs to epoch $T$. We fix an agent $i$. Then, if $\pi_{ij}(t)$ belongs to the previous epoch, by the induction hypothesis it holds:
$$ |x_j(\pi_{ij}(t)) - x_j^*| \leq \left( 1-\alpha_i\right)^{T-1}||x(0)-x^*||_{\infty}$$
 On the other hand, if $\pi_{ij}(t)$ belongs to the current epoch $T$, by the induction hypothesis we have:
 $$ |x_j(\pi_{ij}) - x_j^*| \leq \left( 1-\alpha_i\right)^{T}||x(0)-x^*||_{\infty} \leq \left( 1-\alpha_i\right)^{T-1}||x(0)-x^*||_{\infty}$$
 So, for every neighbour of agent $i$, the value that $i$ stores for this neighbour is "close" to the optimal. Now, using again equation $(1)$, we have:
 \begin{eqnarray*}
|x_i(t+1) - x_i^*| &=& |(1-\alpha_i)\frac{\sum_{j \in N_i}(x_j(\pi_{ij}(t))-x_j^*)}{d}|\\
&\leq& (1-\alpha_i)\frac{\sum_{j \in N_i}|x_j(\pi_{ij}(t))-x_j^*|}{d}\\
&\leq& (1-\alpha_i)\frac{\sum_{j \in N_i}||x(\pi_{ij}(t))-x^*||_\infty}{d}\\
&\leq& \frac{\sum_{j \in N_i}\left(1-\alpha_i\right)^{T-1}||x(0)-x^*||_{\infty}}{d+1}\\
&=& (1-\alpha_i)\left(1-\alpha_i\right)^{T-1}||x(0)-x^*||_{\infty}\\
&=& \left(1-\alpha_i\right)^{T}||x(0)-x^*||_{\infty}
 \end{eqnarray*}

\end{proof}

\begin{corollary}
If the algorithm runs for $O(D \log \frac{1}{\epsilon})$ iterations, it gets $\epsilon$-close to the solution $x^*$.
\end{corollary}

We now turn our attention to the problem of computing the appropriate length $D$ of epochs. A useful fact concerning the coupons collector problem is the following.

\begin{lemma}
Suppose that the collector picks $n\ln n + cn$ coupons, where $n$ is the number of distinct coupons. Then:
$$
\Prob{\text{collector hasn't seen all coupons}} \leq \frac{1}{\me^c}
$$
\end{lemma}

\begin{theorem}
After $t$ rounds, with probability at least $1 - p$: $\norm{\infty}{x^{t}-x^*} \leq (1-a)^{\frac{t}{2d\ln( \frac{dt}{p})} }$
\end{theorem}
\begin{proof}
In our setting, coupon $i$ corresponds to the selection of neighbour $i$. 
Each node is a collector and wants to gather all $d_i$ coupons during each epoch. 
Suppose $d = \max_i d_i$ is the maximum degree of the graph. Then, if we set $n = d$ and $c = \ln (\frac{dt}{p})$ , 
using the previous lemma we get that a node hasn't seen at least one neigbour after $cd + d\ln d$ samples with probability at most $\frac{p}{dt}$. 
This means that if we set $D = cd + d\ln d = d \ln \frac{dt}{p} + d \ln d \geq 2d\ln\frac{dt}{p} $ when $p$ is small enough, 
then the probability that a specific agent at a specific epoch hasn't collected all neighbouring opinions at least once is at most $\frac{p}{dt}$. 
By a simple union bound argument, we get that all agents have seen all their neighbours during all epochs with probability at least $1 - p$. 
We observe that at time $t$ approximately $\frac{t}{D}$ epochs have passed. 
Therefore, using the previous result about the convergence rate, we get that with probability at least $1 - p$:
$$ \norm{\infty}{x^{t}-x^*} \leq (1-a)^{\frac{t}{D}} \leq (1-a)^{\frac{t}{2d\ln( \frac{dt}{p})} }$$

\end{proof}
We are now going to translate this result to one that involves the expected value of the error.
\begin{theorem}If we set $u(t) = \norm{\infty}{x^t-x^*}$, the error after $t$ rounds, then:
$$
\Exp{u(t)} \leq 2(1-a)^{\frac{\sqrt{t}}{4d\ln(dt)\ln\lp( \frac{1}{1-a}\rp)}}
$$
\end{theorem}
\begin{proof}
Using the result of the previous theorem, we obtain:
$$ \Prob{u(t) > (1-a)^{\frac{t}{2d\ln( \frac{dt}{p})} }} \leq p $$
for every probability $p \in [0,1]$. Also, since all the parameters of the problem lie in $[0,1]$, we have
$$\Exp{u(t)|u(t) > r} \leq 1$$
Now, by the conditional expectations identity, we get:
\begin{align*}
\Exp{u(t)} &= \Exp{u(t)|u(t) > r}\Prob{u(t) > r} +\Exp{u(t)|u(t) \leq r}\Prob{u(t) \leq r}\\
&\leq p + r
\end{align*}
where $r = (1-a)^{\frac{t}{2d\ln( \frac{dt}{p})} }$. If we set $p = (1-a)^{\frac{\sqrt{t}}{2d\ln dt}}$, then:
$$
\Exp{u(t)} \leq (1-a)^{\frac{\sqrt{t}}{2d\ln dt}} + (1-a)^{\frac{t}{2d\ln( \frac{dt}{p})} }
$$
We now evaluate $r$ for our choice of probability $p$:
\begin{align*}
r
&= (1-a)^{\frac{t}{2d\ln\lp( \frac{dt}{p}\rp)} }\\
&= (1-a)^{\frac{t}{2d\ln\lp( \frac{dt}{(1-a)^{\frac{\sqrt{t}}{2d\ln dt}}}\rp)} }\\
&= (1-a)^{\frac{t}{2d\ln dt + 2d\frac{\sqrt{t}}{2d\ln dt}\ln\lp( \frac{1}{(1-a)}\rp) }}\\
&\leq (1-a)^{\frac{t}{4d\ln(dt)\ln\lp( \frac{1}{1-a}\rp) \sqrt{t}}}\\
&= (1-a)^{\frac{\sqrt{t}}{4d\ln(dt)\ln\lp( \frac{1}{1-a}\rp)}}
\end{align*}

Using the previous calculation, we obtain:
$$ \Exp{u(t)} \leq (1-a)^{\frac{\sqrt{t}}{2d\ln dt}} + (1-a)^{\frac{\sqrt{t}}{4d\ln(dt)\ln\lp( \frac{1}{1-a}\rp)}} \leq 2(1-a)^{\frac{\sqrt{t}}{4d\ln(dt)\ln\lp( \frac{1}{1-a}\rp)}}$$
\end{proof}
Therefore, this strategy achieves subexponential convergence rate in expectation.

\section{Constant Memory with full window}

In this work we investigate an variant of the above process.
We denote by $V$ the set of agents, and $N_{i} \subset V$ the set of neighbors
of agent $i$.
Let $x(t) \in [0,1]^n$ be the opinion vector at round $t$.
We denote by $x_i(t)$ the opinion of agent $i$ at round $t$.
At round $t+1$, $x(t+1)$ is constructed in the following way.
At first, each agent $i$ gets to see
the opinion $x_j(t)$, where $j \in N_i$ is picked uniformly at random
from the set of her neighbors.  Let $W_i^t$ be the random variable
corresponding to the selected neighbor. Now each agent $i \in V$
updates her opinion as follows
\[
  x_i(t+1) =
  (1-\alpha_i)\frac{\sum_{\tau=1}^{t+1} x_{W_i^\tau}(\tau-1)}{t+1}
  + \alpha_i s_i.
\]
We remark that each agent $i$ get to see \emph{only} the opinion
$x_{W_i^t}$ and not the label $W_i^t$ of her neighbor.

Let $x^* \in [0,1]^n$ be the unique equilibrium point
of the given instance $I$. We prove that
the above stochastic process has the following convergence
rate to $x^*$.
\begin{theorem}

\[
\Exp{\norm{\infty}{x^t - x^*}} =
    \begin{cases}
      \bigOh{\sqrt{\log n}\frac{(\log t)^{2}}{t^\alpha}}
      &\quad\text{if } \alpha \leq 1/2\\
      \bigOh{\sqrt{\log n}\frac{(\log t)^{2}}{t^{1/2}}}
      &\quad\text{if } \alpha > 1/2\\
    \end{cases}
  \]
\end{theorem}
\begin{proof}
  Setting $p = 1/\sqrt{t}$ in Lemma~\ref{l:error_bound} yields
  the result.
\end{proof}

We start by stating the standard Hoeffding bound

\begin{lemma}[Hoeffding's Inequality]\label{l:hoeffding}
  Let $X = (X_1 +\cdots + X_t) /t$, $X_i \in [0,1]$.
  Then,
  \[
    \Prob{|X - \Exp{X} | > \lambda} < 2 e^{-2 t \lambda^2}.
  \]
\end{lemma}

\begin{lemma}\label{l:error_recursion}
  With probability at least $1-p$, $\|x^t - x^*\|_{\infty} \leq e(t)$,
  where $e(t)$ satisfies the following recursive relation
  $$e(t) =
  \delta(t) + (1-\alpha)\frac{\sum_{\tau=0}^{t-1}e(\tau)}{t}
  \text{ and } e(0)=\|x^0 - x^*\|_{\infty}, $$
  where $\delta(t) = \sqrt{ \frac{\log(\pi^2 n t^2/(6 p))}{t}}$.
\end{lemma}
\begin{corollary}\label{c:error_recursion}
  The function $e(t)$ satisfies the following recursive relation
  $$e(t+1)-e(t) + \alpha \frac{e(t)}{t+1}=\delta(t+1)-\delta(t) + \frac{\delta(t)}{t+1}$$
\end{corollary}
\begin{proof}
  As we have already mentioned for any instance $I$ there exists
  a unique equilibrium vector $x^*$. Since
  $W_i^\tau \sim \Unif(N_i)$ we have that $
  \Exp{x^*_{W_i^\tau}} = \frac{\sum_{j \in N_i} x^*_j}{|N_i|}
  $.
  Since $W_i^\tau$ are independent random variables, we can use
  Hoeffding's inequality (Lemma~\ref{l:hoeffding}) to get
  \[
    \Prob
    {
      \lp|
      \frac{\sum_{\tau = 1}^{t} x^*_{W_i^\tau}}{t}
      - \frac{\sum_{j \in N_i} x^*_j}{|N_i|} \rp|
      > \delta(t)
    }
    < \frac{6}{\pi^2} \frac{p}{ n t^2},
  \]
  where $\delta(t) = \sqrt{ \frac{\log(\pi^2 n t^2/(6 p))}{t}}$.
  Therefore, by the union bound
  \begin{align*}
    &\Prob
    {
      \text{for all } t \geq 1 :
      \max_{i \in V}
      \lp|
      \frac{\sum_{\tau = 1}^{t} x^*_{W_i^{\tau}}}{t}
      - \frac{\sum_{j \in N_i} x^*_j}{|N_i|} \rp|
      > \delta(t)} \leq \\
    &\sum_{t =1 }^{\infty} \Prob
    {
      \max_{i \in V}
      \lp|
      \frac{\sum_{\tau = 1}^{t} x^*_{W_i^{\tau}}}{t}
      - \frac{\sum_{j \in N_i} x^*_j}{|N_i|} \rp|
      > \delta(t)} \leq
    \\
    &\sum_{i=1}^{\infty} \frac{6}{\pi^2} \frac{1}{t^2} \sum_{i=1}^n \frac{p}{n} =
    p
  \end{align*}

  As a result with probability $1-p$ we have that for all $t$
  \begin{equation}\label{eq:error_per_round}
    \lp|
    \frac{\sum_{\tau=1}^t x^*_{W_i^\tau}}{t} -
    \frac{\sum_{j \in N_i} x^*_j}{|N_i|}
    \rp| \leq \delta(t)
  \end{equation}
  We will our claim by induction.
  We assume that $\norm{\infty}{x^{\tau}-x^*} \leq e(\tau)$ for all
  $\tau \leq t-1$. Then
  \begin{align}
    x_i(t)
    &=
    (1-\alpha_i)\frac{\sum_{\tau=1}^{t}x_{W_i^{\tau}}(\tau-1)}{t}
    + \alpha_i s_i \nonumber \\
    &\leq
    (1-\alpha_i)\frac{\sum_{\tau=1}^{t}x^*_{W_i^{\tau}} +
      \sum_{\tau=1}^{t} e(\tau-1)}{t} + \alpha_i s_i \label{step:induction_step}\\
    &\leq
    (1-\alpha_i)
    \lp(
    \frac{\sum_{\tau=1}^{t}x^*_{W_i^{\tau}}}{t}+
    \frac{\sum_{\tau=0}^{t-1} e(\tau)}{t}
    \rp)
    + \alpha_i s_i \nonumber\\
    &\leq
    (1-\alpha_i)\lp(\frac{\sum_{j \in N_i}x^*_{j}}{|N_i|} +
    \delta(t) + \frac{\sum_{\tau=0}^{t-1} e(\tau)}{t} \rp) +
    \alpha_i s_i \label{step:error} \\
    &\leq
    x_i^* + \delta(t) + (1-\alpha)
    \lp(
    \frac{\sum_{\tau=0}^{t-1} e(\tau)}{t}
    \rp)
    \nonumber
  \end{align}
  We get (\ref{step:induction_step}) from the induction step and
  (\ref{step:error}) from inequality~(\ref{eq:error_per_round}).
  Similarly, we can prove that
  $x_i(t) \geq x_i^* - \delta(t) - (1-\alpha)
  \frac{\sum_{\tau=1}^t e(\tau)}{t}$.
  As a result $||x_i(t)-x^*||_{\infty} \leq e(t)$.
\end{proof}

In order to bound the convergence time of the system, we just need to bound
the convergence rate of the function $e(t)$.
The following lemma provides us with a simple upper bound for the
convergence rate of our process.

% \begin{lemma}\label{l:error_bound}
%   $$e(t) \leq \frac{e(0)}{t^\alpha} +
%   \frac{\bigOh{\sqrt{\log(\frac{nd}{p})}}}{t^\alpha}
%   \sum_{\tau=1}^t\tau^\alpha\frac{\sqrt{\log \tau}}{\tau^{3/2}}$$
% \end{lemma}

\begin{lemma}\label{l:error_bound}
  Let $e(t)$ be a function satisfying the recursion of
  Corollary~\ref{c:error_recursion}. Then with probability
  at least $1-p$ we have that
  \[
    e(t) =
    \begin{cases}
      \bigOh{\sqrt{\log(\frac{n}{p})}\frac{(\log t)^{3/2}}{t^\alpha}}
      &\quad\text{if } \alpha \leq 1/2\\
      \bigOh{\sqrt{\log(\frac{n}{p})}\frac{(\log t)^{3/2}}{t^{1/2}}}
      &\quad\text{if } \alpha > 1/2\\
    \end{cases}
  \]
\end{lemma}

\begin{proof}
  At first, since $\frac{\delta(t)}{t}$ is a decreasing function
  for $p \leq 1/4$, we have that
  $e(t)\leq (1-\frac{\alpha}{t}) + g(t)$, where $g(t)=\frac{\delta(t)}{t}$.
  \begin{align*}
    e(t)
    &\leq
    (1-\frac{\alpha}{t})e(t-1) + g(t)\\
    &\leq
    (1-\frac{\alpha}{t})(1-\frac{\alpha}{t-1})e(t-2)
    + (1-\frac{\alpha}{t})g(t-1) + g(t)\\
    &\leq
    (1-\frac{\alpha}{t})\cdots (1-\alpha)e(0)
    + \sum_{\tau=1}^t g(\tau)\prod_{i=\tau+1}^t(1-\frac{\alpha}{i})\\
    &\leq
    \frac{e(0)}{t^\alpha}
    + \sum_{\tau=1}^tg(\tau)e^{-\alpha\sum_{i=\tau+1}^t\frac{1}{i}}\\
    &\leq
    \frac{e(0)}{t^\alpha}
    + \sum_{\tau=1}^tg(\tau)e^{-\alpha(H_t-H_{\tau})}\\
    &\leq
    \frac{e(0)}{t^\alpha}
    + e^{-\alpha H_t}\sum_{\tau=1}^tg(\tau)e^{\alpha H_{\tau}}\\
    &\leq \frac{e(0)}{t^\alpha}
    + \frac{\bigOh{\sqrt{\log(\frac{n}{p})}}}{t^\alpha}
    \sum_{\tau=1}^t\tau^\alpha\frac{\sqrt{\log \tau}}{\tau^{3/2}}\\
  \end{align*}

  We observe that
  \[
    \sum_{\tau=1}^t\tau^\alpha
    \frac{\sqrt{\log \tau}}{\tau^{3/2}}
    \leq
    \int_{\tau=1}^t
    \tau^\alpha\frac{\sqrt{\log \tau}}{\tau^{3/2}}d\tau,
  \]
  since $\tau^\alpha\frac{\sqrt{\log \tau}}{\tau^{3/2}}$
  \begin{itemize}
    \item If
  $\alpha\leq 1/2$ then
  \[
    \int_{\tau=1}^t \tau^\alpha\frac{\sqrt{\log \tau}}{\tau^{3/2}}d\tau
    \leq
    \int_{\tau=1}^t
    \tau^{1/2}\frac{\sqrt{\log \tau}}{\tau^{3/2}}d\tau = O((\log t)^{3/2}))
  \]
\item If $\alpha> 1/2$ then\\
  \begin{align*}
    \int_{\tau=1}^t
    \tau^\alpha\frac{\sqrt{\log \tau}}{\tau^{3/2}}d\tau
    &=
    \int_{\tau=1}^t \tau^{\alpha-1/2}\frac{\sqrt{\log \tau}}{\tau}d\tau \\
    &=
    \frac{2}{3} \int_{\tau=1}^t
    \tau^{\alpha-1/2}((\log \tau)^{3/2})'d\tau \\
    &=
    \frac{2}{3}(\log t)^{3/2} - (\alpha-1/2)\frac{2}{3}
    \int_{\tau=1}^t \tau^{\alpha-3/2}(\log \tau)^{3/2}d\tau \\
    &=
    \bigOh{(\log t)^{3/2}}
  \end{align*}

\end{itemize}

\end{proof}

\section{Lower bound for unbiased estimators}

We first state a well known result from point estimation:
\begin{lemma}[Cramer-Rao bound]Let $\hat{\theta}$ be an estimator for the parameter $\theta$ of a distribution $P_\theta$, where $\theta$ is a continuous parameter. Suppose that the estimator is unbiased, that is:$E[\hat{\theta}] = 0$, for all distributions $P_\theta$. Under suitable regularity conditions, which are met by bernoulli distributions, it holds:$$Var[\hat{\theta}] \geq  \frac{1}{nI_\theta}$$, where $I_\theta$ is the Fischer information of distribution $P_\theta$.
\end{lemma}

In the case of Bernoulli random variables, we can easily show that for a Bernoulli with probability $\theta$, $I_\theta = \frac{1}{\theta(1-\theta)} $. Applying the above result we obtain $$ var[\hat{\theta}] \geq \frac{\theta(1-\theta) }{n}\geq \frac{1}{4n} $$ for every unbiased estimator $\hat{\theta}$. We can use this fact to show that a specific class of distributed protocols that solve our problem has "large" variance. Let $P$ be a protocol that when restricted to the star topology acts like an unbiased estimator for the weighted mean value of the neighbours, which can be an arbitrary real number in $[0,1]$. We will show that for all star topologies the variance of the solution is $\geq \frac{(1-a)^2}{4n}$ after $n$ samples. Suppose that for a specific star topology the preceding claim doesn't hold. Then, we notice that if $\hat{\theta}$ is the output of the protocol is this topology, then $\frac{(\hat{\theta}-as)}{1-a}$ is an unbiased estimator of the mean value of the neighbours, and this holds for every possible value of the mean. By our hypothesis, we have:
$$Var[\frac{\hat{\theta}-as}{1-a}] = \frac{1}{(1-a)^2}Var[\hat{\theta}] < \frac{1}{4n}$$, a contradiction, since our constructed estimator is unbiased. 

\bibliography{refs}

\end{document}
